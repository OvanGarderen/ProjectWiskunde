\documentclass[11pt]{amsart}

\usepackage[dutch]{babel}
\usepackage{a4wide}
%\setlength{\parindent}{0pt}

\newtheorem*{vraag}{Vraag}
\theoremstyle{definition}
\newtheorem*{uitwerking}{Uitwerking}

\newcommand{\R}{\mathbb{R}}
\newcommand{\N}{\mathbb{N}}
\newcommand{\Z}{\mathbb{Z}}
\newcommand{\C}{\mathbb{C}}
\newcommand{\A}{\mathbb{A}}
\newcommand{\Q}{\mathbb{Q}}
\newcommand{\F}{\mathbb{F}}
\newcommand{\f}{\varphi}
\newcommand{\e}{\varepsilon}
\renewcommand{\d}{\delta}

\begin{document}

\title{}
\author{Jan Westerdiep}
\maketitle

\[
f(x) = \sin( 2 \pi x )
\]
Stel we gaan $f(x)$ op vier punten $\{ 0, \frac{1}{4}, \frac{2}{4}, \frac{3}{4} \}$ discretiseren en hierop de DFT toepassen, dan vinden we
\[
\vec{x} = (0, 1, 0, -1) \Rightarrow X = (0, -\frac{1}{2}i, 0, \frac{i}{2}).
\]

Herinner je dat
\[
\sin( 2 \pi x ) = \frac{e^{i 2 \pi x} - e^{- i 2 \pi x}}{2i} = -\frac{1}{2}i e^{i 2 \pi x} + \frac{1}{2} i e^{- i 2 \pi x}
\]
\[
= 0 e^{} - \frac{1}{2}i e^{i 2 \pi x \cdot } + 0 e^{} + \frac{1}{2} i e^{} = X_1 e^{} + X_2 e^{} + X_3 e^{} + X_4 e^{}
\]

\end{document}
