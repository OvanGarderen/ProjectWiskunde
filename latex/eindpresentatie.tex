\documentclass{beamer}
\usepackage{beamerthemeAmsterdam}
\usepackage{amsmath}
\usepackage{graphicx}

\renewcommand{\phi}{\varphi}

\title{JPEG-2000}
\subtitle{De Wondere Wereld van Wavelets}
\author{Jan Westerdiep \and Okke van Garderen}
\date{\today}
\institute{Universiteit van Amsterdam}

\renewcommand{\itemize}[1]{\begin{itemize} #1 \end{itemize}}
\renewcommand{\figurename}{}

\begin{document}

\frame{\titlepage}

\section{Intro}

\frame{
  \frametitle{Recap}
  \begin{itemize}
  \item Project bij Rob Stevenson
  \item Originele JPEG $\implies$ Fouriertransformatie
  \item JPEG-2000 $\implies$ Wavelets
  \item Discreet signaal?
  \end{itemize}
  \begin{columns}
	\begin{column}{.3\textwidth}
	  \includegraphics[width=\textwidth]{plaatje.jpg}% picture filename
	\end{column}
	\begin{column}{.5\textwidth}
	  $\left[ \begin{array}{rrrrrr}
		250 & 237 & 215 & 209 & 218 & 238 \\
		237 & 194 & 138 & 109 & 136 & 195 \\
		217 & 133 &  45 &  15 &  46 & 135 \\
		204 & 103 &  14 &   0 &  14 & 105 \\
		215 & 128 &  36 &  11 &  38 & 128 \\
		233 & 186 & 120 &  91 & 120 & 185 \\
	  \end{array}\right]$
	\end{column}
\end{columns}
}

\frame{
  \frametitle{Fouriertransformatie}
  \[ \left\{ \phi_k(x) = e^{i k 2 \pi x}: k \in \mathbb{N}_0 \right\} \]
  \[ f = \sum \langle f, \phi_k \rangle \phi_k\text{, met $\phi_k$ genormaliseerd } \]
  
  \begin{itemize}
  \item Functie schrijven in de Fourierbasis
  \item Co\"effici\"enten kleiner dan $\epsilon$ ``weggooien"
  \item Signaal reconstrueren adhv kleinere set co\"effici\"enten
  \end{itemize}
}

\section{Wavelets}

\frame{
  \frametitle{Wavelettransformatie}
  \begin{columns}
    \begin{column}{0.4\linewidth}
      \includegraphics[width=\linewidth]{4wavelets.jpg}
    \end{column}
    \begin{column}{0.6\linewidth}
      \begin{itemize}
      \item Weer othonormale basis
\[ \left\{\psi_{i,j}(x) = \frac{1}{\sqrt{2^i}}\psi\left(\frac{x - 2^ij}{2^i}\right)\right\} \]
      \item Kleinere drager $\implies$ discontinu\"iteiten alleen locaal zichtbaar
      \item Vari\"eteit aan families en klassen
      \item Schalingsfunctie $\phi$ en Waveletfunctie $\psi$
      \item JPEG-2000
      \end{itemize}
    \end{column}
  \end{columns}
}

\frame{
  \frametitle{Discrete Wavelettransformatie}
  \begin{columns}
    \begin{column}{0.6\linewidth}
      \begin{itemize}
      \item Geneste rij ruimtes
      \item Recursieve relaties: 
		\[\phi(x) = \textstyle\sum_k h_k \phi(2x-k)\]
		\[\psi(x) = \textstyle\sum_k g_k \phi(2x-k)\]
      \item Op zoek naar $a_\lambda = \langle f, \phi_\lambda \rangle, d_\lambda = \langle f,\psi_\lambda\rangle$
      \end{itemize}
      Mallat:
      \begin{eqnarray*}
        a_{(j-1,n)} = \textstyle\sum_k h_{k-2n}a_{(j,k)} \\
        d_{(j-1,n)} = \textstyle\sum_k g_{k-2n}a_{(j,k)}
      \end{eqnarray*}
    \end{column}
    \begin{column}{0.4\linewidth}
	\centering{Haar:}\\
	\bigskip
      \includegraphics[width=\linewidth]{V_0V_1.png}
    \end{column}
  \end{columns}
}

\frame{
  \frametitle{Schema van Algoritme}
  \includegraphics[width=\linewidth]{mallat.png}
}

\frame{
  \frametitle{Voorbeeld met Haar}
  \begin{columns}
    \vspace{-20pt} 
    \begin{column}{0.5\linewidth}
      \vspace{-20pt}
      
      De Haarwavelet heeft filters:
      \begin{itemize}
      \item $h = \frac 1 2 \sqrt 2 \cdot (1,1)$
      \item $g = \frac 1 2 \sqrt 2 \cdot (-1,1)$
      \end{itemize}
\bigskip
      Signaal : $a_3 = (1,2,1,2,3,4,3,4)$ \\

      $a_2 = \frac 1 2 \sqrt 2 \cdot (3,3,7,7)$\\
      $d_2 = \frac 1 2 \sqrt 2 \cdot (-1,-1,-1,-1)$\\

      $a_1 = (3,7)$\\
      $d_1 = (0,0)$\\

      $a_0 = \frac12\sqrt2\cdot (10)$\\
      $d_0 = \frac12\sqrt2\cdot (-4)$
    \end{column}
    \begin{column}{0.6\linewidth}
      \centering
      \begin{columns}
        \begin{column}{0.4\linewidth}
          \includegraphics[width=\linewidth]{HaarWavelet_Psi.jpg}
        \end{column}
        \begin{column}{0.4\linewidth}
          \includegraphics[width=\linewidth]{HaarWavelet_Phi.jpg}
        \end{column}
      \end{columns}
      \begin{eqnarray*}
        a_{(j-1,n)} = \textstyle\sum_k h_{k-2n}a_{(j,k)} \\
        d_{(j-1,n)} = \textstyle\sum_k g_{k-2n}a_{(j,k)}
      \end{eqnarray*}
      
  Uiteindelijk getransformeerde wordt:
  $\frac12\sqrt2\cdot(10,-4,0,0,-1,-1,-1,-1)$

    \end{column}
  \end{columns}
}

\frame{
  \frametitle{Daubechies 4 (ook wel Daubechies 2)}
  \begin{columns}
    \begin{column}{0.5\linewidth}
      \begin{itemize}
        \item Langere filter
        \item[$\implies$] slechter met discontinu\"iteiten
        \item[$\implies$] Gaat beter om met locale gladheid
        \item Subjectief: minder hoekig beeld
      \end{itemize}
    \end{column}
    \begin{column}{0.5\linewidth}
      \includegraphics[width=0.8\linewidth]{daubechies.png}
    \end{column}
  \end{columns}
}

\section{Plaatjes}

\frame{
  \frametitle{2-dimensionaal signaal (Plaatjes)}
\begin{columns}
  \begin{column}{0.6\linewidth}
    \begin{itemize}
    \item E\'en approximatiekwadrant en drie detailkwadranten
    \item Tweedimensionale functies maken:
      \[\begin{array}{c|c}
        \phi \otimes \phi & \psi \otimes \phi \\\hline
        \phi \otimes \psi & \psi \otimes \psi
      \end{array}\]
    \item Generalisatie naar meer dimensies
    \end{itemize}
  \end{column}
  \begin{column}{0.5\linewidth}
    \includegraphics[width=\linewidth]{mallat-schema.png}
  \end{column}
\end{columns}
}

\frame{
  \frametitle{Voorbeelden (20\%)}
  \centering
  \begin{table}
    \begin{tabular}{ c c c c}
      Origineel &
      \includegraphics[width=0.3\linewidth]{gentoo.png} &
      \includegraphics[width=0.3\linewidth]{gentoo_new_20p.png} &
      Fourier \\
      Haar &
      \includegraphics[width=0.3\linewidth]{gentoo_20_haar.png} &
      \includegraphics[width=0.3\linewidth]{gentoo_20_db2.png} &
      Daubechies 2
    \end{tabular}
  \end{table}
}

\frame{
  \frametitle{Voorbeelden (8\%)}
  \begin{table}
    \begin{tabular}{ c c c c}
      Origineel &
      \includegraphics[width=0.3\linewidth]{gentoo.png} &
      \includegraphics[width=0.3\linewidth]{gentoo_new_8p.png} &
      Fourier \\
      Haar &
      \includegraphics[width=0.3\linewidth]{gentoo_08_haar.png} &
      \includegraphics[width=0.3\linewidth]{gentoo_08_db2.png} &
      Daubechies 2
    \end{tabular}
  \end{table}
}

\frame{
  \frametitle{Voorbeelden (1\%)}
  \begin{table}
    \begin{tabular}{ c c c c}
      Origineel &
      \includegraphics[width=0.3\linewidth]{gentoo.png} &
      \includegraphics[width=0.3\linewidth]{gentoo_new_1p.png} &
      Fourier \\
      Haar &
      \includegraphics[width=0.3\linewidth]{gentoo_01_haar_smooth.png} &
      \includegraphics[width=0.3\linewidth]{gentoo_01_db2_smooth.png} &
      Daubechies 2
    \end{tabular}
  \end{table}
}

\section{Filmpjes}

\frame{
  \frametitle{Filmpjes $\implies$ 3D signaal}
  \begin{itemize}
  \item Natuurlijke voortzetting van Mallatdecompositie
  \item Opeenvolgende frames opvatten als 3e dimensie
  \end{itemize}
}

\frame{
  \frametitle{Maar toen...}
  \begin{itemize}
  \item Vermoeden van Rob over filmpjes
  \item Tensorproduct i.p.v. Mallatdecompositie
  \item Potentieel beter bij weinig discontinu\"iteiten
  \end{itemize}
  \centering
  \includegraphics[width=0.5\linewidth]{mallat-vs-tensor.png} 
}

\section{Wat nu?}

\frame{
  \frametitle{Inhoudsopgave}
  \begin{itemize}

  \item Deel over Fourier

    \begin{itemize}
    \item[-] Analyse van Discrete Fourier Transform
    \item[-] Fast Fourier Transform-algoritme
    \item[-] Toepassing van FFT op plaatjes/geluid
    \end{itemize}

  \item Deel over Wavelets

    \begin{itemize}
    \item[-] Theoretische beschouwing van wavelets
    \item[-] Toepassing van Wavelets $\implies$ DWT
    \item[-] Analyse van de DWT
    \item[-] Toepassing van DWT op plaatjes/geluid
    \item[-] Tensorproduct en filmpjes
    \end{itemize}

  \item Vergelijkend deel Wavelets vs. Fourier en Resultaten
  \end{itemize}
}

\frame{
  \frametitle{Wat nog moet gebeuren}
  \begin{itemize}
  \item Kijken naar convergentie-eigenschappen van de fout (hoe goed is de benadering theoretisch)
  \item Afronden van wat we nu hebben
  \item Eventueel nog iets interessants onderzoeken
  \item Code opschonen
  \end{itemize}
}

\frame{
  \frametitle{Preciezere beschrijving}
   Recursieve relatie door geneste ruimtes $V_0 \subset V_1 \subset \ldots \subset L_2$:
   \bigskip

   $V_i := span\{\phi_{i,j} : j \in \mathbb{Z}\},\quad V_{i+1} \supset W_i \perp V_i$.
   %$ = span\{\psi_{i,j}:j \in \mathbb{Z}\}$

   $V_{i+1} = V_i \oplus W_i \implies V_n = V_0 \oplus W_0 \oplus \ldots \oplus W_{n-1}.$\\
   Construeer $\psi_{i,j}$ zo dat $W_i = span\{\psi_{i,j}\}$:
   \bigskip

   $\phi_{i,j}$ te schrijven zijn in termen van $\phi_{i+1,k}$ want $V_i \subset V_{i+1}$.\\
   $\psi_{i,j}$ ook want $W_i \subset V_{i+1}$.\\
   %Ook $\psi_i$ zijn een basis voor $V_n$
}

\end{document}
