\documentclass[11pt]{amsart}

\usepackage[dutch]{babel}
\usepackage{a4wide}
\usepackage{algpseudocode}
\usepackage{algorithmicx}
%\setlength{\parindent}{0pt}

\newtheorem*{vraag}{Vraag}
\theoremstyle{definition}
\newtheorem*{uitwerking}{Uitwerking}

\newcommand{\R}{\mathbb{R}}
\newcommand{\N}{\mathbb{N}}
\newcommand{\Z}{\mathbb{Z}}
\newcommand{\C}{\mathbb{C}}
\newcommand{\A}{\mathbb{A}}
\newcommand{\Q}{\mathbb{Q}}
\newcommand{\F}{\mathbb{F}}
\newcommand{\f}{\varphi}
\newcommand{\e}{\varepsilon}
\renewcommand{\d}{\delta}

\begin{document}

\section{Pseudocode FFT}
\begin{algorithmic}
\Function{FFT}{$x$}
\State $n \gets \text{lengte}(x)$ \Comment Assumptie: $n = 2^m$ voor een $m$, oftewel $n$ is een tweemacht
\If {$n == 1$}
	\State{$X \gets x$}
\Else
	\State $E \gets FFT(x[0::2])$ \Comment{Pak alle even indices}
	\State $O \gets FFT(x[1::2])$ \Comment{Pak alle oneven indices}
	\For{$i = 0$ to $n-1$}
		\If{$i < n/2$}
			\State $X[i] \gets E[i] + e^{-2i \pi k/n} \cdot O[i]$
		\Else
			\State $X[i] \gets E[i] - e^{-2i \pi k/n} \cdot O[i]$
		\EndIf
	\EndFor
\EndIf
\State \Return{$X$}
\EndFunction
\end{algorithmic}


\section{Bewijs van tijdscomplexiteit FFT}
We willen bewijzen dat we de Fouriertransformatie van een of andere discrete functie kunnen vinden in $\mathcal{O}(n \log n)$ tijd, waar $n$ de lengte van de invoer is.

Een college in complexiteitstheorie geeft ons de volgende twee definities. 

\[
f \in \theta(g) \Leftrightarrow \exists c \in \R_+: \lim_{n \to \infty} \frac{f(n)}{g(n)} = c
\]

Een \emph{recurrente betrekking} $T(n)$ is volgt te omschrijven:
\[
T(n) = \begin{cases}
  c &\text{ als } n \leq d \\
  a T(n/b) + f(n) &\text{ anders,} \\
\end{cases}
\]
en geeft ``het aantal stappen" van een bepaald type algoritme weer in formulevorm.

De (vereenvoudigde) stelling van Akra-Bazzi levert ons daarna het volgende resultaat:
\[
\exists k \in \N \text{ zodat }f(n) \in \theta(n^{\log a/\log b} \log^k n) \text{ dan } T(n) \in \theta(n^{\log a / \log b} \log^{k+1}n).
\]

Kijkende naar de opbouw van de algoritme, zien we dat $a=b=2$ en $c=d=1$. Verder is $f(n)$ te omschrijven als ``het extra werk wat verricht moet worden om een tussenantwoord te krijgen", in andere woorden: het uitrekenen van de sommatie. Dit is in ons geval lineair in de lengte van de invoer, dus $k=0$. Dit betekent dat $T(n) \in \theta(n \log n)$. Dus het aantal stappen (tijdscomplexiteit) van de FFT is $\theta(n \log n)$ dus zeker $\mathcal{O}(n \log n)$.

\end{document}
