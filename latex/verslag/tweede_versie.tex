%We moeten even beslissen welke van de twee we willen gebruiken
\documentclass[11pt]{uvamath}
%\documentclass[11pt]{amsart}
\usepackage{graphicx}
\usepackage[pdfborder={0 0 0}]{hyperref}

\usepackage{amssymb}
\usepackage{amsmath}
\usepackage{amsthm}

\usepackage{parcolumns}

\usepackage{caption}
\usepackage{subcaption}
\usepackage{geometry}
\usepackage{lipsum}

% PYTHON CODE
\usepackage{listings}
\usepackage{color}

\DeclareCaptionFont{black}{ \color{black} }
\DeclareCaptionFormat{listing}{
  \parbox{\textwidth}{\hfill#3}
}
\captionsetup[lstlisting]{ format=listing, textfont=black, singlelinecheck=false, margin=0pt, font={footnotesize} }

\definecolor{dkgreen}{rgb}{0,0.6,0}
\definecolor{gray}{rgb}{0.5,0.5,0.5}
\definecolor{mauve}{rgb}{0.58,0,0.82}

\lstset{
  language=Python,
  aboveskip=3mm,
  belowskip=3mm,
  frame=single,
  showstringspaces=false,
  columns=flexible,
  basicstyle={\small\ttfamily},
  numbers=none,
  numberstyle=\tiny\color{gray},
  keywordstyle=\color{blue},
  commentstyle=\color{dkgreen},
  stringstyle=\color{mauve},
  breaklines=true,
  breakatwhitespace=true,
  captionpos=b,
  tabsize=3
}
%PYTHON CODE

\usepackage[dutch]{babel}
\usepackage{a4wide}
\usepackage{algpseudocode}
\usepackage{algorithmicx}
\usepackage[square,super]{natbib}

\makeatletter
%replaces : \def\@endtheorem{\endtrivlist\@endpefalse }
% with:
%\def\@endtheorem{\endtrivlist}
\makeatother

\newcommand{\R}{\mathbb{R}}
\newcommand{\N}{\mathbb{N}}
\newcommand{\Z}{\mathbb{Z}}
\newcommand{\C}{\mathbb{C}}
\newcommand{\A}{\mathbb{A}}
\newcommand{\Q}{\mathbb{Q}}
\newcommand{\F}{\mathbb{F}}
\newcommand{\e}{\epsilon}
\renewcommand{\O}{\mathcal{O}}

\newcommand{\FFT}{\text{FFT}}

\theoremstyle{plain}                    
\newtheorem{stelling}{Stelling}[chapter]
\newtheorem{lemm}[stelling]{Lemma}     
\newtheorem*{algo}{Algoritme}      

\theoremstyle{definition}
\newtheorem{definitie}[stelling]{Definitie}  

\theoremstyle{remark}
\newtheorem{gevolg}{Gevolg}[stelling]
\newtheorem*{opmerk}{Opmerking}
\newtheorem*{voorbeeld}{Voorbeeld}

\newcommand{\eq}[1]{\begin{eqnarray*} #1 \end{eqnarray*}}
\newcommand{\mogelijkheden}[1]{\begin{cases} #1 \end{cases}}
\newcommand{\repr}[1]{{#1}^{\!\!-1}}

\newcommand{\coefficient}{co\"effici\"ent}
\newcommand{\coefficienten}{co\"effici\"enten}

\newcommand{\dx}{\text{d}x}
\newcommand{\dy}{\text{d}y}
\newcommand{\dz}{\text{d}z}
\newcommand{\largediv}{\,\big|\,}
\newcommand{\Ldnorm}[1]{{||#1||_{L_2}}}
\newcommand{\inpr}[2]{\langle #1 , #2 \rangle}
\newcommand{\DFT}{\text{DFT}}
\newcommand{\dpii}{{2\pi i}}
\newcommand{\abso}[1]{{\left| #1 \right|}}
\renewcommand{\d}[1]{{\mathrm{d} #1}}

\setlength\parindent{0pt}
\parskip = \baselineskip
\setcounter{tocdepth}{2}
\setcounter{secnumdepth}{1}

\title{JPEG-2000: de wondere wereld van Wavelets}
\date{\today}
\author[10219242, janner@gmail.com]{Jan Westerdiep}
\author[10191429, ovangarderen@gmail.com]{Ogier van Garderen}
\what{Projectverslag jaar 2}
\supervisors{Rob Stevenson}
%\secondgrader{dr.\ Ben Moonen}
\coverimage{
	\includegraphics[width=0.6\linewidth]{plaatjes/voorkant.png}
}

%\includeonly{chap_implementatie, chap_wavelet, chap_reflectie_en_discussie}
\begin{document}
\maketitle

\begin{abstract}
\lipsum[2-3]

Op het voorblad is te zien, vanaf linksboven en dan met de klok mee:
\begin{itemize}
	\item Het originele UvA logo-plaatje;
	\item Compressie met een Haarwavelet waarbij 1\% van de data opgeslagen is;
	\item Compressie met een Daubechies-2 wavelet met 1\% van de data;
	\item Compressie met een Fouriertransformatie met 1\% van de data.
\end{itemize}
\end{abstract}

\tableofcontents
\newpage
\chapter{Basisbegrippen}

\section{Implementatie}
\subsection{Signaaluitbreiding}
\label{signaal}
Beide algoritmes kunnen enkel omgaan met signalen die een tweemacht lang zijn. Om te zorgen dat een willekeurig signaal ook getransformeerd kan worden, moet het dus uitgebreid worden voorbij zijn definitiegebied. De meeste bronnen onderscheiden de volgende manieren om het signaal uit te breiden. Laat $x_1, x_2, \ldots x_n$ het signaal.
\begin{description}
\item[Zero-padding] $x' = 0, \ldots, 0| x_1, x_2, \ldots, x_n| 0, \ldots, 0$. Het gevolg van deze uitbreiding is dat het signaal niet langer continu hoeft te zijn op de randen;
\item[Constant padding] $x' = x_1, \ldots, x_1| x_1, x_2, \ldots, x_n| x_n, \ldots, x_n$. Het gevolg is nu dat het signaal geen continue afgeleide meer hoeft te hebben;
\item[Symmetric padding] $x' = x_n, \ldots, x_1| x_1, x_2, \ldots, x_n| x_n, \ldots, x_1$. Het gevolg is nu dat het signaal geen continue afgeleide meer hoeft te hebben;
\item[Periodic padding] $x' = x_1, \ldots, x_n| x_1, x_2, \ldots, x_n| x_1, \ldots, x_n$. Het nieuwe signaal hoeft wederom niet continu te worden.
\end{description}
Wij hebben ervoor gekozen om signalen uit te breiden door middel van zero-padding.

\subsection{Gebruik van Python}
Voor de implementatie van de Fourier-transformatie is de taal Python gebruikt,
enkele redenen hiervoor waren:
\begin{itemize}
\item Voorzieningen om beeld en geluid te laden m.b.v. het \emph{scipy} pakket.
  Dit pakket liet ons bijvoorbeeld toe om beelden te laden als matrix van kleurwaarden en deze weer op te slaan
  nadat we deze gereconstrueerd hadden. Zo hoefden we ons niet te bekommeren om andere beeldcompressie implementaties.
\item Duidelijke relatie van wiskunde naar code. Python heeft een syntax die een wiskundige manier van denken en werken
  ondersteunt, in tegenstelling tot meer declaratieve talen zoals \emph{C} of object-georienteerde talen zoals \emph{Java}.
\item Goede abstractie van onderliggende processen. Python is een geinterpreteerde taal, dit betekent in concreto
  dat onze programma's draaien op een platform dat de allocatie van geheugen en rekenkracht. Dit vergemakkelijkt
  de implementatie en zorgt dat er makkelijk veranderingen zijn aan tee brengen in de code zonder het programma te crashen.
\end{itemize}
Het enige nadeel dat deze keuze teweeg heeft gebracht is dat de snelheid van al onze algoritmes tegenvalt
omdat we ons niet bezighouden met kleine optimalisaties en vertrouwen op de Python \emph{interpreter}.

\section{Wiskunde}

\subsection{Notatie}
De blokhaaknotatie $f[x]\in K$ duidt op een discrete functie $A\subset \Z\to K$ met $K$ een willekeurige ruimte.
Als $A$ bovendien eindig is, dan is $f$ ook te karakteriseren aan de hand van zijn beeld, een vector in $K^{N}$.
In het bijzonder is deze notatie dus een manier om een vector aan te duiden.
Evenzo kunnen we de notatie uitbreiden naar meer dimensies door te schrijven
\begin{equation*}
\begin{split}
f: A_1\times \cdots \times A_n\subset \Z^n \to& K \\
       (x_1,\ldots,x_n) \mapsto& f[x_1,\ldots,x_n],
\end{split}
\end{equation*}
waarbij de $A_i$'s meestal eindige intervallen zijn. 
Uit zo'n $m$-dimensionale functie $f$ kunnen we vervolgens op een natuurlijke manier een $m-k$-dimensionale
functie contrueren door $k$ co\"ordinaten vast te nemen:
\begin{equation*}
\begin{split}
f\largediv_{x_{i_1}=a_{i_1},\ldots,x_{i_k}=a_{i_k}}: A_{i_{k+1}}\times\cdots\times A_{i_{m}} \to& K \\
(a_{i_{k+1}},\ldots,a_{i_{m}}) \mapsto& f[a_1,\ldots,a_m]
\end{split}
\end{equation*}
Hier zijn $i_1$'de tot en met de $i_k$'de co\"ordinaat vast genomen. Voor een matrix kan deze functie
bijvoorbeeld gezien worden als een rij of een kolom, in dat voorbeeld geeft $i_1$ aan of de rij of kolom
vast staat en geeft $a_{i_1}$ aan welke rij respectievelijk kolom bekeken wordt.

\subsection{Complexiteit}

Aangezien er algoritmes behandeld worden in het verslag willen we hiervan de tijdscomplexiteit bepalen.
Deze eigenschap bepaalt namelijk hoe de tijd die het een machine kost oploopt met de grootte van de input.
We voeren daarom zowel \emph{Big O} als \emph{Little O} en \emph{Theta} notatie in.
\begin{equation*}
\begin{array}{cccccccc}
  f \in \O(g)     &\Leftrightarrow& \exists c     \,&\exists x_0  \,:&\, 
  \forall x>x_0 \quad&& |f(x)| &\leq c|g(x)|  \\
  f \in o(g)      &\Leftrightarrow& \forall c     \,&\exists x_0  \,:&\, 
  \forall x>x_0 \quad&& |f(x)| &\leq c |g(x)| \\ 
  f \in \theta(g) &\Leftrightarrow& \exists c_1,c_2\,&\exists x_0  \,:&\, 
  \forall x>x_0 \quad& c_1|g(x)| \leq & |f(x)| &\leq c_2|g(x)| 
\end{array}
\end{equation*}
Merk op dat geldt $f\in o(g) \implies f\in \O(g)$ en $f\in \theta(g) \implies f\in \O(g)$.
We zullen in de praktijk algoritmes $f$ bekijken die een 1,2 of 3 dimensionale lijst data als input hebben.
We zijn dan ge\"interesseerd in de tijdscomplexiteit in termen van de grootte van bijvoorbeeld een $N\times N$ 
afbeelding, dus is $f$ een functie van $N$.

\subsection{Ruimtes}
Een belangrijk 
\begin{lemm}[Parsevalgelijkheid\cite{parseval}]
  \label{parseval}
  Laat $f$ een functie in $L^2(K)$ met $K \subset \C^n$ of $K\subset\R^n$, die geschreven kan worden in een aftelbare 
  orthonormale basis $\mathcal{B}=\{g_m\}$ dan geldt
  \[
  \|f\|^2 = \sum_{g_m\in\mathcal{B}} | \langle f, g_m \rangle |^2.
  \]
\end{lemm}



\chapter{Fourier}

Bij het analyseren van periodieke functies is de Fourier-transformatie het gereedschap bij uitstek.
Het is een manier om een signaal te karakteriseren aan de hand van een bereik aan frequenties.
Dit maakt het een overtuigende manier om `nette' periodieke functies te beschrijven.
We kunnen de eis van periodiciteit ook loslaten als we naar functies op een interval kijken door dit op te vatten
als \'e\'en fase van een periodieke functie.

Iets concreter kijken we naar functies $f\in L_2([a,b])$.


De basisfuncties waarmee we vervolgens de analyse uitvoeren worden gegeven door de complexe $e$-machten.
\begin{definitie}[Fourierbasis] Bekijk de functieruimte $L_2([a,b])$, we defini\"eren de verzameling $F_{a,b}$ 
door:
\[
  F_{a,b} := \left\{ \phi_k(x) = \tfrac{1}{\sqrt{b-a}} e^{\dpii \cdot k \frac{x-a}{b-a}} \largediv k \in \Z \right\}
\]
We noemen $F_{a,b}$ de Fourierbasis van $L_2([a,b])$
\end{definitie}
Het woord basis is hier met recht gebruikt, het is immers bekend dat de complexe $e$-machten loodrecht staan onder
het inproduct dat we gedefinieerd hebben. Om nu een willekeurige functie te schrijven in deze basis, introduceren
we de Fouriergetransformeerde.
\begin{definitie}[Fouriergetransformeerde]
Zij gegeven een functie $f\in L_2([a,b])$, schrijf dan $\hat f : \Z\to\C$ met entries gedefinieerd volgens:
\[
  \hat f [n] = \frac{1}{\sqrt{b-a}} \cdot \inpr{f}{\phi_n} = \frac{1}{b-a} \int_a^b f(x) \cdot e^{-2 \pi i \cdot k \frac{x-a}{b-a}}\d{x}.
\]
We noemen $\hat f$ de Fouriergetransformeerde van $f$.
\end{definitie}
\begin{definitie}[Inverse Fouriertransformatie]
  Gegeven een Fouriergetransformeerde $\hat f$ van een functie $f \in L_2([a,b])$ is de reconstructie $f^\circ$ van $f$ gegeven door:
  \[
    f^\circ (x) = \sum_{k=-\infty}^\infty \hat f [k] \phi_k(x)
  \]
\end{definitie}

Het kan worden aangetoond dat $f^\circ$ een reconstructie geeft van $f$ die voldoet aan: \cite{fourier-rec} 
\begin{eqnarray}
  \Ldnorm{f-f^\circ}{a,b}=0 \quad\quad \text{wanneer }& f\in L_2([a,b]) \\
  f(x) = f^\circ(x) \quad\quad \text{wanneer }& f \in C^1 
\end{eqnarray}

%-----------------------------------------------------------------------------------------------------------------
\section{De \emph{Discrete Fourier Transform}}
Zoals we gezien hebben in het vorige hoofdstuk kan de Fouriertransformatie gebruikt worden om continue signalen te karakteriseren voor verschillende frequenties. Een groot gebied binnen de signaalanalyse is echter van discrete aard aangezien hier veelal digitale instrumenten worden gebruikt. De toepassingen waarnaar wij op zoek zijn liggen in dit digitale domein (JPEG is een beeldcompressie-algoritme) en dus zal het verslag zich verder afspelen in deze discrete setting.

Om discrete signalen te analyseren lijkt het voor de hand te liggen om een deze als stapfuncties te zien. Een stapfunctie zit immers in $L_2$. De discontinuiteiten van de stapfunctie leiden echter tot ongewenste resultaten, zoals is te zien aan de reconstructie van een blokgolf door middel van Fouriertransformatie. TODO MISCHIEN PLAATJE?
Erger nog, elke eindige som van continue functies is weer continu dus voor een perfecte reconstructie van een discreet signaal is met deze methode altijd een oneindige rij \coefficient en nodig. Bovendien is het moeilijk om uit deze \coefficient en relevante informatie te destilleren over de aard van het signaal.

In plaats van de discrete signalen in te bedden in $L_2$ zullen we een discreet analogon gebruiken voor de Fouriertransformatie.
Hiervoor zullen we de Fourier-basis discretiseren en ons richten op de ruimte $\R^n$.
We veranderen daarvoor de co\"ordinaten naar een discrete $j$ volgens
\[
\frac{x-a}{b-a} \leftrightarrow \frac j n 
\]
Op deze manier zal de discretisatie in de limiet naar het continue geval overgaan. 

\begin{definitie}[Discrete Fourierbasis] Gegeven de ruimte $\C^n$, definieer dan de verzameling
\[
  S_n := \left\{ s_k [j] = e^{\dpii\cdot k j/n } \largediv k,j \in \{1, \ldots, n\} \right\},
\]
als de \emph{discrete Fourierbasis} op deze ruimte met basisvectoren $s_k$.
\end{definitie}
We weten dat de basisvectoren loodrecht staan vanwege de eigenschap:
\[
  \inpr{s_k}{s_j} = 
  \sum_{m=1}^n s_k[m]\cdot \overline{s_j[m]} = 
  \sum_{m=1}^n e^{\dpii\cdot m (k-j)/n} =
  \begin{cases}
    0 \quad \text{als } k\neq j\\
    n \quad \text{als } k = j
  \end{cases}
\]
Vanwege deze eigenschap kunnen we een discreet signaal $x$ schrijven in termen van deze basis door 
een vector van inproducten te defini\"eren.
we noemen $X$ de discrete Fouriergetransformeerde (DFT) met entries:
\[
  X[k] = \tfrac{1}{n} \inpr{x}{s_k} \quad k\in\{1, \ldots ,n\}
\]
Vervolgens hebben we een inverse voor deze operatie die $X$ afbeeldt op $x^\circ$ volgens:
\[
  x[j] = \inpr{X}{s_j^{-1}} \quad j\in \{1, \ldots, n\}
\]
Waarbij we de notatie $s_j^{-1} = (s_j[1]^{-1},...)$ gebruiken.
Volgens de regels van de Lineaire Algebra is dit is een perfecte reconstructie ($x[k] = x^\circ[k]$).

Om de claim te ondersteunen dat de DFT-methode een echte discretisatie is van de continue Fouriertransformatie 
willen we bewijzen dat dit algoritme voor een steeds fijnere selectie van waardes van een functie in de 
limiet hetzelfde resultaat geeft als de continue Fouriertransformatie. 
Zoals gebruikelijk bij het overschakelen van een discrete naar een continue setting, kunnen we dit doen 
door de definitie van de Riemann integraal toe te passen op de sommatie die voor handen ligt.

\begin{stelling}[Limiet van discrete Fourier-transformatie]
  Gegeven een functie $f\in L_2([a,b])$ op het interval $[a,b]$,
  bekijk een discretisatie van $f$ in $n$ gelijke intervallen zodat de discrete $f$ precies 
  de randwaarde waarde van elk interval inneemt.
  Dan geldt dat de discrete Fouriergetransformeerde limiteert naar de algemene Fouriergetransformeerde wanneer $n\to\infty$.
\end{stelling} 
\begin{proof}[Bewijs]
Gegeven een interval $[a,b]$ kunnen we een partitie $P$ maken in $n$ gelijke delen, 
ofwel laat $P=\{a=t_0,t_1,..,t_n=b\}$ met $t_j = a+\tfrac{j(b-a)}{n}$.
We discretiseren onze functie $f$ door uit elk interval $[t_{j-1},t_{j}]$ van de partitie de randwaarde in 
$x_j = t_j$ te selecteren, dus
\[
f[j] = f(x_j) = f(\frac{b-a}{n}j + a)
\]
De DFT van de discrete functie $f[\cdot]$ wordt dan gegeven door:
\[
F[k] = \frac1n\sum_{j=1}^n f[j] \cdot s_k^{-1}[j].
\]
We schrijven dit om in termen van onze non-discrete functie door $x_j\in[a,b]$ 
omschrijven naar zijn discrete tegenhanger en krijgen zo:
\begin{eqnarray*}
  F[k] =& \frac{1}{n} \sum_{j=1}^n f[j]\cdot s_k^{-1}[n\cdot\tfrac{x_j-a}{b-a}] \\
       =& \frac{1}{n} \sum_{j=1}^n f(x_j)\cdot e^{-\dpii \cdot k \tfrac{x_j-a}{b-a}}\\
       =&  \frac{1}{\sqrt{b-a}}\sum_{j=1}^n f(x_j)\cdot \phi^*_k(x_j) \cdot\frac{b-a}{n} 
\end{eqnarray*}
We merken op dat de term $\frac{b-a}{n}$ precies de grootte is van de subintervallen van de partitie interval 
en dat we het geheel omgeschreven hebben in termen van onze continue functies $f$ en $\phi_k$.
Omdat het product $f\cdot\phi_k$ integreerbaar is moet voor elke partitie $P$ met 
waarden in punten $x_j$ uit elk interval deze sommatie convergeren naar de integraal
\[
  \frac{1}{\sqrt{b-a}} \int_{[a,b]} f(x) \phi^*_k(x) \d{x}
\]
wanneer we de maaswijdte ($\tfrac{b-a}{n}$) naar $0$ laten gaan. 
Dit is duidelijk het geval wanneer we de limiet $n\to\infty$ nemen. 
Dus is de DFT een goede discretisatie van de Fouriergetransformeerde. 
\end{proof}
 
Voor de werking van het DFT algoritme als signaalcompressie-algoritme is het van belang dat er een 
inverse algoritme bestaat dat het getransformeerde inputsignaal weer terugtransformeert 
zonder verlies van informatie. We zullen nu bewijzen dat dit mogelijk is door de DFT en iDFT te gebruiken.

%-----------------------------------------------------------------------------------------------------------------
\section{De Fast Fourier Transform}
\label{fft_sec}
De snelheid van het DFT algoritme valt in de praktijk nogal tegen, het nemen van $n$ inproducten over vectoren 
van lengte $n$ heeft namelijk een tijdscomplexiteit van $\O(n^2)$. Dit staat de directe implementatie van de DFT 
voor praktische toepassingen in de weg. Daarom is er een alternatief algoritme, de \emph{Fast Fourier Transform}.  \bigskip

%%%%%
\begin{algo}[Fast Fourier Transform]
Gegeven zij een inputsignaal $x$ van lengte $n=2^m$, dan geeft het algoritme $\FFT$ 
een lijst terug van waardes $X$ van lengte $n=2^m$ als volgt:

Als $m=0$ dan geeft de $\FFT$ de lijst (van \'e\'en element) direct terug:
\[
X = x.
\]
Wanneer $m\neq0$ splitsen we de lijst $x$ op in lijsten $\e,o$ van zijn even en oneven indices:
\eq{
  \e[k]   =& x[2k]   &\quad \text{voor } k < n/2\\
   o[k]   =& x[2k+1] &\quad \text{voor } k < n/2
}
Vervolgens voeren we hierop het $\FFT$ algoritme uit om de volgende lijsten te verkrijgen:
\eq{
  E =& \FFT(\e) \\
  O =& \FFT(o)
}
Hiermee wordt de output van het algoritme geconstrueerd als volgt:
\[
  X[k] = \left\{\begin{array}{llll}
    E[k]         &+& e^{-\dpii k/n}\cdot O[k] &  k< n/2 \\
    E[k-n/2] &-& e^{-\dpii (k-n/2)/n}\cdot O[k-n/2] &  k\geq n/2 
  \end{array}\right.
\]
\end{algo}
%%%%%

Dit is dus een recursief gedefinieerd algoritme dat een signaal meermaals halveert en in zichzelf terugvoert.
Het is gegarandeerd dat dit algoritme afloopt vanwege de conditie op $n=0$ samen met de halvering van de input bij elke stap. Een belangrijke voorwaarde voor de relevantie van de FFT is nu dat het algoritme hetzelfde resultaat geeft als het DFT algoritme en dit zullen we nu dan ook bewijzen. 

\begin{stelling}[]
  Het uitvoeren van het Fast Fourier Transform algoritme op een dataset van lengte $n=2^m$ geeft
  dezelfde getransformeerde als de discrete Fouriertranformatie.
\end{stelling}
\begin{proof}[Bewijs]
We gebruiken hier een inductief bewijs met inductie naar $n$. Onze aanname is dat het FFT-algoritme voor $x$ van lengte $n=2^m$ gelijk is aan de DFT van $x$, ofwel
\eq{
  X[k] = \sum^{n}_{k=1} x[j] \cdot e^{-2\pi i \cdot jk/n}
}
Dit geldt duidelijkerwijs wanneer $m=0$, onze basistap. Hiervoor geldt namelijk:
\eq{
  X[k] = x[k] = x[1] = \sum^{2^0}_{k=1} x[j] \cdot e^{-2\pi i \cdot 1/2^0}
}
Vervolgens passen we inductie toe naar $m$ door onze aanname voor $m-1$ te gebruiken,
we vullen hiermee $E[k]$ en $O[k]$ in de vergelijking voor $X[k]$ in, deze hebben immers lengte $n=2^{m-1}$.
\eq{
  X[k] = \left\{\begin{array}{llll}
    \sum^{n/2}_{j=1} \e[j] 
    \cdot e^{-2\pi i \cdot kj \cdot 2/n} &+& 
    e^{-2\pi i \cdot k/n}
    \sum^{n/2}_{j=1} o[j] 
    \cdot e^{-2\pi i\cdot kj \cdot 2/n} &  k< n/2 \\
    \sum^{n/2}_{j=1} \e[j] 
    \cdot e^{-2\pi i\cdot (k-n/2) j\cdot 2/n} &-& 
    e^{-2\pi i\cdot (k-n/2)/n}
    \sum^{n/2}_{j=1} o[j] \cdot e^{-2\pi i\cdot (k-n/2)j\cdot 2/n} &  k\geq n/2 
  \end{array}\right.
}
We merken op dat we de e-machten in het tweede geval kunnen vereenvoudigen volgens
\eq{
  e^{-2\pi i\cdot (k-n/2)j \cdot 2/n} 
  = - e^{-2\pi i\cdot kj\cdot 2/n} \quad,\quad e^{-2\pi i\cdot(k-n/2)/n} 
= -e^{-2\pi i\cdot k/n},
}
waardoor het gevalsonderscheid wegvalt, aangezien beide vergelijkingen nu identiek zijn.
We verkrijgen $X[k]$ als sommatie over de lijsten $e$ en $o$, we vullen de relatie voor $e,o$ met $x$ in, en nemen de factor voor de oneven indices mee in de sommatie.
\eq{
  X[k] = \sum^{n/2}_{j=1} x[2j] \cdot e^{-2\pi i\cdot k (2j)/n} + 
    \sum^{n/2}_{j=1} x[2j+1] \cdot e^{-2\pi i\cdot k (2j+1)/n} 
    = \sum^n_{j=1} x[j] \cdot e^{-2\pi i\cdot k j/n}
}
Dit bewijst dat de FFT hetzelfde resultaat levert als het DFT algoritme, het bewijs voor de gelijkheid van de iDFT en de inverse FFT is 
hetzelfde wanneer men de substitutie $-2\pi i \rightarrow 2\pi i$ uitvoert.
\end{proof}

\begin{opmerk}
We hebben hier telkens aangenomen, en zullen deze aanname ook doorzetten, 
dat de lengte van het ingangssignaal een macht van $2$ is. Dit is een belangrijke
eigenschap waar de variant van het FFT-algoritme dat hier gebruikt wordt door werkt. Deze versie van FFT wordt
de Radix-2 Decimation In Time van het Cooley-Tukey FFT algoritme genoemd. Algemenere vormen van het algoritme
worden ook toegepast in geoptimaliseerde algoritmes maar om de implementatie te versimpelen 
is voor Radix-2 gekozen. Eventuele verschillen in afmetingen tussen een signaal en 
een 2-macht zijn opgelost met signaalextensie, zoals eerder beschreven.
\end{opmerk}

%-----------------------------------------------------------------------------------------------------------------
\subsection{Complexiteit van de Fast Fourier Transform}
We zullen de claim bewijzen dat de complexiteit van de FFT werkelijk beter lager is dan die van de DFT.
Hiervoor hebben we de volgende stelling uit de Complexiteitstheorie nodig.\footnote{Dit is een speciaal geval van de stelling, voor de volledige stelleing en het bewijs zie \cite{akra-bazzi}.}

\begin{stelling}[Akra-Bazzi]
    Zij $T:\N\to\R$ een recurrente betrekking van de vorm
    \[
    T(n) = \begin{cases}
      c_0 &\text{ als } n \leq d \\
      a T(n/b) + f(n) &\text{ anders} \\
    \end{cases},
    \]
    waarbij $a,b,d\in\N$, $c_0\in\R$ en $f$ een functie $f:\N\rightarrow\R$ die voldoet aan 
    \[
    \exists k \in \N \,:\, f(n) \in \theta(n^{\log a/\log b} \log^k n),
    \]
    dan wordt de orde van $T(n)$ gegeven door:
    \[
      T(n) \in \theta(n^{\log a / \log b} \log^{k+1}n).
    \]
\end{stelling}
We beschouwen hier $T(n)$ als het aantal stappen dat een machine nodig heeft om het algoritme uit te voeren.
Deze stelling is voldoende om een uitspraak te kunnen doen over de complexiteit 
\begin{stelling}[Complexiteit van de FFT]
  Het Fast Fourier Transform algoritme heeft een tijds-complexiteit $\O(n\log n)$ voor input van lengte $n=2^m$ 
\end{stelling} 
\begin{proof}[Bewijs]
We schrijven het FFT algoritme in pseudocode.

\begin{algorithmic}
\Function{FFT}{$x$}
\State $n \gets \text{lengte}(x)$ \Comment Assumptie: $n = 2^m$ voor een $m$
\If {$n == 1$}
	\State{$X \gets x$}
\Else
	\State $E \gets FFT(x[0::2])$ \Comment{FFT op even indices}
	\State $O \gets FFT(x[1::2])$ \Comment{FFT oneven indices}
	\For{$i = 0$ to $n-1$}
		\If{$i < n/2$}
			\State $X[i] \gets E[i] + e^{-2i \pi k/n} \cdot O[i]$
		\Else
			\State $X[i] \gets E[i] - e^{-2i \pi k/n} \cdot O[i]$
		\EndIf
	\EndFor
\EndIf
\State \Return{$X$}
\EndFunction
\end{algorithmic}

Dit algoritme is recursief, dus kunnen we de complexiteit schrijven door middel van een recurrente betrekking. Laat hiervoor $T(n)$ het aantal berekeningen zijn dat het algoritme kost bij een invoersignaal van lengte $n$. We maken een gevalsonderscheid: als de lijst lengte $1$ heeft geven we deze direct terug (1 berekening). Bij een lijst van lengte $>1$ splitsen we de lijst op in de even en oneven entries en voeren we op beiden weer het FFT algoritme uit, vervolgens voeren we nog $n$ maal een vast aantal ($N$) berekeningen uit om tot het eindresultaat te komen. In formulevorm geeft dit de \emph{recurrente betrekking}
\[
T(n) = \begin{cases}
    1 &\text{ als } n = 1 \\
    2\cdot T(n/2) + N\cdot n &\text{ anders}. \\
\end{cases}
\]
We zullen nu bovenstaande (vereenvoudigde) stelling van Akra-Bazzi gebruiken. We rekenen hierbij niet de $\O$ van de FFT uit maar de strictere $\theta$ die gedefinieerd is volgens:
\[
f \in \theta(g) \Leftrightarrow \exists c \in \R_+: \lim_{n \to \infty} \frac{f(n)}{g(n)} = c
\]
Zonder verder al te veel in te gaan op de implicaties die $\theta$ heeft op het gedrag van de FFT vermelden we dat in ieder geval geldt dat $f \in \theta(g)$ impliceert dat $f \in \O(g)$. 

De recurrente betrekking voor de complexiteit van de FFT is inderdaad van dezelfde vorm als die van $T(n)$ in bovenstaande stelling.
Laat hiervoor namelijk $a=b=2$, $c_0=d=1$ en $f (n) = N\cdot n$:
\eq{
  f(n) \in \theta(n^{\log 2/\log 2} \log^0 n)=\theta(n).
}
Dit betekent dat $T(n) \in \theta(n \log n)$ en dus zeker $T(n) \in \O(n \log n)$.
Hiermee hebben we bewezen dat de FFT en daarmee de iFFT binnen tijdscomplexiteit $\O(n\log n)$ lopen. 
\end{proof}

\begin{opmerking}
Omdat het algoritme niet in lineaire tijd loopt, neemt de wachttijd snel toe bij grote signalen. 
In meer dimensies wordt dit al snel een praktisch bezwaar, dit is dan ook een reden waarom we geen driedimensionale 
signalen hebben bekeken bij de implementatie van de Fouriertransformatie.
\end{opmerking}

%-----------------------------------------------------------------------------------------------------------------
\section{Discrete Fourier Transform in meer dimensies}
Een eigenschap van het DFT-algoritme is dat het op een natuurlijke manier uit te breiden is naar hogere dimensies.
Per extensie daarvan is er een manier om het FFT-algoritme mee te laten schalen, die we ook zullen behandelen.
Het idee hierbij is om het Tensorproduct te nemen van meerdere bases.

\begin{definitie}[Multidimensionale Discrete Fourierbasis] 
Gegeven zij een signaalruimte van de vorm
$\R^{n_1} \times \R^{n_2} \times \ldots \times \R^{n_m}$ 
waarvan de elementen $m$-dimensionale discrete signalen zijn waarbij de $i$-de richting een lengte $n_i$ heeft.
We defini\"eren de multidimensionale discrete Fourierbasis behorende bij deze signaalruimte als
\[
  S_{\bold n}= S_{n_1}\otimes S_{n_2} \otimes \ldots \otimes S_{n_m} = 
  \left\{s_{\bold k}[{\bold j}]  = s_{k_1}[j_1]\cdot s_{k_2}[j_2]\cdot\ldots\cdot s_{k_m}[j_m] 
  \largediv s_{k_i} \in S_{n_i}, j_i \in \{1,\ldots, n_i\} \right\}
\]
Met basisvectoren $s_{\bold k}$, waar $\bold k$ een indexvector uit $\{1, \ldots, n_1\}\times\ldots\times\{1\ldots, n_m\}$ is.
Het feit dat dit een basis is, kan gevonden worden in \cite{topo}.
\end{definitie}

Wanneer we nu een $m$-dimensionaal signaal bekijken van lengte $n_1 \times \cdots \times n_m$ etc., dan kunnen we
hierop een multidimensionale Fouriertransformatie op defini\"eren door het inproduct te generaliseren naar 
onze $m$-dimensionale signaalruimte.
\begin{equation}
  \label{algemeen_fourier_schema}
  X[\bold k] = \tfrac{1}{\bold n}
  \sum_{\bold j = \bold 1}^{\bold n} x[\boldsymbol j] s^{-1}_{\bold k}[\boldsymbol j] 
  =
  \left(\prod_{i=1}^m \tfrac{1}{n_i}\right) \cdot 
  \sum_{j_1=1}^{n_1} \ldots \sum_{j_m=1}^{n_m} 
  x[j_1,\ldots,j_m] \cdot 
  e^{-\dpii\cdot k_m \cdot j_m /n_m } \cdot \ldots \cdot e^{-\dpii\cdot k_1 \cdot j_1 /n_1 } 
\end{equation}

Maar zoals we voor de DFT al gezien hadden is het uitrekenen van al deze inproducten erg complex (in de tijd).
We zullen daarom de Multidimentionale Discrete FourierTransformatie (MDFT) op een andere manier defini\"eren 
zodat we beter gebruik kunnen maken van de DFT en FFT algoritmen die we al gevonden hebben.

\begin{algo}[Multidimensionaal DFT-algoritme]
Gegeven is een $m$-dimensionale input $x$ en een huidig level $t$, 
we schrijven het algoritme met output $X_t$ volgens
\begin{equation}
  \label{mdft_cases}
  X_{t}[\boldsymbol k] = \begin{cases}
  \DFT(X_{t-1}\largediv_{x_1 = k_1,\ldots,x_{t-1} = k_{t-1},x_{t+1} = k_{t+1},\ldots,x_m =k_m})[k_t] & \text{als } t>0\\
  x[k_t] & \text{als } t=0
  \end{cases}
\end{equation}
Waar we de notatie $X\largediv_{x_i=c}$ gebruiken voor het object van kleinere dimensie dat verkregen
wordt door de $i$-de co\"ordinaat vast te leggen op $c$.
We beweren dat vervolgens $X_m$ de multidimensionale Fouriertransformatie van een $m$-dimensionaal signaal geeft zoals
in \ref{algemeen_fourier_schema}.
\end{algo}
\begin{proof}[Bewijs]
We voeren inductie naar de dimensie $m$ uit met de inductiehypothese dat voor een dimensie $m$ 
de vergelijking voor $X_m$ uit het algoritme gegeven wordt door \ref{algemeen_fourier_schema}.\\ 
Als $m=1$ dan geldt:
\[
X_1[k_1] = DFT(X_0)[k_1] = DFT(x)[k_1]
\]
Wat dus inderdaad de DFT in 1 dimensie geeft.
We gaan nu door met de inductiestap, voor algemene $m>1$ geldt:
Schrijf even kort
\[
\tilde X_{t-1}[k_t] = 
X_{t-1}\largediv_{x_1 = k_1,\ldots,x_{t-1} = k_{t-1},x_{t+1} = k_{t+1},\ldots,x_m =k_m}[k_t]
\]
\[
  X_m [\boldsymbol k] = 
  \DFT(\tilde X_{m-1})[k_m]
  = \frac 1 {n_m}\sum_{j_m=1}^{n_m} \tilde X_{m-1}[j_m] s^{-1}_{k_m}[j_m]
\]
We vatten nu $\tilde X$ op als een vector van lengte $n_m$ van $m-1$ dimensionale Fouriergetransformeerden
dit mag omdat
\[
X_{m-1}\largediv_{x_1=k_1,\ldots,x_{m-1}=k_{m-1}}[k_m] = X_{m-1}\largediv_{x_m=k_m}[k_1,\ldots,k_{m-1}]
\]
en daarmee $X_{m-1}\largediv_{x_m=k_m}$ een $m-1$-dimensionaal object is dat weer wordt gegeven door de relatie
in \ref{mdft_cases}.
Omdat we de aanname voor $m-1$ dimensies al bewezen hebben geldt nu:
\[
X_m[\boldsymbol k]  = \frac 1 {n_m}\sum_{j_m=1}^{n_m} 
\left( \frac 1 {n_{m-1}} \sum_{j_{m-1}=1}^{n_{m-1}} \left ( \cdots 
\frac 1 {n_1} \sum_{j_1=1}^{n_1} 
x[j_1,\ldots,j_m] 
\cdot s^{-1}_{k_1}[j_1]
\cdots \right ) s^{-1}_{k_{m-1}}[j_{m-1}] \right) 
s^{-1}_{k_m}[j_m]
\]
Wat precies is wat we wilden aantonen.
\end{proof}

Een belangrijk gevolg van de definitie van dit algoritme is dat de DFT term in \ref{mdft_cases}
gemakkelijk vervangen kan worden door een FFT term, beiden geven immers dezelfde output. 
Daarmee hebben we ook direct een MFFT gevonden.
De complexiteit van deze algoritmen is dan $\O( \sum_{i=1}^m \log(n_i) \prod_{j=1}^m n_j)$  voor de MFFT
t.o.v. $\O( \sum_{i=1}^m n_i \prod_{j=1}^m n_j)$ voor de MDFT.

%-----------------------------------------------------------------------------------------------------------------
\section{Compressie van een signaal onder FFT}
Tot zover is de Discrete Fourieranalyse besproken aan de hand van perfecte reconstructie, 
aan de hand van een  complete set co\"efficienten. Het doel van dit project is echter om
signalen te comprimeren; te reconstrueren aan de hand van een gelimiteerde dataset.
Om deze analyse te vegemakkelijken, zullen in deze sectie een aantal bewijzen aan bod komen die
de geinduceerde fout van zo een reconstructie relateren aan de grootte van de dataset.

We zullen het convergentie-gedrag gaan bepalen van de Fourier-transformatie voor functies die $C^k$ zijn.
We weten immers dat voor differentieerbare functies de Fourieranalyse perfect reconstrueerbaar is.
Met het convergentie gedrag van deze functies kunnen we dan kwantificeren hoe goed een functie te 
benaderen is met een eindige subset van de Fourier-getransformeerde.
We trachten daarom te bewijzen dat er een algebra\"isch verband bestaat tussen de fout en de frequentie
van de wavelets en bovendien een exponentieel verband tussen de fout en de `gladheid' van de functie.

Voor dit bewijs zullen we het Lemma van Riemann-Lebesgue gebruiken: 
\begin{lemm}[Riemann-Lebesque{\cite{fourier-fout}}]
Wanneer $g \in L_1(\R)$, dan geldt 
\eq{
	G(z) = \left|\int_{-\infty}^\infty g(t) e^{- 2 \pi i z t} dt\right| \to 0 \text{ voor } z \to \infty.
}
\end{lemm}

\begin{stelling}[Daling van de co\"effici\"enten van de Fouriergetransformeerde]
\label{fourier_daling}
  Laat $f \in L_2([a,b])$. Stel er is een $n$ z\'o dat $f \in C^n$ en voor $0\leq j\leq n$ geldt dat $f^{(j)}(a) = f^{(j)}(b)$. Dan geldt dat de $k$-de entry van de Fouriergetransformeerde $\hat f[k]$ in absolute waarde
  daalt met $k$ volgens $\O(|k|^{-n})$.
\end{stelling}
\begin{proof}[Bewijs]
  Laat $f \in L_2([a,b])$ zodat $f$ voldoet aan de voorwaarden in de stelling voor een bepaalde $n$. 
  De Fourier-getransformeerde van $f$ wordt gegeven door:
  \eq{
    \hat f [k] = \tfrac{1}{\sqrt{b-a}} \inpr{f}{\phi_k}.
  }
  De constante in deze vergelijking heeft geen invloed op de orde, dus richten we ons op het inproduct. 
  We schrijven dit uit tot een integraal en voeren vervolgens, omdat $f$ differentieerbaar is, partie\"ele
  integratie uit.
  \eq{
    \abso{\inpr{f}{\phi_k}} 
    =& \abso{\int_a^b f(x) e^{- 2 \pi i k \tfrac{x-a}{b-a}} dx}
    = \abso{\tfrac{b-a}{2 \pi i k}\left[ f(x) \cdot  e^{- 2 \pi i k \tfrac{x-a}{b-a}} \right]_a^b} 
    + \abso{\frac{b-a}{2 \pi i k}\right| \left|\int_0^1 f'(x) e^{-2 \pi i k \tfrac{x-a}{b-a}} dx} \\
    =& \abso{ f(b) \cdot 1 - f(a) \cdot 1} + \frac{b-a}{2 \pi \abso{k}} 
    \abso{ \int_a^b f'(x) e^{-2 \pi i k \tfrac{x-a}{b-a}} dx }
    = \tfrac{b-a}{2 \pi |k|} \left| \int_a^b f'(x) e^{-2 \pi i k \tfrac{x-a}{b-a}} dx \right|.
  }
  Hier hebben we gebruikt dat alle afgeleiden van $0$ tot $n$ periodiek zijn.
  Omdat $f$ nu $C^n$ is en de afgeleiden weer periodiek zijn kunnen we dit herhaald toepassen,
  we verkrijgen daarmee
  \[
  \left| \int_a^b f(x) e^{-2 \pi i k \tfrac{x-a}{b-a}} dx \right| 
  = (\tfrac{2 \pi}{b-a} |k|)^{-n}\left| \int_a^b f^{(n)}(x) e^{- 2 \pi i k \tfrac{x-a}{b-a}} dx \right|.
  \]
  
  Vermenigvuldig beide kanten met $(\tfrac{2 \pi}{b-a} |k|)^n$ om te vinden dat
  \[
  (\tfrac{2 \pi}{b-a} |k|)^n \abso{\inpr{f}{\phi_k}} 
  = \abso{\int_a^b f^{(n)}(x) e^{- 2 \pi i k \tfrac{x-a}{b-a}} dx}.
  \]
  We willen graag het lemma van Riemann-Lebesgue toepassen op de rechterkant. 
  Merk daartoe op dat omdat $f \in C^n$, $f^{(n)}$ continu op $[a,b]$ 
  dus neemt hier een maximum en een minimum aan. 
  Daarmee is de integraal van $|f|$ begrensd en dus is $f\in L_1([a,b])$. 
  Maar een functie die integreerbaar is op $[a,b]$ en daarbuiten nul, is integreerbaar op $\R$. 
  Dus we mogen Riemann-Lebesgue gebruiken om te zien dat
  \[
  (\tfrac{2 \pi}{b-a} |k|)^n \abso{\inpr{f}{\phi_k}} \to 0 \text{ als } |k| \to \infty.
  \]
  Maar dit betekent precies dat $\abso{\inpr{f}{\phi_k}} \in  o((\tfrac{2 \pi}{b-a} |k|)^{-n}) = o(|k|^{-n})$
\end{proof}

De analyse van de co\"effici\"enten die we hier gegeven hebben vertaalt direct naar de fout die we krijgen
bij reconstructie van een Fourier-getransformeerde functie wanneer we een gelimiteerde dataset gebruiken.

\begin{gevolg}
Schrijf namelijk $f|_{N}$ voor de reconstructie met de eerste $N$ basisfuncties. Dan hebben we de relatie
\[
  || f - f|_N||^2_{L_2([a,b])} = \sum_{k=1}^\infty |\inpr{f-f|_N}{\phi_k}|^2 = \sum_{k=1}^\infty |\inpr{f}{\phi_k} 
  - \inpr{f|_N}{\phi_k}|^2 = \sum_{k=N+1}^\infty |\inpr{f}{\phi_k}|^2,
\]
vanwege de Parsevalgelijkheid \ref{parseval}. Omdat stelling \label{fourier_daling} geldt, zijn er $k_0, c$ zodat $\abso{\inpr{f}{\phi_k}} < c\cdot k^{-n}$ voor $k > k_0$. Dus
\[
	\sum_{k=N+1}^\infty | \langle f, \phi_{k} \rangle |^2 \leq \sum_{k=N+1}^\infty c \cdot k^{-2n} < c \cdot \int_{N}^\infty x^{-2n} dx = c \cdot \left[ \frac{x^{1-2n}}{1-2n} \right]^\infty_N = c \cdot \frac{N^{1-2n}}{2n-1}.
\]
Dan
\[
N^{2(n-1)} \cdot \sum_{N+1}^\infty |\inpr{f}{\phi_k}|^2 < N^{2(n-1)} \cdot c \cdot \frac{N^{1-2n}}{2n-1} = \frac{c}{2n-1} N^{-1} \to 0 \quad \text{als } N\to\infty,
\]
wat precies betekent dat
\[
||f-F||^2_{L_2([a,b])} \in o\left ( N^{-2(n-1)} \right) \implies ||f - F||_{L_2([a,b])} \in o\left(N^{-(n-1)}\right).
\]
\end{gevolg}

\subsection{De fout in discrete geval}
Wanneer we een discrete functie bekijken zullen we het criterium van \emph{gladheid} niet kunnen gebruiken,
we bekijken de functie immers op een deelverzameling van $\Z$ en daar is geen enkele 
niet-constante functie continu op.
We beroepen ons daarom op de analogie tussen het discrete en non-discrete geval.
De gladheid van een discrete functie $f:A\to\R$ wordt bepaald door de verschillen $f[x]-f[x+1]$,
wanneer de verschillen klein zijn en de verzameling $A$ groot is komt dit overeen met de afgeleide van een 
continue functie en zeggen we dat de functie glad is. Zo kunnen we analoga geven voor hogere afgeleiden.
Het is mogelijk om een rigoreuze beschrijving van gladheid te geven door parti\"eele sommaties te gebruiken,
maar daar zullen we niet op in gaan.



\chapter{Wavelets}
De Fouriertransformatie bestaat al honderden jaren en is een grote speler geworden in de \emph{signal processing}. Een groot nadeel van deze transformatie is dat zij slecht reageert op discontinue signalen door de globale dragers van de basisfuncties. Hierdoor worden alle Fourierco\"effici\"enten be\"invloed door een discontinu\"iteit.

In de loop van de vorige eeuw is een nieuwe transformatie ontstaan met een eigenschap die de Fouriertransformatie nooit kende. Deze noemt men nu ook wel de Wavelettransformatie.

\begin{definitie}
  Een wavelet is simpelweg een functie $\psi: \R \to \R$ die voldoet aan
  \[
  \int_{-\infty}^{\infty} \psi(t) dt = 0.
  \]
  Met deze functie $\psi$ kunnen we een familie functies $\psi_{u,s}$ bouwen door middel van schaling en translatie:
  \[
  \psi_{u,s}(t) := \frac{1}{\sqrt{s}} \psi\left(\frac{t-u}{s}\right).
  \]
\end{definitie}

Deze familie geeft aanleiding tot een Wavelettransformatie $W_f$ van $f$:
\[
W_f(u,s) = \int_{-\infty}^\infty f(t) \psi^*_{u,s}(t) dt.
\]

Het is nu mogelijk om wavelets te construeren die met deze schaling en translatie een basis voor de $L_2(\R)$ vormen. Over het algemeen kijken we dan naar
\[
\left\{ \psi_{j,n}(t) = \sqrt{2^j} \psi\left( 2^j t - n\right) : (j,n) \in \Z^2 \right\}.
\]
De kunst is nu om de basiselementen loodrecht op elkaar te laten staan, zodat er een orthogonale (en dus een orthonormale) basis gevormd wordt. Zie figuur~\ref{fig:wavelets} voor twee wavelets.

\begin{figure}[h]
  \centering
  \begin{subfigure}{0.48\linewidth}
    \includegraphics[width=\linewidth]{plaatjes/db1.pdf}
  \end{subfigure}
  \begin{subfigure}{0.48\linewidth}
    \includegraphics[width=\linewidth]{plaatjes/db4.pdf}
  \end{subfigure}
  \caption{Links: Haar. Rechts: De Daubechies-4. De gestippelde grafiek is een translatie naar rechts en staat in beide gevallen loodrecht op de continue lijn die de waveletfunctie voorstelt.}
\label{fig:wavelets}
\end{figure}

\begin{gevolg}We kunnen een functie $f$ in $L^2(\R)$ schrijven in deze basis:
  \[
  f(t) = \sum_{j=-\infty}^{\infty} \sum_{n=-\infty}^{\infty} \langle f, \psi_{j,n} \rangle \psi_{j,n}(t),
  \]
  waarbij $\langle \cdot, \cdot \rangle$ het standaardinproduct op de $L_2(\R)$ aangeeft.
\end{gevolg}

Het grote nadeel van de Fouriertransformatie maakt compressie van discrete signalen moeilijk. Veel van deze wavelets worden nu z\'o geconstrueerd dat dit probleem (deels) verholpen wordt. We zijn namelijk op zoek naar een wavelet die een eindige drager heeft. Het blijkt dat deze bestaat en dat er zelfs een hele grote verzameling wavelets is, elk met eigen gewilde eigenschappen.

Omdat wij naar de toepassing van wavelets binnen de beeldcompressie bekijken, zijn we natuurlijk vooral ge\"interesseerd in het discrete geval. We kijken dus naar de benadering van $f$. Dit geeft aanleiding tot een rij geneste ruimtes die uiteindelijk naar de $L_2(\R)$ toe gaat:

\begin{equation}
  \label{multires}
  V_0 \subset V_1 \subset \ldots \subset L_2(\R)
\end{equation}
genaamd een multiresolutie.
\begin{definitie}
  Een rij geneste ruimtes $\{ V_j: j \in \N_0 \}$ zoals in~\ref{multires} heet een multiresolutie wanneer voldaan wordt aan de volgende eigenschappen:
  \begin{eqnarray}
    \forall j, k: f(t) \in V_j \implies f(t - 2^j k) \in V_j, \\
    \forall j: V_{j-1} \subset V_j, \\
    \forall j: f(t) \in V_j \iff f(t/2) \in V_{j-1}, \\
    \bigcup_{j=0}^{\infty} V_j = \lim_{j\to\infty} V_j = L_2(\R), \\
    \text{ Er is $\phi: \R \to \R$ zo dat $\{ \phi(t-n): n \in \Z \}$ een orthonormale basis voor $V_0$ is.}
  \end{eqnarray}
\end{definitie}

\begin{voorbeeld} We bekijken een multiresolutie van stuksgewijs constante functies. De ruimte $V_j$ wordt hiermee
  \[
  V_j = \left\{ g(t) \in L_2(\R): g(t)\text{ is constant voor }t \in [n 2^{-j}, (n+1)2^{-j}) \right \}
    \]
    met $n \in \Z$. De basisfunctie $\phi$ voor $V_0$ wordt in dit geval $\phi(t) = 1_{[0,1)}(t)$.
\end{voorbeeld}

\section{Schalingsfuncties}
Gegeven zo'n orthonormale basis voor $V_0$ willen we graag een orthonormale basis voor $V_j$ construeren.
\begin{stelling}[{\cite[T7.1]{mallat}}]
  Laat $\{ V_j \}$ een multiresolutie en laat $\{\phi(t-n) \}$ de basis voor $V_0$. Laat verder
  \[
  \phi_{j,n}(t) := \sqrt{2^j} \phi\left( t2^j - n \right).
  \]
  Dan is $\{ \phi_{j,n}: n \in \Z \}$ een orthonormale basis voor $V_j$. De functie $\phi$ is ook wel de \emph{schalingsfunctie}.
\end{stelling}
\subsection{Benadering} De orthogonale projectie van $f$ op $V_j$ is, zoals we weten, de beste benadering van $f$ in $V_j$. Deze is nu te vinden door
\[
P_{V_j} f = \sum_{n=-\infty}^\infty \langle f, \phi_{j,n} \rangle \phi_{j,n}.
\]
De co\"effici\"enten $a_j[n] = \langle f, \phi_{j,n} \rangle$ geven ons op deze manier een discrete benadering van $f$ op resolutie $2^j$.

\section{Filters}
Wanneer we een schalingsfunctie $\phi$ defini\"eren (en dus een $V_0$), dan lijkt $V_1$ (en dus $V_j$) al redelijk beschreven te worden.\footnote{In stelling \ref{filter} wordt bewezen dat deze hele multiresolutie vast ligt.} We zullen daarom deze schalingsfunctie nader onderzoeken.

Per definitie van de multiresolutie weten we dat $V_{j-1} \subset V_j$. In het bijzonder geldt dat $\phi(t) \in V_0 \subset V_1$ en omdat $\{ \sqrt{2}\phi(2t - n): n \in \Z\}$ een orthonormale basis voor $V_1$ is, kunnen we $\phi(t)$ nu schrijven als
\[
\phi(t) = \sum_{n=-\infty}^{\infty} \left\langle \sqrt{2} \phi\left(2t-n\right), \phi(t) \right\rangle \sqrt{2}\phi(2t-n).
\]

\begin{definitie}
  Deze inproducten hebben een speciale naam, want de rij $\{h[n]: n \in \Z\}$ met
  \[
  h[n] := \left\langle \sqrt{2} \phi\left(2t-n\right), \phi(t) \right\rangle
  \]
  wordt nu ook wel de \emph{filter} van $\phi$ genoemd.
\end{definitie}
\begin{stelling}[{\cite{mallat}[T7.2]}]
  \label{filter}
  Laat $\phi \in L^2(\R)$ een schalingsfunctie die ook integreerbaar is. Dan ligt de multiresolutie vast.

  Andersom, als $h[n]$ een filter is zodat $\hat{h}(\omega)$ periodiek $2\pi$ is en continu differentieerbaar in een omgeving van $\omega = 0$ en als daarnaast geldt
  \begin{align*}
    \forall \omega \in \R: | \hat{h}(\omega)|^2 + |\hat{h}(\omega + \pi)|^2 = 2, \\
    \hat{h}(0) = \sqrt{2}, \\
    \inf_{\omega \in [-\pi/2, \pi,2]} |\hat{h}(\omega)| > 0,
  \end{align*}
  dan is de functie $\phi$ waarvan de Fouriergetransformeerde voldoet aan
  \[
  \hat{\phi}(\omega) = \prod_{p=1}^\infty \frac{\hat{h}(2^{-p}\omega)}{\sqrt{2}}
  \]
  een schalingsfunctie in $L^2(\R)$.
\end{stelling}
We zullen enkel de gevolgen gebruiken: namelijk dat de multiresolutie vast ligt met een goede keuze van $\phi$, en dat voor een goed gekozen $h[n]$, $\phi$ ook vast ligt.

\begin{voorbeeld}
  Bekijk weer het geval $\phi(t) = 1_{[0,1)}(t)$. Dan vinden we dat
    \[
    h[n] = \left\langle \sqrt{2} \phi\left(2t-n\right), \phi(t) \right\rangle = \begin{cases} \frac{1}{\sqrt{2}} & \text{ als } n \in \{0,1\} \\ 0 & \text{ anders.} \end{cases}
    \]
\end{voorbeeld}

\section{Terugkeer van de wavelet}
We weten dat $V_{j-1}$ bevat is in $V_{j}$. Laat nu $W_{j-1}$ het orthogonale complement van $V_{j-1}$ in $V_{j}$:
\begin{equation}
  \label{ruimterec}
  V_{j} = W_{j-1} \oplus V_{j-1}
\end{equation}
De projectie van $f$ op $V_{j-1}$ kan dus geschreven worden als som van projecties:
\begin{equation}
  \label{projectie_rec}
  P_{V_{j}} f = P_{V_{j-1}} f + P_{W_{j-1}} f.
\end{equation}
Omdat $V_j \subset V_{j+1}$ is alle informatie over $f$ die beschikbaar is in $V_j$, ook beschikbaar in $V_{j+1}$. Ook is het goed mogelijk dat door deze grovere benadering, informatie zoek gaat. Deze `details' worden op die manier zichtbaar in $P_{W_j} f$.

Het kan bewezen worden \cite{mallat}[T7.3] dat, gegeven een schalingsfunctie $\phi$ (en daarmee een filter $h$) er een functie $\psi$ bestaat zo dat
\[
\left\{ \psi_{j,n}(t) := \sqrt{2^j} \psi\left(2^jt - n\right) : n \in \Z \right\}
\] een orthonormale basis is voor $W_j$ en $\{ \psi_{j,n}: (j,n) \in \Z^2 \}$ een basis voor $L_2(\R)$. Deze functie is dan een \emph{orthogonale} wavelet, omdat $W_j \perp V_j$.
Omdat nu $W_j \subset V_{j+1}$ en dus in het bijzonder $\psi(t) \in W_0 \subset V_1$ en omdat $\{ \sqrt{2}\phi(2t-n): n \in \Z \}$ een orthonormale basis is voor $V_1$, kunnen we ook $\psi(t)$ in termen schrijven als:
\[
\psi\left(t\right) = \sum_{n=-\infty}^{\infty} \left\langle \psi\left(t\right), \sqrt{2}\phi(2t-n) \right\rangle \sqrt{2}\phi(2t-n).
\]
Ook deze inproducten hebben een speciale naam: de rij $g[n]$ met
\[
g[n] := \left\langle \psi\left(t\right), \sqrt{2}\phi(2t-n) \right\rangle
\]
wordt nu ook wel de filter van $\psi$ genoemd. De twee filters zijn gerelateerd aan elkaar volgens de vergelijking\cite{wavelet_filter[V13]}\cite{daubechies[P958]}
\begin{equation}
\label{highpassfilter}
g[n] = (-1)^{n}h[1-n].
\end{equation}

Zoals nu wel duidelijk geworden is, wordt met een filter $h$ (die voldoet aan bepaalde eigenschappen: zie stelling~\ref{filter}) een schalingsfunctie $\phi$ en een filter $g$ met waveletfunctie $\psi$ geconstrueerd.

\begin{voorbeeld}
  We keren nog een laatste keer terug naar het voorbeeld waarin $\phi(t) = 1_{[0,1)}$. We vinden met de gelijkheden uit voorgaande paragrafen dat
    \[
    \psi\left(t\right) = \sum_{n=-\infty}^{\infty} (-1)^{n}h[1-n] \sqrt{2}\phi(2t-n),
    \]
    en omdat $h[0] = h[1] = 2^{-1/2}, h[n] = 0$ voor $n \in \Z \setminus \{0,1\}$ zoals we eerder vonden, herschrijft dit tot
    \[
    \psi\left(t\right) = \sqrt{2}\left(\phi(2t) - \phi(2t - 1)\right)
    \]
    met als gevolg dat
    \[
    \psi(t) = \begin{cases} 1 & \text{ als } t \in [0,1/2) \\ -1 & \text{ als } t \in [1/2,1) \\ 0 & \text{ anders.} \end{cases}
    \]

    Deze wavelet $\psi$ wordt ook wel de Haarwavelet genoemd en is uitgevonden voor Alfred Haar in 1909, hoewel het onderzoeksgebied van de wavelets toen nog niet bestond. In het vervolg zullen we nog verdere aandacht aan deze wavelet besteden.
\end{voorbeeld}

\section{Het kiezen van een wavelet}
Bij het kiezen of vinden van een wavelet is men over het algemeen op zoek naar bepaalde eigenschappen. Voor compressie zijn we op zoek naar een wavelet die een klein aantal grote co\"effici\"enten en een groot aantal kleine teweeg brengt: een soort concentratie van de belangrijke informatie. Dit wordt vooral bepaald door drie factoren: gladheid van $f$ (waar we niks aan kunnen doen), de grootte van de drager (welke hierna aan bod komt) en de zogenaamde orde van de wavelet.

\begin{definitie}
  Wanneer de waveletfunctie loodrecht staat ($\langle \psi, q\rangle = 0$) op alle polynomen van graad $p-1$ of lager, spreken we van een wavelet van orde $p$. Dit komt overeen met te zeggen dat
  \[
  \int_{-\infty}^\infty x^k \psi(x) dx = 0 \text{ voor } k \in \{ 0, \ldots p-1 \}.
  \]
\end{definitie}

\begin{gevolg}
  Gevolg van deze eigenschap is dat we van de functie $f$ elk polynoom van graad $p-1$ af mogen trekken zonder een verschil in inproduct:
  \[
  \langle f, \psi_{j,n} \rangle = \langle f - q, \psi_{j,n} \rangle \text{ voor $q$ een polynoom van graad $p-1$}.
  \]
  Intu\"itief is deze eigenschap natuurlijk gewild: we winnen immers een heel stel keuzevrijheden. We zullen dit argument in een volgende sectie formaliseren.
\end{gevolg}

Eerder spraken we het verlangen uit om een wavelet met eindige drager te vinden zodat discontinu\"iteiten alleen lokaal zichtbaar zijn. We zullen hier de dragers van $h[n], \psi$ en $\phi$ aan elkaar relateren.

\subsection{Compacte drager}
Hoewel $\phi$ een functie uit $L_2$ is, en $h$ een functie uit $\ell_2$, is het toch mogelijk het begrip drager in beide ruimtes te beschrijven alsof ze hetzelfde zijn. Laat daartoe $h'$ de stuksgewijs constante functie van $\R$ naar $\R$ die $x$ stuurt naar $h(\lfloor x \rfloor)$. Dan is de drager van $h$ gedefini\"eerd als de drager van $h'$.

\begin{stelling}[{\cite[P7.2]{mallat}}]
  De volgende relaties gelden voor de dragers.
  \begin{enumerate}
  \item De schalingsfunctie $\phi$ heeft een compacte drager dan en slechts dan als het filter $h[n]$ een compacte drager heeft, en deze zijn hetzelfde.
  \item Als de drager van $\phi$ gelijk is aan $[N_1,N_2]$ dan is de drager van $\psi$ gelijk aan $[(N_1 - N_2 + 1)/2, (N_2 - N_1 + 1)/2]$.
  \end{enumerate}
\end{stelling}
\begin{proof}[Bewijs 1] Als $\phi$ een compacte drager heeft dan $h[n]$ ook: we weten dat
  \[
  h[n] = \left\langle \sqrt{2} \phi\left(2t-n\right), \phi(t) \right\rangle,
  \]
  dus er kunnen maar eindig veel $n$ ongelijk nul zijn. De omgekeerde bewering staat bewezen in \cite{daubechies[P965-967]}. TODO?

  Om deze dragers gelijk te krijgen, stel dat de drager van $h[n]$ gelijk $[N_1,N_2]$ is, en die van $\phi$ $[K_1, K_2]$. De drager van $\phi(t/2)$ is $[2K_1, 2K_2]$ en de drager van de rechterzijde van $\ref{phi_t2}$ is $[N_1 + K_1, N_2 + K_2]$. We concluderen dat $K_1 = N_1$ en $K_2 = N_2$.
\end{proof}
\begin{proof}[Bewijs 2]
  Kijk nu naar
  \[
  \psi\left(t\right) = \sum_{n=-\infty}^{\infty} g[n] \phi(2t-n) = \sum_{n=-\infty}^{\infty} (-1)^{n}h[1-n] \phi(2t-n).
  \]
  Met de informatie uit het begin van de stelling kunnen we de drager van de rechterkant vinden: $[N_1 - N_2 + 1, N_2 - N_1 + 1]$. De functie $\psi(t/2)$ is nu precies een dilatie met factor twee dus de drager van $\psi(t)$ moet wel gelijk zijn aan $[(N_1 - N_2 + 1)/2, (N_2 - N_1 + 1)/2]$.
\end{proof}

\subsection{Daubechieswavelets}
Hoewel de constructie van de Daubechieswavelet buiten het spectrum van dit artikel valt,\footnote{Voor een goede beschrijving van deze constructie, zie \cite{mallat} of \cite{daubechies}.} willen we toch een kort licht schijnen op deze speciale familie van wavelets. Deze worden gemaakt met de noties van eerder, namelijk dat we de drager willen minimaliseren maar de orde maximaliseren. Daubechies heeft bewezen\cite{daubechies} dat een filter $h$ met orde $p$, minimaal een drager van lengte $2p$ moet hebben. De zogenaamde Daubechieswavelet van orde $p$ heeft precies een filter van lengte $2p$. In het bijzonder is de Haarwavelet de eerste in de familie van Daubechieswavelets.

Wij hebben in het praktische deel van ons project aandacht besteed aan de zogenaamde Daubechies-2 wavelet die haar naam ontleent aan het feit dat zij van orde 2 is.

\begin{figure}[h]
  \centering
  \begin{subfigure}{0.48\linewidth}
    \includegraphics[width=\linewidth]{plaatjes/db2_phi.pdf}
  \end{subfigure}
  \begin{subfigure}{0.48\linewidth}
    \includegraphics[width=\linewidth]{plaatjes/db2_psi.pdf}
  \end{subfigure}
  \caption{Links: de Daubechies-2 schalingsfunctie. Rechts: de Daubechies-2 waveletfunctie.}
\end{figure}

\section{Fast Wavelet Transform}

Door de recursieve relatie van de ruimtes in \ref{ruimterec} herhaald toe te passen,
kunnen we de ruimte $V_j$ schrijven in termen van $V_0$ en een scala aan
orthogonaal-complementsruimtes $W_k$.
\begin{equation}
  \label{ruimte_splitsing}
  V_j = V_k \oplus W_k \oplus \cdots \oplus W_{j-1} = \ldots
  = V_0 \oplus W_0 \oplus \cdots \oplus W_{j-1}
\end{equation}
We willen nu een functie $f$ benaderen in de ruimte $V_j$ door $P_{V_j}f$ te schrijven in
de basis van de ruimtes $V_0$ en de verschillende $W_k$. Hiervoor dienen we de inproducten
te bereken van $f$ met de basisvectoren in deze ruimtes.
We onderscheiden dan de \emph{approximatie}
co\"efficienten $a_j[n]$ en de \emph{detail} co\"efficienten $d_j[n]$, die
$f$ geven als projectie op  respectievelijk $V_j$ en $W_j$.

Het is echter veel rekenwerk om al deze co\"efficienten uit te rekenen, dit gebeurt immers
door het berekenen van integralen. We beperken ons daarom tot her uit rekenen van de
co\"efficienten op het niveau $j$ en proberen vervolgens een recursieve relatie te vinden
om hieruit de approximatie en detailco\"efficienten te vinden.

Vanwege de relatie in \ref{ruimterec} kunnen we de basisfuncties uit de ruimtes $W_{j-1}$,
$V_{j-1}$ schrijven in termen van de basisfuncties van de ruimte $V_j$.
We schrijven daarvoor $\phi_{j-1,n}$ om in termen van de basisfuncties $\phi_{j,k}$; evenzo
voor $\psi_{j-1,n}$.
\begin{equation}
  \label{phi_rec}
  \phi_{j-1,n} = \sum_{k=-\infty}^{\infty} \inpr{\phi_{j-1,n}}{\phi_{j,k}} \phi_{j,k}
\end{equation}
\begin{equation}
  \label{psi_rec}
  \psi_{j-1,n} = \sum_{k=-\infty}^{\infty} \inpr{\psi_{j-1,n}}{\phi_{j,k}} \phi_{j,k}
\end{equation}
We rekenen vervolgens de inproducten uit om deze om te schrijven naar een filterco\"efficient,
bekijk
\eq{
  \inpr{\phi_{j-1,n}}{\phi_{j,k}}
  =& \int_{-\infty}^\infty \sqrt{2^{j-1}}\phi(2^{j-1}t -n)
  \sqrt{2^{j}}\phi^\star(2^j t - k) \d{t}\\
  =& \int_{-\infty}^\infty \tfrac{1}{2^j} 2^{j-1}
  \sqrt{2} \phi(t') \phi(2t' - k + 2n) \d{t'}\\
  =& \inpr{\phi(t)}{\sqrt{2}\phi(2t-k+2n)} \\
  =& h[k-2n],
}
\eq{
  \inpr{\psi_{j-1,n}}{\phi_{j,k}}
  =& \int_{-\infty}^\infty \sqrt{2^{j-1}}\psi(2^{j-1}t -n)
  \sqrt{2^{j}}\phi^\star(2^j t - k) \d{t}\\
  =& \int_{-\infty}^\infty \tfrac{1}{2^j} 2^{j-1}
  \sqrt{2} \psi(t') \phi(2t' - k + 2n) \d{t'}\\
  =& \inpr{\psi(t)}{\sqrt{2}\phi(2t-k+2n)} \\
  =& g[k-2n],
}
waarbij we de co\"ordinaat-transformatie $t' = 2^{j-1}t - n$ hebben gebruikt.
Door nu aan beide zijden van vergelijkingen \ref{phi_rec}, \ref{psi_rec} het
inproduct met $f$ te nemen kunnen we de coe\"fficienten van de resolutie $2^{j-1}$ schrijven
in termen van de coe\"fficienten op de resolutie $2^j$.
\begin{equation}
  \label{approx_rec}
  a_{j-1}[n] = \inpr{f}{\phi_{j-1,n}}
  = \sum_{k=-\infty}^{\infty} h[k-2n] \inpr{f}{\phi_{j,k}}
  = \sum_{k=-\infty}^\infty h[k-2n] a_{j}[k]
  = (a_j \star \bar h)[2n]
\end{equation}
\begin{equation}
  \label{detail_rec}
  d_{j-1}[n] = \inpr{f}{\psi_{j-1,n}}
  = \sum_{k=-\infty}^{\infty} g[k-2n] \inpr{f}{\phi_{j,k}}
  = \sum_{k=-\infty}^\infty g[k-2n] a_{j}[k]
  = (a_j \star \bar g)[2n]
\end{equation}
Waarbij $\bar f: x \mapsto f(-x)$ en $x\star y$ de discrete convolutie aangeeft,
als kortere schrijfwijze van deze sommatie.
De relatie die hier gevonden is geeft aanleiding tot een algoritme.

\begin{algo}[Fast Wavelet Transform]
  Gegeven een rij co\"effici\"enten $a_j\in\R^{2^j}$ dan definie\"eren we een recursief
  algoritme $\mathrm{FWT}:\R^{2^j}\to\R^{2^j}$ volgens:\\
  Als $j=0$ dan geldt
  \[
  \mathrm{FWT}(a_j)[n] = a_j[n]
  \]
  Als $j>0$ bereken $a_{j-1}$ en $d_{j-1}$ volgens \ref{approx_rec}, \ref{detail_rec}.
  Vervolgens geldt
  \begin{equation}
    \label{FWT_cases}
    \mathrm{FWT}(a_j)[n] = \begin{cases}
      \mathrm{FWT}(a_{j-1})[n] & \text{als } n\leq 2^{j-1} \\
      d_{j-1}[n] & \text{als } n>2^{j-1} \end{cases}
  \end{equation}
\end{algo}
We gaan hierbij uit van een eindige filter en beschouwen onze ruimte $V_j$ als functies
op een interval in $\R$, dan versimpelen de oneindige sommaties in \ref{approx_rec},
\ref{detail_rec}. Voor een discrete functie $f\in\R^{2^j}$ kunnen we nu
$\mathrm{FWT}(f)$ nemen als de fast wavelet transform,
we nemen dan aan dat $\inpr{f}{a_{j,k}} = f[k]$ voor het
grootste niveau $V_j$. Deze aannames komen ruwweg neer op de bewering dat
$V_0 = \{\phi_{0,0}\}$ en dat zowel $\phi$ als $\psi$ een compacte drager heeft.

We kunnen vervolgens de transformatie inverteren aan de hand van het volgende algoritme
\begin{algo}[Inverse Fast Wavelet Transform]
  Gegeven een rij co\"effici\"enten $x_j\in\R^{2^j}$ dan definie\"eren we een recursief
  algoritme $\mathrm{iFWT}:\R^{2^j}\to\R^{2^j}$ volgens:\\
  Als $j=0$ dan geldt
  \[
  \mathrm{iFWT}(x_j)[n] = x_j[n]
  \]
  Als $j>0$ laat $x_{j-1}[n] = x_j[n]$ en $d_{j-1}[n] = x_j[n+2^{j-1}]$ voor
  $1\leq n\leq 2^{j-1}$. Bereken hiermee $a_{j-1} = iFWT(x_{j-1})$,
  dan geldt
  \begin{equation}
    \label{reconstr_FWT}
    \mathrm{iFWT}(x_j)[n] = (\breve a_{j-1}\star h)[n] + (\breve d_{j-1}\star g)[n]
  \end{equation}
  Waarbij $\breve y [2n] = y[n]$ en $\breve y [2n+1] = 0$.
\end{algo}
\begin{stelling}
  De iFWT (links) samengesteld met de FWT geeft de identiteit voor een signaal.
\end{stelling}
\begin{proof}[Bewijs]
  We bekijken de verschillende gevallen.Voor het geval $j=0$ geldt duidelijk dat
  \[
  \mathrm{iFWT}(\mathrm{FWT}(a_0)) = \mathrm{iFWT}(a_0) = a_0.
  \]
  We zullen dus verder moeten bewijzen dat \ref{reconstr_FWT} een inverse vormt voor
  \ref{FWT_cases}.
  Schrijf hiervoor de basisfuncties van $V_j$ in termen van de basisfuncties van $V_{j-1}$
  en $W_{j-1}$, ofwel:
  \[
  \phi_{j,n} = \sum_{k=-\infty}^\infty \inpr{\phi_{j,n}}{\phi_{j-1,k}}\phi_{j-1,k}
  + \sum_{-\infty}^\infty \inpr{\phi_{j,n}}{\psi_{j-1,k}}\psi_{j-1,k}
  \]
  Deze inproducten komen weer overeen met de filterco\"efficienten, nemen we dus aan
  beide zeiden het inproduct dan volgt
  \[
  a_j[n] = \sum_{k=-\infty}^\infty h[n-2k]a_{j-1}[k]
  + \sum_{-\infty}^\infty g[n-2k]d_{j-1}[k]
  \]
  Door nu in de sommatie de variabele $k$ te vervangen door $k'=2k$ vereenvoudigt dit tot
  \[
  a_j[n] = \sum_{k'=-\infty}^\infty h[n-k']\breve a_{j-1}[k']
  + \sum_{-\infty}^\infty g[n-2k]\breve d_{j-1}[k']
  \]
  Daarmee geeft de iFWT een inverse voor de FWT.
\end{proof}
\section{Analyse van de Wavelettransformatie}
Met de theoretische beschouwing van wavelets en de Fast Wavelet Transform achter de rug, kunnen we wat verder kijken naar practische obstakels.

\subsection{Eindige signalen}
Een van de eerste aannames die we tot nu toe steeds maakten is die van de oneindige signalen. Wanneer echter de functie $f$ een compacte drager heeft, worden een aantal zaken wat lastiger. Neem als eerste aan dat de drager van $f$ gewoon $[0,1]$ is.\footnote{Door translatie en dilatie kan elk signaal met compacte basis omgevormd worden tot een signaal met drager in $[0,1]$. We verliezen hier dus geen algemeenheid.} In dit geval zou het kunnen dat de waveletfuncties met een drager die $t=0$ of $t=1$ doorsnijdt, niet meer de gewenste eigenschappen heeft. Er zijn in de literatuur oplossingen voor dit probleem gevonden. Hier zullen wij verder niet op in gaan.

Wanneer we nu een benadering van $f$ maken op resolutie $2^J$ (door bijvoorbeeld een Fast Wavelet Transform), bekijken we
\[
V_0 \subset V_{1} \subset \ldots \subset V_{J}.
\]
Een discreet signaal van lengte $2^{J}$ kan zo perfect getransformeerd worden in een waveletbasis op resolutie $2^{J}$. Dit is precies waarom de Fast Wavelet Transform zo veel gebruikt wordt bij het analyseren van discrete signalen.

\subsection{Signaaluitbreiding}
Een probleem waar we in het geval van eindige signalen nog meer mee te maken krijgen is dat de algoritme niet goed omgaat met de randen. De convolutie moet nu ineens \emph{buiten het definitiegebied} van het signaal `kijken'. Eerder in sectie \ref{signaal} hebben we al gezien hoe signalen naar een tweemacht uitgebreid kunnen worden. Precies dezelfde methoden kunnen gebruikt worden om het signaal nog verder uit te breiden.

Om niet te veel tijd te verliezen met het ondersteunen van meerdere mogelijkheden hebben wij er voor gekozen om \emph{periodic padding} op alle signalen toe te passen. Dit omdat de zogenaamde \emph{circulaire convolutie} ingebouwd zit in de bibliotheek die wij gebruikt hebben.

\subsection{Complexiteit van de algoritme}
NOOT: rob heeft hier "nog niet" staan. wat?
Als de lengte van de filter $h$ gelijk is aan $K$, en de lengte van het originele signaal $a_L$ gelijk is aan $N = 2^{L}$, kunnen we voor $j \in \{0, \ldots, L\}$ zien dat $a_j$ en $d_j$ beide $2^{j}$ elementen bevatten. Nu kunnen $a_{j-1}$ en $d_{j-1}$ gemaakt worden door $2^{j}K$ operaties zodat elke stap van de algoritme $2^{j} \cdot K$ operaties kost. Dan kost het hele algoritme
\[
\sum_{j=L}^0 2^{j} \cdot K = K \sum_{j=L}^0 2^{j} = K \cdot (2^{1+L} - 1) < 2 \cdot K 2^{L} = 2KN
\]
operaties. Dus deze DWT is een $\mathcal{O}(KN)$ algoritme. Ook de complexiteit van de inverse wordt op dezelfde manier van orde $KN$.

\section{Meer dimensies: de Mallatdecompositie}
Met een orthonormale waveletbasis $\{ \psi_{j,n}: (j,n) \in \Z^2\}$ van $L_2(\R)$ volgt een natuurlijke voortzetting naar twee dimensies door
\[
\{ \psi_{j_1,n_1}(x_1) \psi_{j_2,n_2}(x_2): (j_i,n_i) \in \Z^4 \},
\]
welke een orthonormale basis voor $L_2(\R^2)$ is. We zien direct dat we op de $x_1$-as met resolutie $2^{j_1}$ kijken terwijl de $x_2$-as resolutie $2^{j_2}$ kent.

Mallat vond dit iets om te vermijden \cite[\S 7.7]{mallat} en legt in zijn analyse dan ook de eis $j_1 = j_2 =: j$ op. Wij zullen in het vervolg \'o\'ok kijken naar het zogenaamde Tensorproduct, wat de eis $j_1 = j_2$ \emph{niet} oplegt.

Zoals in 1 dimensie is de notie van `resolutie' geformaliseerd in het begrip multiresolutie. De definitie van deze multiresolutie is wederom een natuurlijke voortzetting van het eendimensionale geval. Wanneer we spreken over een separeerbare multiresolutie, gaat het om een ruimte $V_j^{(2)} := V_j \otimes V_j$. In \cite[A.5]{mallat} wordt nu bewezen dat, gegeven een orthonormale basis $\{ \phi_{j,m}: m \in \Z \}$ voor $V_j$, de verzameling
\begin{equation}
  \label{phi_phi_basis}
  \{ \phi^{(2)}_{j,n} := \phi_{j,n_1} \otimes \phi_{j,n_2}: n \in \Z^2 \}
\end{equation}
een orthonormale basis voor $V_j^{(2)}$ is.

\begin{voorbeeld}
Bekijk weer $V_j$ de ruimte van stuksgewijs constante functies op het interval 
\[
 [2^{-j} m, 2^{-j}(m+1) ), m \in \Z.
\] 
We vinden voor $V_j^{(2)}$ nu de ruimte van stuksgewijs constante functies op vierkanten $[2^{-j}n_1, 2^{-j}(n_1+1)) \times [2^{-j}n_2, 2^{-j}(n_2+1))$. De tweedimensionale schalingsfunctie wordt op die manier
\[
	\phi^{(2)}(x_1,x_2) = \phi(x_1)\phi(x_2) = \begin{cases} 1 & \text{ als } x_1 \in [0,1)\text{ en }x_2 \in [0,1) \\ 0 & \text{ anders.} \end{cases}
\]
\end{voorbeeld}

\subsection{Tweedimensionale Waveletfuncties}
We weten dat $V_j^{(2)}$ bevat is in $V_{j+1}^{(2)}$. Bekijk het orthogonale complement $\boldsymbol{U}_j \perp V_j^{(2)}$:
\begin{equation}
  \label{2d_ruimte_rec}
  V_j^{(2)} \oplus \boldsymbol{U}_j = V_{j+1}^{(2)}.
\end{equation}
Om nu een orthogonale waveletbasis voor $\boldsymbol{U}_j$ en (dus in de limiet) $L^2(\R^2)$ te vinden, gaan we als volgt te werk.
\begin{stelling}[{\cite[T7.24]{mallat}}]
\label{mallatbasis}
  Laat $\phi$ een schalingsfunctie en $\psi$ de bijhorende wavelet die en basis voor de $L^2(\R)$ voortbrengt. Maak drie wavelets
  \begin{equation}
    \label{psi_k_defs}
    \psi^1(x) = \phi(x_1)\psi(x_2) \quad \psi^2(x) = \psi(x_1) \phi(x_2) \quad \psi^3(x) = \psi(x_1)\psi(x_2)
  \end{equation}
  en laat voor $k \in \{1,2,3\}$ nu
  \[
  \psi^k_{j,n}(x) = 2^j \psi^k\left( 2^jx_1 - n_1, 2^j x_2 - n_2 \right).
  \]

  Dan is
  \[
  \{ \psi^1_{j,n}, \psi^2_{j,n}, \psi^3_{j,n}: n \in \Z^2 \}
  \] een basis voor $\boldsymbol{U}_j$
  en
  \[
  \{ \phi_{0, n}^2: n \in \Z^2 \} \cup \{ \psi^1_{j,n}, \psi^2_{j,n}, \psi^3_{j,n}: n \in \Z^2, j \in \N_0 \}
  \] een basis voor $L^2(\R^2)$.
\end{stelling}
\begin{proof}[Bewijs]
  We weten
  \[
  V_{j+1}^{(2)} = V_j^{(2)} \oplus \boldsymbol{U}_j \implies V_{j+1} \otimes V_{j+1} = ( V_j \otimes V_j ) \oplus \boldsymbol{U}_j.
  \]
  Vul nu $V_{j+1} = V_j \oplus W_j$ in om te vinden
  \[
  ( V_j \oplus W_j ) \otimes (V_j \oplus W_j ) = (V_j \otimes V_j) \oplus \boldsymbol{U}_j
  \]
  \[
  \implies (V_j \otimes V_j) \oplus (V_j \otimes W_j) \oplus (W_j \otimes V_j) \oplus (W_j \otimes W_j) = (V_j \otimes V_j) \oplus \boldsymbol{U}_j
  \]
  \[
  \implies (V_j \otimes W_j) \oplus (W_j \otimes V_j) \oplus (W_j \otimes W_j) = \boldsymbol{U}_j.
  \]
  Nu is het duidelijk dat $\{ \psi^1_{j,n}, \psi^2_{j,n}, \psi^3_{j,n}: n \in \Z^2 \}$ een basis is voor $\boldsymbol{U}_j$.

Omdat nu via vergelijking \ref{ruimterec} moet gelden
\[
	L^2(\R^2) = \lim_{j \to \infty} V_j^{(2)} = \lim_{j \to \infty} ( V_0^{(2)} \oplus \boldsymbol{U}_0 \oplus \cdots \oplus \boldsymbol{U}_{j-1} ) = \left( \bigoplus_{j=0}^\infty \boldsymbol{U}_j \right) \oplus V_0^{(2)}
\]
is $\{ \phi_{0, n}^2: n \in \Z^2 \} \cup \{ \psi^1_{j,n}, \psi^2_{j,n}, \psi^3_{j,n}: n \in \Z^2, j \in \N_0 \}$ een basis voor $L^2(\R^2)$.
\end{proof}
\begin{gevolg}
In het bewijs wordt nu ook duidelijk dat 
\[
	\left\{ \phi_{0, n}^2: n \in \Z^2 \right\} \cup \left\{ \psi^1_{j,n}, \psi^2_{j,n}, \psi^3_{j,n}: n \in \Z^2, j \in \{0, \ldots, L \}  \right\}
\]
een basis voor $V_L^{(2)}$ is.
\end{gevolg}

Bovenstaande basis heeft dus schalingsfuncties op \'e\'en niveau en waveletfuncties op alle niveaus. Net zoals in het eendimensionale geval kunnen we met deze basis een algoritme formuleren.

\subsection{Naar een tweedimensionaal algoritme}
Nu we een relatie hebben gevonden $(\ref{2d_ruimte_rec})$ tussen de ruimtes op verschillende
niveau's volgens
\begin{equation}
  \label{2d_ruimte_decomp}
V_{j+1}^{(2)} = (V_j\otimes W_j) \oplus (W_j\otimes V_j) \oplus
(W_j\otimes W_j) \oplus (V_j\otimes V_j)
\end{equation}
kunnen we het eendimensionale algoritme uitbreiden naar twee dimensies.
Hiervoor kijken we wederom naar de inproducten van een functie $f$ met
onze basisfuncties.
We defini\"eren daarvoor weer de \emph{approximatie}~co\"effici\"ent en een drietal
van \emph{detail}~co\"effici\"enten volgens.
\[
a_j[n] := \langle f, \phi^{(2)}_{j,n} \rangle \quad d^k_j[n] := \langle f, \psi^k_{j,n} \rangle ,\quad k \in \{1,2,3\}.
\]
We zullen nu in het bijzonder kunnen zeggen (maak gebruik van $\ref{phi_phi_basis}$)
dat we de basis-functies op het niveau $j$ kunnen ontbinden volgens:
\begin{equation}
  \label{phi_phi_som}
  \phi^{(2)}_{j,(n_1,n_2)} = \sum_{k_1=-\infty}^\infty \sum_{k_2=-\infty}^\infty
  \inpr{\phi_{j,n_1}\otimes\phi_{j,n_2}}{\phi_{j+1,k_1}\otimes\phi_{j+1,k_2}}
  \phi^{(2)}_{j+1,(k_1,k_2)}
\end{equation}
\begin{equation}
  \label{psi_k_som}
  \psi^{p}_{j,(n_1,n_2)} = \sum_{k_1=-\infty}^\infty \sum_{k_2=-\infty}^\infty
  \inpr{\psi^p_{j,(n_1,n_2)}}{\phi_{j+1,k_1}\otimes\phi_{j+1,k_2}}
  \phi^{(2)}_{j+1,(k_1,k_2)}
\end{equation}
We maken nu gebruik van \ref{weidmann}[\S 3.4] om het inproduct horende bij het tensorproduct
van twee \emph{Hilbertruimten} uit te schrijven volgens:
\[
\inpr{a\otimes b}{c\otimes d} = \inpr{a}{c}\cdot \inpr{b}{d}
\]
Waarbij de inproducten lopen over de respectievelijke ruimtes.
Dit versimpelt de inproducten die we zoeken volgens:
\[
\inpr{\phi_{j,n_1}\otimes\phi_{j,n_2}}{\phi_{j+1,k_1}\otimes\phi_{j+1,k_2}}
=\inpr{\phi_{j,n_1}}{\phi_{j+1,k_1}}\inpr{\phi_{j,n_2}}{\phi_{j+1,k_2}}
=h[k_1-2n_1]\cdot h[k_2-2n_2]
\]
Waarbij we de filterrelatie voor \'e\'en dimensie toepassen. We kunnen dit nog een
stuk verbeteren door het product $h[\_]\cdot h[\_]$ om te schrijven naar een tensorproduct,
dan is:
\[
h\otimes h : \Z\times\Z \to \R :: (x_1,x_2) \mapsto h[x_1]\cdot h[x_2]
\]
Deze stappen kunnen nu ook met hetzelfde argument toegepast worden op de $\psi^k$'s,
we verkrijgen dan met een blik op de definities in ($\ref{psi_k_defs}$) de vergelijkingen:
\[
\inpr{\psi^1_{j,(n_1,n_2)}}{\phi^{(2)}_{j+1,(k_1,k_2)}} = (h\otimes g) [k_1-2n_1,k_2-2n_2]
\]
\[
\inpr{\psi^2_{j,(n_1,n_2)}}{\phi^{(2)}_{j+1,(k_1,k_2)}} = (g\otimes h) [k_1-2n_1,k_2-2n_2]
\]
\[
\inpr{\psi^3_{j,(n_1,n_2)}}{\phi^{(2)}_{j+1,(k_1,k_2)}} = (g\otimes g) [k_1-2n_1,k_2-2n_2]
\]
We richten ons nu weer op de \emph{approximatie} en \emph{detail}~co\"effici\"enten door
aan beide zeiden van de vergelijkingen $\ref{phi_phi_som}$,$\ref{psi_k_som}$ het inproduct
met $f$ te nemen. We kunnen dit aan de hand van onze nieuwe 2-dimensionale filters
opschrijven met een convolutie in twee dimensies, namelijk
\begin{eqnarray}
  \label{2d_coef_rec}
  a_{j}[n_1,n_2] =& (a_{j+1} \star (\bar{h} \otimes \bar{h}))[2n_1,2n_2] \\
  d^1_{j}[n_1,n_2] =&( a_{j+1} \star (\bar{h} \otimes \bar{g}))[2n_1,2n_2] \\
  d^2_{j}[n_1,n_2] =& (a_{j+1} \star (\bar{g} \otimes \bar{h}))[2n_1,2n_2] \\
  \label{2d_coef_rec_last}
  d^3_{j}[n_1,n_2] =& (a_{j+1} \star (\bar{g} \otimes \bar{g}))[2n_1,2n_2]
\end{eqnarray}
waarbij de convolutie gegeven wordt door
\[
(x \star y)[n_1,n_2] := \sum_{p_1=-\infty}^\infty \sum_{p_2 = -\infty}^\infty
x[n_1 - p_1,n_2 - p_2] \cdot y[n_1-p_1, n_2 - p_2].
\]


\begin{algo}[Tweedimensionale Fast Wavelet Transform]
  Gegeven een matrix van co\"effici\"enten $a_j\in\R^{2^j}\times\R^{2^j}$ dan definie\"eren
  we een recursief algoritme $\mathrm{FWT}_2:\R^{2^j}\times\R^{2^j}\to\R^{2^j}\times\R^{2^j}$
  volgens:\\
  Als $j=0$ dan geldt
  \[
  \mathrm{FWT}_2(a_j)[n_1,n_2] = a_j[n_1,n_2]
  \]
  Als $j>0$ bereken $a_{j-1}$,$d^1_{j-1}$,$d^2_{j-1}$ en $d^3_{j-1}$ uit $a_{j}$
  volgens $(\ref{2d_coef_rec}-\ref{2d_coef_rec_last})$
  Vervolgens geldt
  \begin{equation}
    \label{FWTd_def}
  \mathrm{FWT}_2(a_j)[n_1,n_2] = \begin{cases}
    \mathrm{FWT}_2(a_{j-1})[n_1,n_2] & \text{als } n_1 \leq 2^{j-1} \text{ en } n_2 \leq 2^{j-1}\\
    d^1_{j-1}[n_1,n_2]
    & \text{als } n_1\leq 2^{j-1} \text{ en } n_2>2^{j-1} \\
    d^2_{j-1}[n_1,n_2]
    & \text{als } n_1>2^{j-1} \text{ en } n_2\leq 2^{j-1} \\
    d^3_{j-1}[n_1,n_2] & \text{als } n_1>2^{j-1} \text{ en } n_2>2^{j-1} \end{cases}
  \end{equation}
\end{algo}
\begin{algo}[Inverse Tweedimensionale Fast Wavelet Transform]
  Gegeven een matrix van  co\"effici\"enten $x_j\in\R^{2^j}\times\R^{2^j}$ dan definie\"eren we hierop het recursieve
  algoritme $\mathrm{iFWT}_2:\R^{2^j}\times\R^{2^j}\to\R^{2^j}\times\R^{2^j}$ volgens:\\
  Als $j=0$ dan geldt
  \[
  \mathrm{iFWT}_2(x_j)[n_1,n_2] = x_j[n_1,n_2]
  \]
  Als $j>0$, splits dan $x_j$ in zijn vier kwadranten; laat voor $k_1,k_2\in \{1,\ldots,2^{j-1}\}$
  \begin{eqnarray*}
    y_{j-1}[k_1,k_2]   =& x_j[k_1,k_2] \\
    d^1_{j-1}[k_1,k_2] =& x_j[k_1,k_2+2^{j-1}] \\
    d^2_{j-1}[k_1,k_2] =& x_j[k_1+2^{j-1},k_2] \\
    d^3_{j-1}[k_1,k_2] =& x_j[k_1+2^{j-1},k_2+2^{j-1}]
  \end{eqnarray*}
  Bereken hiermee vervolgens $a_{j-1} = \mathrm{iFWT}_2(y_{j-1})$,
  dan geldt
  \begin{equation}
    \label{iFWTd_def}
    \begin{split}
      \mathrm{iFWT}_2(a_j)[n_1,n_2] =& \breve{a}_{j-1} \star (h \otimes h)[n_1,n_2] 
      + \breve{d}_{j-1}^1 \star (h \otimes g)[n_1,n_2] \\
      +& \breve{d}_{j-1}^2 \star (g \otimes h)[n_1,n_2] 
      + \breve{d}_{j-1}^3 \star (g \otimes g)[n_1,n_2]
    \end{split}
  \end{equation}
  Waarbij 
  \[
  \breve y [n_1,n_2] = \begin{cases} 
    y[n_1/2,n_2/2] & \text{als } 2|n_1 \text{ en } 2|n_2 \\ 
    0 &\text{anders}\end{cases}
  \]
\end{algo}
\begin{stelling}
  De $\mathrm{iFWT}_2$ (links) samengesteld met de $\mathrm{FWT}_2$ geeft de identiteit voor een signaal.
\end{stelling}
\begin{proof}[Bewijs]
  We bekijken de verschillende gevallen. Voor het geval $j=0$ geldt duidelijk dat
  \[
  \mathrm{iFWT}_2(\mathrm{FWT}_2(a_0)) = \mathrm{iFWT}_2(a_0) = a_0.
  \]
  Voor het geval $j>0$ merken we op dat wanneer de $\mathrm{iFWT}_2$ werkt voor $j'<j$ de co\"effici\"enten-matrices
  $a,d^1,d^2,d^3$ precies zijn wat de $\mathrm{FWT}_2$ zou geven op dit niveau. 
  We zullen dus verder moeten bewijzen dat \ref{iFWTd_def} een inverse vormt voor
  \ref{FWTd_def}.
  Schrijf hiervoor de basisfuncties van $V^{(2)}_j$ in termen van de basisfuncties van 
  $V_{j-1}\otimes V_{j-1}$, $V_{j-1}\otimes W_{j-1}$, $W_{j-1}\otimes V_{j-1}$ en $W_{j-1}\otimes W_{j-1}$
  dit is geoorloofd zoals we gezien hebben in een vorige sectie.
  \begin{equation*}
    \begin{split}
      \phi_{j,(n_1,n_2)} = 
      \sum_{k_1=-\infty}^\infty\sum_{k_2=-\infty}^\infty 
      \inpr{\phi^{(2)}_{j,(n_1,n_2)}}{\phi^{(2)}_{j-1,(k_1,k_2)}} \phi^{(2)}_{j-1,(k_1,k_2)} \\
      + \sum_{k_1=-\infty}^\infty\sum_{k_2=-\infty}^\infty 
      \inpr{\phi^{(2)}_{j,(n_1,n_2)}}{\psi^1_{j-1,(k_1,k_2)}} \psi^1_{j-1,(k_1,k_2)} \\
      + \sum_{k_1=-\infty}^\infty\sum_{k_2=-\infty}^\infty 
      \inpr{\phi^{(2)}_{j,(n_1,n_2)}}{\psi^2_{j-1,(k_1,k_2)}} \psi^2_{j-1,(k_1,k_2)} \\
      + \sum_{k_1=-\infty}^\infty\sum_{k_2=-\infty}^\infty 
      \inpr{\phi^{(2)}_{j,(n_1,n_2)}}{\psi^3_{j-1,(k_1,k_2)}} \psi^3_{j-1,(k_1,k_2)}
      \end{split}
  \end{equation*}
  Deze inproducten tussen basisfuncties komen weer overeen met de filterco\"efficienten. Nemen we dus aan
  beide zeiden het inproduct met $f$ dan volgt de vergelijking voor de approximatie-co\"effici\"enten
  \begin{equation*}
    \begin{split}
      a_{j}[n_1,n_2] = 
      \sum_{k_1=-\infty}^\infty\sum_{k_2=-\infty}^\infty 
      (h\otimes h)[n_1-2k_1,n_2-2k_2] a_{j-1}[k_1,k_2] \\
      + \sum_{k_1=-\infty}^\infty\sum_{k_2=-\infty}^\infty 
      (h\otimes g)[n_1-2k_1,n_2-2k_2] d^1_{j-1}[k_1,k_2] \\
      + \sum_{k_1=-\infty}^\infty\sum_{k_2=-\infty}^\infty 
      (g\otimes h)[n_1-2k_1,n_2-2k_2] d^2_{j-1}[k_1,k_2] \\
      + \sum_{k_1=-\infty}^\infty\sum_{k_2=-\infty}^\infty 
      (g\otimes g)[n_1-2k_1,n_2-2k_2] d^3_{j-1}[k_1,k_2]
      \end{split}
  \end{equation*}
  Door nu in de sommatie de variabelen $k_1,k_2$ te vervangen door $k_1'=2k_1$ respectievelijk $k_2'=2k_2$ 
  vereenvoudigt dit tot
  \begin{equation*}
    \begin{split}
      a_{j}[n_1,n_2] = 
      \sum_{k_1'=-\infty}^\infty\sum_{k_2'=-\infty}^\infty 
      (h\otimes h)[n_1-k_1',n_2-k_2'] \breve a_{j-1}[k_1',k_2'] \\
      + \sum_{k_1'=-\infty}^\infty\sum_{k_2'=-\infty}^\infty 
      (h\otimes g)[n_1-k_1',n_2-k_2'] \breve d^1_{j-1}[k_1',k_2'] \\
      + \sum_{k_1'=-\infty}^\infty\sum_{k_2'=-\infty}^\infty 
      (g\otimes h)[n_1-k_1',n_2-k_2'] \breve d^2_{j-1}[k_1',k_2'] \\
      + \sum_{k_1'=-\infty}^\infty\sum_{k_2'=-\infty}^\infty 
      (g\otimes g)[n_1-k_1',n_2-k_2'] \breve d^3_{j-1}[k_1',k_2']
      \end{split}
  \end{equation*}
  Wat precies gelijk is aan de convolutie in \ref{iFWTd_def}. Daarmee geeft de iFWT een inverse voor de FWT.
\end{proof}

\subsection{Meer dan twee dimensies}
Het $n$-dimensionale geval van de Mallatdecompositie is nog TODOOO

\subsection{Eindige signalen in $n$ dimensies}
De notie van eindige signalen is al eerder langsgekomen. We bekijken functies met een compacte drager $[0,1]$. Het gevolg hiervan is dat de complete waveletbasis teveel elementen bevat. Alleen wavelets waarvan de drager het interval doorsnijdt, zijn voor ons interessant. Wanneer we nu naar meer dimensies gaan, praten we over een $n$-dimensionaal eenheidsinterval $[0,1]^n =: \Box$. 

\iffalse
Zoals in een eerdere sectie in 1 dimensie al aan werd gegeven, kijken we in de praktijk vaak naar signalen met een compacte drager, zeg $\Box := [0,1]^n$. Dan focussen we ons dus op functies $f \in L_2(\Box)$ en niet meer op functies in $L_2(\R^n)$. Net zoals eerder komen er problemen voor wavelets `op de rand'. Wij zullen hier wederom geen verdere aandacht aan besteden.

Wanneer we $L_2(\Box)$ als deelruimte van $L_2(\R^n)$ beschouwen, is er een natuurlijke basis voor deze deelruimte te vinden. Neem namelijk alle basisfuncties met een drager die $\Box$ doorsnijdt (de rest is op $\Box$ namelijk gewoon de nulfunctie). Definieer $\nabla$ als de verzameling indices $\lambda := (j, n)$ van wavelets die drager doorsnijden met $[0,1]$. Dan wordt $\Psi := \{ \psi_\lambda: \lambda \in \nabla \}$ een basis voor $L_2([0,1])$. Definieer daarnaast $|\lambda| := |(j,n)| = j$ als het niveau.

NOOT: dit stukje is poep.


Concreet zullen we echter niet gebruik maken van signalen die leven in $L^2(\R^n)$. We zullen eerder op zoek zijn naar de Wavelettransformatie van een signaal met compacte drager. Neem dus aan dat $f$ leeft in $L^2([0,1]^n)$. Dit $n$-dimensionale eenheidsinterval wordt ook wel met $\Box$ aangegeven. Omdat $\Box \subset \R^n$, kunnen we een waveletbasis voor $L^2(\R^n)$ ook als basis nemen voor $L^2(\Box)$. Maar eigenlijk is dit iets te veel (gezien het feit dat veel basisfuncties hun drager geheel buiten het interval zullen hebben). Daarom wordt de verzameling van indices $\lambda := (j,n)$ van wavelets die drager doorsnijden met $[0,1]$ ook wel $\nabla$ genoemd. Dus $\Psi = \{ \psi_\lambda: \lambda \in \nabla \}$ wordt nu een basis voor $L^2([0,1])$. Defini\"eer $|\lambda| = |(j,n)| = j$ als het niveau.
Verder zullen we vanaf nu aannemen dat $\boldsymbol\psi$ een compacte drager heeft (zoals we in de praktijk altijd willen) en van orde $p$ is.
\fi

\section{Tensorproduct}
Herinner dat we voor de Mallat-decompositie gebruik maakte van de gelijkheid (zie \ref{2d_ruimte_decomp}):
\[
V_{j}^{(2)} = (V_{j-1}\otimes W_{j-1}) \oplus (W_{j-1}\otimes V_{j-1}) \oplus
(W_{j-1}\otimes W_{j-1}) \oplus (V^{(2)}_{j-1})
\]
Door dit herhaald toe te passen kregen we een basis van de vorm:
\[
\{\phi^{(2)}_{0,(n_11,n_2)}\largediv n_1,n_2\in\Z\}\cup
\{\psi^k_{j,(n_1,n_2)} \largediv k=1,2,3\quad j\in \N_0 \quad n_1,n_2\in\Z\}
\]
We zullen nu echter zien dat er ook een andere manier is om deze ruimte te decomposeren.
Bedenk dat we in 1 dimensie de decompositie zoals in \ref{ruimte_splitsing} gebruiken volgens:
\[
V_j = V_0 \oplus W_0 \oplus \cdots \oplus W_{j-1}
\]
Daarmee schrijven we de ruimte $V^{(2)}_j$ als het tensorproduct van $V_j$ met zichzelf:
\[
V^{(2)}_j = V_j\otimes V_j = (V_0\oplus W_0\oplus\cdots W_{j-1})\otimes(V_0\oplus W_0\oplus\cdots W_{j-1})
\]
Dit geeft aanleiding tot een nieuwe basis, namelijk het tensorproduct van de basis van $V_j$ met zichzelf,
we duiden deze basis dan ook aan als de \emph{Tensorbasis}.
\footnote{In de literatuur wordt de Mallatdecompositie ook regelmatig een Tensorproduct genoemd. 
De verwarring ontstaat hier doordat in de Mallat Basis de basisfuncties ook tensorproducten zijn van wavelet- 
en schalingsfuncties. We doelen echter in onze naamgeving op \emph{het tensorproduct van twee bases}.}
\begin{equation}
  \label{tensor_basis_def}
  \begin{split}
    \boldsymbol \Psi_T :=& (\{\phi_{0,n}\largediv n\in\Z\}\cup \{\psi_{j,n}\largediv j\in\N_0\quad n\in\Z\})^2\\
                        =& \{\phi_{0,n_1}\otimes\phi_{0,n_2}\largediv n_1,n_2\in\Z\} 
                        \cup \{\phi_{0,n_1}\otimes\psi_{j,n_2}\largediv j\in\N_0\quad n_1,n_2\in\Z\} \\    
                         & \cup \{\psi_{j,n_1}\otimes\phi_{0,n_2}\largediv j\in\N_0\quad n_1,n_2\in\Z\} \\
                         & \cup \{\psi_{j_1,n_1}\otimes\psi_{j_2,n_2}\largediv j_1,j_2\in\N_0\quad n_1,n_2\in\Z\} \\
  \end{split}
\end{equation}
Waar $A^2 = \{a\otimes b \largediv a,b\in A\}$ een tensorproduct van een basis met zichzelf geeft.

Een karakteristiek van de Mallatbasis is, dat deze $\psi \otimes \psi$ functies bevat van enkel dezelfde schaal,
bij de Tensorbasis hebben we nu $j_1$ en $j_2$ die ongelijk aan elkaar zijn.
De Mallatbasis moest door deze eis aangevuld worden met functies van de vorm $\phi\otimes\psi$ en $\psi\otimes\phi$,
dit is bij de Tensorbasis niet meer aan de orde (met uitzondering van enkele samenstellingen van $\phi_0$ met $\psi_j$'s)

We willen nu een algoritme bedenken dat een tweedimensionaal schrijft in termen van de Tensorbasis,
we zullen zien dat dit neerkomt op het uitvoeren van FWT op alle rijen en kolommen van de ingangssignaal (een matrix)

\begin{algo}[Tweedimensionale Tensor Fast Wavelet Transform]
Gegeven een ingangssignaal $a_j\in\R^{2^j}\times\R^{2^j}$ dan defini\"eren we een bijbehorend sequentieel algoritme 
$\mathrm{TFWT}_2:\R^{2^j}\times\R^{2^j}\to \R^{2^j}\times\R^{2^j}$ door het schema:\\
Bereken $\tilde a_j$ zo dat geldt:
\[
\tilde a_j[n_1,n_2] = \mathrm{FWT}(a_j\largediv_{x_1=n_1})[n_2] 
\]
Dan wordt de $\mathrm{TFWT}_2$ gegeven door:
\[
\mathrm{TFWT}_2(a_j)[n_1,n_2] = \mathrm{FWT}(\tilde a_j\largediv_{x_2=n_2})[n_1]
\] 
Waarbij de notatie $a\largediv_{x_i=c}$ staat voor de vector die verkregen wordt door uit $a$
de $i$'de co\"ordinaat vast te zetten (e.g. $a\largediv_{x_1=c}[d] = a[c,d]$).
\end{algo}

We zullen aantonen dat de $\mathrm{TFWT}_2$ ook inderdaad een decompositie geeft in termen van de Tensorbasis.
We zijn namelijk op zoek naar een matrix van de vorm:
\begin{equation}
\label{TFWT_mat}
\begin{bmatrix}
\inpr{f}{\phi_0\otimes\phi_0}     & \inpr{f}{\phi_0\otimes\psi_0}     & \cdots & \inpr{f}{\phi_0\otimes\psi_\lambda} \\
\inpr{f}{\psi_0\otimes\phi_0}     & \inpr{f}{\psi_0\otimes\psi_0}     & \cdots & \inpr{f}{\psi_0\otimes\psi_\lambda} \\ 
           \vdots                 &         \vdots                    & \ddots &                \vdots  \\ 
\inpr{f}{\psi_\lambda\otimes\phi_0} & \inpr{f}{\psi_\lambda\otimes\psi_0} 
& \cdots & \inpr{f}{\psi_\lambda\otimes\psi_\lambda} \\
\end{bmatrix}
\end{equation}
Waarbij $\lambda$ een index geeft over de interessante basisfuncties.
We weten nu dat wanneer we de FWT nemen in \'e\'en co\"ordinaat door de andere co\"ordinaat vast te nemen,
 dit ons voor elke vaste waarde een vector geeft volgens:
\[
y \mapsto
\begin{bmatrix}
\inpr{f\largediv_{x_2=y}}{\phi_0} & 
\inpr{f\largediv_{x_2=y}}{\psi_0} & \cdots & 
\inpr{f\largediv_{x_2=y}}{\psi_\lambda}
\end{bmatrix}
\]
Hieruit kunnen we ook een vector van functies maken door de rol van functie en matrix-index om te wisselen:
\[
\begin{bmatrix}
y \mapsto \inpr{f\largediv_{x_2=y}}{\phi_0} & 
y \mapsto \inpr{f\largediv_{x_2=y}}{\psi_0} & \cdots & 
y \mapsto \inpr{f\largediv_{x_2=y}}{\psi_\lambda}
\end{bmatrix}
\]
Wanneer we nu voor elke functie in deze vector de FWT bereken zullen we weer een vector inproducten krijgen
voor elke co\"ordinaat in de originele vector, schrijf deze vector van vectoren (equivalent met een matrix) uit als 
$M[\lambda_1,\lambda_2]$,
\footnote{Hier wordt het inproduct van $\phi_0$ weggelaten om de schrijfwijze te verduidelijken.}
dan volgt:
\[
M[\lambda_1,\lambda_2] = \inpr{y\mapsto \inpr{f\largediv_{x_2=y}}{\psi_{\lambda_1}}}{\psi_{\lambda_2}}
\]
We kunnen dit omschrijven door de definitie van het inproduct op functieruimten toe te passen:
\[
\inpr{y\mapsto \inpr{f\largediv_{x_2=y}}{a}}{b} = 
\int_{-\infty}^\infty\int_{-\infty}^\infty f(x_1,x_2) \cdot a(x_1) \d{x_1} \cdot b(x_2) \d{x_2} = \inpr{f}{a\otimes b} 
\]
Waardoor de matrix $M$ identiek is aan \ref{TFWT_mat}. 

Het vinden van een inverse voor dit algoritme is nu ook triviaal, de iFWT geeft samengesteld met de FWT
immers de identiteit, zodat het toepassen op kolommen en rijen van de iFWT de $\mathrm{TFWT}_2$ inverteert.

\subsection{Meer dan twee dimensies}
Er is nu een natuurlijke voortzetting van het Tensorproduct naar meerdimensionale signalen.
Dit doen we namelijk door voor bijvoorbeeld $n$ dimensies als basis voor $V^{(n)}_j$ de basis te kiezen die we verkrijgen 
door de basis van $V_j$ met tensorproducten tot de $n$-de macht te verheffen.
Vervolgens is het algoritme aan te passen zodat het steeds over \'e\'en co\"ordinaat de FWT uitvoert, 
met bijbehorende inverse.

Omdat de stap in complexiteit van de notatie vele male groter is dan de benodigde denkstap laten we hier een rigoreuze
behandeling van de meerdimensionale Tensor Fast Wavelet Transform achterwege.

\subsection{Mengvormen van Tensor en Mallat}

TODO :D:D:D:D:D:D

\section{Analyse van de fout van beide decomposities}

Bij compressie is men op zoek naar een manier om stukjes data weg te kunnen gooien of te schrijven op zo'n manier dat het minder ruimte inneemt. Wij zijn in het bijzonder ge\"interesseerd naar \emph{hoe dichtbij} we bij perfecte reconstructie zitten wanneer we een vooraf bepaald aantal data opslaan. Er is al uitgebreid onderzoek gedaan naar hoe dit werkt bij de waveletbasis en een aantal resultaten hiervan zullen we opnemen in ons verslag.

Laat $f$ een functie in $L^2(\Box)$ met aftelbare orthonormale basis $\mathcal{B} = \{ g_m \}$. Dan valt $f$ in deze basis te schrijven als
\[
f = \sum_{m = 0}^\infty \langle f, g_m \rangle g_m.
\]

\begin{lemm}[Parsevalgelijkheid\cite{parseval}]
  Als nu een functie $f$ in $L^2(\Box)$ geschreven wordt in $\mathcal{B}$ dan geldt
  \[
  \|f\|^2 = \sum_{m=1}^\infty | \langle f, g_m \rangle |^2.
  \]
\end{lemm}
\begin{proof}[Bewijs]
  TODO: bewijs skippen, reference er in?
  We hebben te maken met een Hilbertruimte dus
  \[
  \|f\|^2 = \langle f, f \rangle = \left\langle \sum_{m=1}^\infty \langle f, g_m \rangle g_m, \sum_{n=0}^\infty \langle f, g_n \rangle g_n \right\rangle = \sum_{m=1}^\infty \sum_{n=0}^\infty \left\langle \langle f, g_m \rangle g_m, \langle f, g_n \rangle g_n \right \rangle
  \]
  \[
  = \sum_{m=1}^\infty \sum_{n=1}^\infty \langle f, g_m \rangle \overline{\langle f, g_n \rangle}\langle g_m, g_n \rangle = \sum_{m=1}^\infty \sum_{n=1}^\infty \langle f, g_m \rangle \overline{\langle f, g_n \rangle} \delta_{m,n}
  \]
  \[ = \sum_{m=1}^\infty \langle f, g_m \rangle \overline{\langle f, g_m \rangle} = \sum_{m=1}^\infty |\langle f, g_m \rangle |^2.
  \]
\end{proof}

\begin{gevolg}
\label{linfout}
Wanneer we nu niet de hele basis, maar zeg alleen de eerste $N$ elementen pakken, krijgen we een verzameling $\mathcal{B}_N \subset \mathcal{B}$ zodat
\[
f_{\mathcal{B}_N} := \sum_{m = 1}^N \langle f, g_m \rangle g_m.
\]

In het bijzonder zijn we nu op zoek naar de \emph{fout} $\| f - f_{\mathcal{B}_N} \|$:
\[
\| f - f_{\mathcal{B}_N} \|^2 = \left\| \sum_{m=1}^\infty\langle f, g_m \rangle g_m - \sum_{m=1}^N \langle f, g_m \rangle g_m \right\|^2 = \left\| \sum_{m=N+1}^\infty\langle f, g_m \rangle g_m \right\|^2 = \sum_{m=N+1}^\infty | \langle f, g_m \rangle |^2.
\]
Duidelijk moge zijn dat voor $N \to \infty$, $\| f - f_{\mathcal{B}_N} \|^2 \to 0$.
\end{gevolg}

\begin{definitie}[Sobolevruimte]
Een Sobolevruimte $H^p(\Omega)$ over $\Omega \subset \R^n$ is de verzameling van alle functies $u \in L_2(\Omega)$ z\'o dat voor elke $\alpha \in \R^n$ geldt dat $|\alpha| \leq p \implies D^\alpha u \in L_2(\Omega)$. De zwakke parti\"ele afgeleide $D^\alpha u$ betekent
\[
	D^\alpha u := \frac{\partial^{|\alpha|} u}{\partial x_1^{\alpha_1} \cdots \partial x_n^{\alpha_n} }.
\]

De norm op $H^p$ wordt nu
\[
	\| u \|_{H^p(\Omega)} := \sum_{|\alpha| \leq k} \| D^\alpha u \|_{L_2(\Omega)}.
\]
\end{definitie}

\subsection{Fout van de Mallatdecompositie}
Wij zijn op het moment ge\"interesseerd in de Mallat-waveletbasis $\boldsymbol\Phi$ die we vonden in stelling \ref{mallatbasis}. Deze is ook duidelijk aftelbaar dus we kunnen gevolg~\ref{linfout} gebruiken. 

Defini\"eer $J_M := \{ l \in \N^n: \| l \|_\infty \leq M \}$. Laat $\boldsymbol\Phi_M := \{ \boldsymbol{\psi}_{\boldsymbol{\lambda}} \in \boldsymbol\Phi: |\boldsymbol\lambda| \in J_M \}$ met $|\boldsymbol\lambda| = (|\lambda_1, \ldots, |\lambda_n|)$ de verzameling basisfuncties tot een niveau $M$.

\begin{stelling}[Fout van Mallatdecompositie]
Wanneer $f \in H^p(\Box)$, zal de fout $\| f - f_{\mathcal{\boldsymbol\Phi}_M} \|$ bij een Mallatdecompositie met de basisfuncties tot niveau $M$ hoogstens proportioneel met $N^{-p/n}$ zijn, met $N := \# \boldsymbol\Phi_M$ het aantal basisfuncties tot niveau $M$.
\end{stelling}
\begin{proof}

  We maken gebruik van de zogenaamde Jacksonongelijkheid \cite{jackson} die zegt dat 
  \[
  \inf_{q \in \mathbb{P}_{p-1}} ||f - q||_{L_2(\Box)} \simeq 2^{-jp} ||f||_{H^p(\Box)}
  \]
  wanneer $f \in H^p(\Box)$.

  Voor elke dimensie zitten er $\mathcal{O}(2^M)$ basisfuncties in $\{ \psi_\lambda: |\lambda| \leq M \}$. Er zijn $n$ dimensies dus $2^{Mn} = N$ basisfuncties in totaal.

  Bekijk de fout:
  \[
  \left\| f - f_{\boldsymbol\Phi_M} \right\|^2_{L_2(\Box)} = \sum_{{\boldsymbol\psi} \in \boldsymbol\Phi \setminus \boldsymbol\Phi_M} | \langle f, \boldsymbol\psi \rangle |^2 \simeq \sum_{|\boldsymbol\lambda| > M} 2^{-|\boldsymbol\lambda|p} \simeq 2^{-Mp},
  \]
  waarbij het laatste isteken voortkomt uit
  \[
  \sum_{k=M+1}^\infty 2^{- kp} = \frac{2^{-Mp}}{2^p-1}
  \]
  en de notie dat $p$ constant is voor een keuze van de wavelet. De fout is nu $2^{-Mp/2} \simeq 2^{-Mp}$. Omschrijven geeft dat dit overeenkomt met een fout van $N^{-p/n}$.
\end{proof}

\subsection{Fout van het Tensorproduct}
We bekijken een aftelbare basis van $L_2(\Box)$. In 1 dimensie wordt deze basis gevonden door $\{ \psi_\lambda: \lambda \in \nabla \}$, waarbij $\nabla$ precies de indexverzameling van deze basis is.

Volgens \cite[L3.1.7]{tammo} is 
\[ 
  \boldsymbol\Psi = \Psi \otimes \cdots \otimes \Psi = \{ \boldsymbol{\psi_\lambda} := \psi_{\lambda_1} \otimes \cdots \otimes \psi_{\lambda_n}: \lambda_i \in \nabla \}
\]
met $\boldsymbol\lambda := (\lambda_1, \ldots, \lambda_n) \in \boldsymbol{\nabla} := \nabla^n$ nu een orthogonale basis voor $L^2(\Box)$.

Laat vervolgens $I_M := \{ l \in \N^n_0: ||l||_1 \leq M \}$ en maak de \emph{sparse grid index set} $\boldsymbol{\nabla}_M := \{ \boldsymbol{(j,n)} \in \boldsymbol{\nabla}: \boldsymbol{j} \in I_M \}$.

\begin{lemm}{\cite[P3.2.3]{tammo}}
  Voor $f \in H^p(\Box)$ geldt dat de fout van de benadering op basis van de sparse grid index set $\boldsymbol{\nabla}_M$ hoogstens voldoet aan
  \[
  \left\| f - f_{\boldsymbol\nabla_M} \right\|_{L_2(\Box)} \lesssim 2^{-pM} M^{(n-1)/2} \| f \|_{H^p(\Box)}
  \]
\end{lemm}

Het aantal elementen in deze verzameling $\boldsymbol{\nabla}_M$ nu, kunnen we vinden.
\begin{lemm}{\cite[L3.3.1]{tammo}}
  Het aantal elementen in $\boldsymbol{\nabla}_M$ is proportioneel met $2^M M^{n-1}$.
\end{lemm}

\begin{lemm}
  Wanneer er voor twee functies $f, g$ geldt dat $f(J) \lesssim J^{-p}\log_2(J)^\mu$ en $g(J) = \log_2(J)^\nu J =: N$, dan
  \[
  (f \circ g^{-1})(N) \lesssim N^{-p} \log_2{N}^{\mu + \nu p}.
  \]
  TODO: bewijs klopt (nog) niet
\end{lemm}
\iffalse
\begin{proof}[Bewijs]
  We weten dat $g(J) = N$ dus $g^{-1}(N) = J$. Dan
  \[
  (f \circ g^{-1})(N) = f(J) \simeq J^{-p}\log_2(J)^{\mu}.
  \]
  Omdat verder $J \log_2(J)^\nu = N$, geldt $J^{-1} = N^{-1}\log_2(J)^\nu$. Vul dit in in bovenstaande om te krijgen
  \[
  f(J) \simeq N^{-s} \log_2(J)^{\nu s} \log_2(J)^\mu = N^{-s} \log_2(J)^{\nu s + \mu}????
  \]
\end{proof}
\fi

Met bovenstaande drie lemma's is het nu mogelijk een goede afschatting te maken.
\begin{stelling}[Fout van het Tensorproduct]
  Laat $f \in H^p(\Box)$ en $N = \#\boldsymbol{\nabla}_M$. Dan:
  \[
  \left\| f - f_{\boldsymbol\nabla_M} \right\|_{L_2(\Box)} \lesssim N^{-p} \log_2(N)^{(n-1)(1/2 + p)} \| f \|_{H^p(\Box)}.
  \]
\end{stelling}
\begin{proof}
  Gebruik het tweede lemma om te vinden dat $N \simeq M^{n-1}2^M$. Nu vinden we via het eerste lemma dat
  \[
  \left\| f - f_{\boldsymbol\nabla_M} \right\|_{L_2(\Box)} \lesssim M^{(n-1)/2}2^{-Mp}\| f \|_{H^p(\Box)}
  \]
  zodat
  \[
  \frac{\left\| f - f_{\boldsymbol\nabla_M}  \right\|_{L_2(\Box)}}{\| f \|_{H^p(\Box)}} \lesssim M^{(n-1)/2}2^{-Mp}.
  \]

  We willen graag lemma drie toepassen. Door
  \[
  J^{-p}\log_2(J)^\mu = M^{(n-1)/2} 2^{-Mp}
  \]
  volgt $J = 2^M$ en $\mu = (n-1)/2$. Door
  \[
  N = M^{n-1}2^M = \log_2(J)^\nu J = 2^M M^\nu
  \] volgt $\nu = n-1$. TODO: wat is $f$ en $g$?
\end{proof}

\subsection{Vergelijkend}
We vinden dus dat voor een voldoend gladde $f$ dat (gebruikend een beperkte hoeveelheid basisfuncties) de Mallatdecompositie hoogstens een convergentiesnelheid $N^{-p/n}$ bereikt, terwijl het Tensorproduct een snelheid $N^{-p} \log_2(N)^{(n-1)(p+1/2)}$ bereikt. De zogenaamde \emph{curse of dimensionality} kan dus verbroken worden door het gebruik van een Tensorproduct.


In de praktijk hebben we echter nooit te maken met compleet gladde functies. Gevolg is dat deze stellingen niet helemaal opgaan. Niet \emph{helemaal}, omdat per constructie van onze wavelet, de basis compact is en dus `zoomen we in' op de functie. Lokaal is de mogelijkheid dat $f$ glad genoeg is ineens een stuk meer in zicht. Het gevolg is wel dat je op zo'n moment waarschijnlijk lokaal een hoger niveau wil gebruiken. De bewijzen van hierboven zijn op basis van een \emph{niet-adaptieve} deelverzameling, dat wil zeggen dat ze geen rekening houden met lokaal een hoger niveau.

Hoewel de vorige zin misschien klinkt alsof er roet in ons eten gegooid wordt, is het in de praktijk toch goed mogelijk om de gevolgen te zien. Dit zullen we zien wanneer we de twee decomposities zullen gaan vergelijken.



\chapter{Onze implementie}

Een discreet signaal in 1 dimensie kan over het algemeen geschreven worden als een vector re\"ele getallen van een bepaalde lengte (de signaallengte). Wanneer we plaatjes willen beschrijven, bekijken we daarom vaak matrices. Dit is een matrix van grijswaardes, waarbij vier kanalen (rood, groen, blauw, doorzichtigheid) elk een eigen matrix hebben die bij elkaar de representatie geven van een kleurenplaatje.

Omdat het menselijk oog niet heel nauwkeurig is, en door de limitaties van computers, slaan we de matrixco\"effici\"enten niet op als re\"ele getallen. In plaats daarvan worden vaak integerwaardes tussen 0 en 255 gebruikt. Omdat elk van de 3 kleurenkanalen zo 256 waardes krijgt, is het kleurenpallet toch groot. In elk van onze implementaties worden beelden gescheiden in haar vier kanalen en wordt op elk kanaal afzonderlijk het algoritme toegpast. De reden voor deze keuze is dat de verschillende kanalen niet veel met elkaar van doen hoeven hebben.

Bewegend beeld is zo op precies dezelfde manier te zien als een driedimensionale matrix.

\section{Implementatie van Fouriertransformatie in Python}
Met de theoretische basis uit voorgaande secties is de implementatie van het 
\emph{Fast Fourier Transform}-algoritme nu aan bod gekomen. 
In deze sectie zullen we uitweiden over enkele genomen hordes en gemaakte keuzes, de resultaten volgen
in een latere sectie.

\subsection{Fast Fourier Transform Implementatie}
De Pythoncode die gebruikt is om het FFT-algoritme in te programmeren volgt precies het schema van de eerder gegeven pseudocode.

\begin{lstlisting}[caption={FFT algoritme in Python, voert de pseudocode uit zoals in sectie~\ref{fft_sec}}]
def FFT( xs ):
    N = len(xs)
    if N <= 1:                  # randconditie
        return xs
    else:
        even = FFT(xs[0::2])    # voer FFT uit op even indices
        odd  = FFT(xs[1::2])    # voer FFT uit op oneven indices

        return [0.5*(even[k] + exp(-2j*pi*k/N)*odd[k]) for k in range(N/2)] + 
               [0.5*(even[k] - exp(-2j*pi*k/N)*odd[k]) for k in range(N/2)]
\end{lstlisting}as

Hierbij dient uitgelegd te worden dat de notatie $[x \text{ for } y] + [z \text{ for } w]$ twee lijsten construeert en achter elkaar zet. Dit algoritme werkt -- zoals eerder beschreven -- enkel op signalen waarvan de lengte
$N$ een macht van $2$ is. We hebben daarom signaalextensie toe moeten passen door middel van zero-padding
om het programma ook op andersvormige signalen te laten werken.

\begin{lstlisting}[caption={Zero-padding algoritme in Python, voegt nullen toe tot een tweemacht is bereikt}]
def Zero_Padding( xs ):
    N_old = len(xs)
    N_new = 2**ceil(log(N_old,2))         # rond logaritme af voor kleinste tweemacht
    return [xs[k] if (k < N_old) else 0 for k in range(N_new)]
\end{lstlisting}

Vervolgens hebben we deze code toegepast in twee dimensies. We maken hier gebruik van de definitie van het
MFFT-algoritme, dat grofweg zegt dat een meerdimensionaal algoritme kan worden geconstrueerd door herhaald 
het 1-dimensionale geval toe te passen. Voor 2 dimensies in het bijzonder betekent dit dat we simpelweg 
FFT konden toepassen op rijen en kolommen.

\begin{lstlisting}[caption=2-Dimensionaal FFT algoritme]
def FFT_2D( xss ):
    xss = map(FFT, xss)        # voer FFT uit op rijen

    xss_t = transpose(xss)     # verwissel rijen met kolommen
    xss_t = map(FFT, xss_t)    # voer FFT uit op kolommen
    xss   = transpose(xss_t)   # maak verwisseling ongedaan 

    return xss
\end{lstlisting}
Hier is de Pythonfunctie \texttt{map} gebruikt die ruwweg gedefinieerd is als 
\[
	\texttt{map}(f,x) = [f(y) \text{ for } y \text{ in } x]
\]
Met het oog op duidelijkheid is hier de Zero-Padding fase weggelaten, 
dit algoritme verwacht nu nog een $2^n \times 2^m$ matrix.
Dit is echter gemakkelijk te implementeren door Zero-Padding toe te passen op rijen en kolommen.

\subsection{Compressie}
Zoals in de introductie van ons verslag al aan bod gekomen is, hebben we compressie bereikt door de kleinste 
co\"effici\"enten weg te gooien en aan de slag te gaan met een kleinere lijst. 
Dit brengt twee problemen met zich mee: (1) welke waarde nemen we als \emph{cutoff}, 
en (2) hoe slaan we de overgebleven co\"effici\"enten op?
\subsubsection{Cutoff}
Hoewel ons eerste programma een voorgegeven cutoff verwachtte, wilden we liever op compressieratio selecteren.
Daarom hebben latere versies de volgende routine gebruikt. 
\begin{lstlisting}[caption=Het vinden van een goede cutoff-waarde gegeven een gewenst compressieniveau]
def find_cutoff( mat, ratio ):
	if ratio < 0 or ratio > 1: return
	length = reduce( lambda x, y: x*y, array.shape ) #aantal cellen in de matrix 
	vector = reshape(array, length) 				 #reshape naar 1 lange vector
	sort(modulus(vector)) 							 #sorteer de vector
	return vector[ int( ratio*len(vector) ) ]
\end{lstlisting}
We sorteren hier de co\"effici\"enten in \texttt{mat} op grootte en kiezen de cutoff als de co\"effici\"ent
die de lijst opdeelt ten opzichte van de \texttt{ratio}.

\subsubsection{Opslaan van de co\"effici\"enten}
Voor het opslaan van de co\"effici\"enten hebben we een adaptieve basis gebruikt
we moesten daarom bijgehouden \emph{welke} co\"effici\"enten er opgeslagen worden. 
Hier hebben wij een feature van Python gebruikt, de \emph{dictionary}.
Een dictionary in Python is niks meer dan een lijst van (naam, waarde)-paren. 
De naam is in ons geval de plek in de matrix en de waarde is het (complexe) getal wat in die cel hoort te staan. 
Dit geeft aanleiding tot het schrijven van twee routines \texttt{dict2mat} en \texttt{mat2dict}.
\begin{lstlisting}[caption=Matrix naar dictionary conversie]
def mat2dict( mat, cutoff ):
	N, M = mat.shape
	dict = {}
	for y in range(N):
		for x in range(M):
			if modulus( mat[y][x] ) >= cutoff:
				dict[(y,x)] = mat[y][x]
	return (dict, N, M)
\end{lstlisting}

\begin{lstlisting}[caption=Dictionary naar matrix conversie]
def dict2mat( dict, N, M ):
	mat = []
	for y in range( N ):
		row = []
		for x in range( M ):
			i = (y, x)
			row[x] = dict[i] if i in dict else 0.0
		mat[y] = row
	return mat
\end{lstlisting}

Duidelijk moge zijn dat we, door deze twee functies na elkaar uit te voeren, een matrix terugkrijgen met nullen waar eerst kleine co\"effici\"enten stonden.

\subsection{Het .wvg-beeldformaat}
Hoewel we een eigen beeldformaat gemaakt hebben, is het ons niet gelukt om de plaatjes echt in minder ruimte op te slaan dan hun ongecomprimeerde versie. Dit heeft met twee dingen te maken. Ten eerste is de manier waarop wij de dictionaries opgeslagen hebben niet optimaal geweest.\footnote{Later bedachten we nog een manier die ongeveer 50\% van de oorspronkelijke ruimte in zou nemen, door niet (index, waarde)-paren op te slaan maar met een zogenaamde bitmap.} Ten tweede is de Fouriertransformatie een complex algoritme in de zin dat zij complexe co\"effici\"enten teruggeeft. Door het isomorfisme $\C \simeq \R^2$ moesten wij zodoende twee waardes voor elke co\"effici\"ent opslaan.

\section{Wavelets in 1 dimensie}
Een orthogonale wavelet wordt discreet eigenlijk helemaal gekenmerkt door haar filter $h$. Zoals beschreven in de sectie over de FWT, hebben we daarmee (via vergelijkingen~\ref{highpassfilter}, \ref{approx_rec}, \ref{detail_rec}, \ref{reconstr_FWT}) eigenlijk het algoritme al helemaal in handen. Definieer
\[
	\texttt{dec\_l} := \bar{h} \quad \texttt{dec\_h} := \bar{g} \quad 
        \texttt{rec\_l} := h; \quad \texttt{rec\_h} := g.
\]
Dan kunnen we met twee routines in elke richting het algoritme simpel implementeren.
\begin{lstlisting}[caption=De FWT]
def next(wavelet, signal):
	low = downsampling_convolution( signal, wavelet.dec_l )
	hi = downsampling_convolution( signal, wavelet.dec_h )
	return concatenate(low, hi)
	
def fwt(wavelet, signal):
	signal = zero_pad(signal)
	steps = int( log( len(signal), 2 ) ) #neem de 2-log van de lengte
	
	for i in range( steps ):
		k = len(output) / 2**i #pak eerst alles, dan de helft, dan een kwart etc.
		signal[0:k] = next( signal[0:k], wavelet )
	return signal
\end{lstlisting}
\begin{lstlisting}[caption=De iFWT]
def prev(wavelet, signal):
	N = len(signal)
	low = upsampling_convolution( signal[:N//2], wavelet.rec_l )
	hi = upsampling_convolution( signal[N//2:], wavelet.rec_h )
	return  sum(low, hi) 	#elementsgewijze sommatie
	
def ifwt( wavelet, signal, original_length ):
	steps = int( log( len(signal), 2) ) #neem de 2-log van de lengte
	
	for i in range( steps ):
		j = steps - i - 1 #we gaan andersom
		k = len(signal) / 2**j
		signal[0:k] = prev( signal[0:k], wavelet)
	return signal[0:original_length] #het signaal was aangevuld tot een tweemacht

\end{lstlisting}

\section{Wavelets in twee dimensies}
Zoals eerder beschreven is, kunnen we in meer dimensies eigenlijk twee kanten op: de Mallatdecompositie en het Tensorproduct.

\subsection{Mallatdecompositie}
De Mallatdecompositie is in de implementatie eigenlijk heel simpel. We gebruiken nog steeds de eendimensionale \texttt{next} en \texttt{prev} routines door deze bij elke stap op rijen en kolommen uit te voeren, dit is equivalent
met een tweedimensionale convolutie.
Omdat we hier in twee richtingen kunnen werken, moet het signaal aangevuld worden tot een vierkante tweemacht.\footnote{Het is in theorie mogelijk om een rechthoek te transformeren maar deze abstractie heeft weinig practisch nut.}

\begin{lstlisting}[caption=De Mallatdecompositie in 2 dimensies]
def next_2d( wavelet, signal ):
	x, y = signal.shape
	for j in range(x): #1d-fwt op de rijen
		signal[j, :] = next( wavelet, signal[j, :] )
	for i in range(y): #1d-fwt op de kolommen
		signal[:, i] = next( wavelet, signal[:, i] )
	return signal

def fwt2d(wavelet, signal):
	signal = zero_pad_2d(signal)
	steps = int( log( len(signal), 2 ) ) #neem de 2-log van de lengte
	
	for i in range( steps ):
		k = len(output) / 2**i #pak eerst alles, dan de helft, dan een kwart etc.
		signal[0:k,0:k] = next_2d( signal[0:k,0:k], wavelet )
	return signal
\end{lstlisting}
\begin{lstlisting}[caption=De omgekeerde Mallatdecompositie in 2 dimensies]
def prev_2d( wavelet, signal ):
	x, y = signal.shape
	for j in range(x): #1d-ifwt op de rijen
		signal[j, :] = prev( wavelet, signal[j, :] )
	for i in range(y): #1d-ifwt op de kolommen
		signal[:, i] = prev( wavelet, signal[:, i] )
	return signal

def ifwt2d( wavelet, signal, N, M ): #N en M zijn originele lengte en breedte
	steps = int( log( len(signal), 2) ) #neem de 2-log van de lengte
	
	for i in range( steps ):
		j = steps - i - 1 #we gaan andersom
		k = len(signal) / 2**j
		signal[0:k,0:k] = prev_2d( signal[0:k,0:k], wavelet)
	return signal[0:N,0:M]
\end{lstlisting}

\subsection{Tensorproduct}
Het Tensorproduct in meer dimensies gaat eigenlijk net zoals de Fouriertransformatie in meer dimensies, door op rijen en kolommen het algoritme in 1 dimensie toe te passen.
\begin{lstlisting}[caption=De 2d FWT in het Tensorgeval]
def fwt2d_tensor(wavelet, signal):
	signal = zero_pad(signal)
	steps = int( log( len(signal), 2 ) ) #neem de 2-log van de lengte
	
	for i in range( steps ):
		k = len(output) / 2**i #pak eerst alles, dan de helft, dan een kwart etc.
		for p in range( len( signal ) ):
			signal[p, 0:k] = next( signal[p, 0:k], wavelet )
		for q in range( len( signal ) ):
			signal[0:k, q] = next( signal[0:k, q], wavelet )
	return signal
\end{lstlisting}
\begin{lstlisting}[caption=De 2d iFWT in het Tensorgeval]
def ifwt2d_tensor( wavelet, signal, original_length ):
	steps = int( log( len(signal), 2) ) #neem de 2-log van de lengte
	
	for i in range( steps ):
		j = steps - i - 1 #we gaan andersom
		k = len(signal) / 2**j
		for q in range( len( signal ) ):
			signal[0:k, q] = prev( signal[0:k, q], wavelet )
		for p in range( len( signal ) ):
			signal[p, 0:k] = prev( signal[p, 0:k], wavelet )
	return signal[0:original_length] #het signaal was aangevuld tot een tweemacht
\end{lstlisting}

\section{Hogere dimensies}
In hogere dimensies wordt het nog interessanter. In bijvoorbeeld drie dimensies kunnen we naast de Mallatdecompositie en het Tensorproduct, ook een \emph{mengvorm} toepassen. Wij hebben er voor gekozen om in dit geval de Mallatdecompositie in het $x,y$-vlak te gebruiken, en het Tensorproduct in de $z$-richting hier loodrecht op.

\subsection{Mallatdecompositie}
De Mallatdecompositie is werkelijk echt niet zo interessant, maar omdat we overal al stappen over het meerdimensionale geval, here goes:
\begin{lstlisting}[caption=De Mallatdecompositie in 3 dimensies]
def next_3d( wavelet, signal ):
	x, y, z = signal.shape
	for j in range(x): #2d-fwt op de rijen
		signal[j, :, :] = next_2d( wavelet, signal[j, :, :] )
	for k in range(y):
		for l in range(z): #1d-fwt op de kolommen
			signal[:,k,l] = next( wavelet, signal[:,k,l] )
	return signal

def fwt3d(wavelet, signal):
	signal = zero_pad_3d(signal)
	steps = int( log( len(signal), 2 ) ) #neem de 2-log van de lengte in 1 richting
	
	for i in range( steps ):
		k = len(output) / 2**i #pak eerst alles, dan de helft, dan een kwart etc.
		signal[0:k,0:k,0:k] = next_3d( signal[0:k,0:k,0:k], wavelet )
	return signal
\end{lstlisting}
\begin{lstlisting}[caption=De omgekeerde Mallatdecompositie in 3 dimensies]
def prev_3d( wavelet, signal ):
	x, y, z = signal.shape
	for j in range(x): #2d-ifwt op de rijen
		signal[j, :, :] = prev_2d( wavelet, signal[j, :, :] )
	for k in range(y):
		for l in range(z): #1d-ifwt op de kolommen
			signal[:,k,l] = prev( wavelet, signal[:,k,l] )
	return signal

def ifwt3d( wavelet, signal, N, M, O ): #N, M en O zijn originele lengte, breedte en diepte
	steps = int( log( len(signal), 2) )
	
	for i in range( steps ):
		j = steps - i - 1 #we gaan andersom
		k = len(signal) / 2**j
		signal[0:k,0:k,0:k] = prev_3d( signal[0:k,0:k,0:k], wavelet)
	return signal[0:N,0:M,0:O]
\end{lstlisting}

\subsection{Mengvorm}
\begin{lstlisting}[caption=De mengvorm in 3 dimensies]
def fwt3d_mix(wavelet, signal):
	signal = zero_pad_3d(signal)
	steps = int( log( len(signal), 2 ) ) #neem de 2-log van de lengte in 1 richting
	
	for i in range( steps ):
		m = len(output) / 2**i #pak eerst alles, dan de helft, dan een kwart etc.
		for j in range(len( signal )): #2d-fwt op de rijen
			signal[j, 0:k, 0:k] = next_2d( wavelet, signal[j, 0:k, 0:k] )
		for k in range(len( signal )):
			for l in range(len( signal )): #1d-fwt op de kolommen
				signal[0:k,m,l] = next( wavelet, signal[0:k,m,l] )

	return signal
\end{lstlisting}
\begin{lstlisting}[caption=De omgekeerde mengvorm in 3 dimensies]
def ifwt3d_mix( wavelet, signal, N, M, O ): #N, M en O zijn originele lengte, breedte en diepte
	steps = int( log( len(signal), 2) )
	
	for i in range( steps ):
		j = steps - i - 1 #we gaan andersom
		m = len(output) / 2**j #pak eerst alles, dan de helft, dan een kwart etc.
		for n in range(len( signal )): #2d-fwt op de rijen
			signal[n, 0:k, 0:k] = prev_2d( wavelet, signal[n, 0:k, 0:k] )
		for k in range(len( signal )):
			for l in range(len( signal )): #1d-fwt op de kolommen
				signal[0:k,m,l] = prev( wavelet, signal[0:k,m,l] )
	return signal[0:N,0:M,0:O]
\end{lstlisting}


\chapter{Resultaten}
In dit hoofdstuk zullen we een kwantitatieve vergelijking tussen de verschillende algoritmes maken. Hierbij zullen we het begrip \emph{Peak Signal To Noise Ratio} gebruiken. Dit is een veelgebruikte methode om de reconstructiequaliteit van een lossy compressie-algoritme te testen.

\begin{definitie}[Peak Signal To Noise Ratio (PSNR)]
  Gegeven een signaal $f: \Z^k \to \R^k$ en een reconstructie $\hat f$, is de $PSNR$ gelijk aan
  \[
  PSNR(f) := 20 \cdot \log_{10}( M ) - 10 \cdot \log_{10}(S),
  \]
  waarbij
  \[
  M := \max( \max_{\boldsymbol j \in \Z^k} (f[\boldsymbol k]), \max_{\boldsymbol j \in \Z^k} (\hat f[\boldsymbol k]))
  \]
  en
  \[
  S := \sum_{\boldsymbol j \in \Z^k} (f[\boldsymbol j] - \hat f[\boldsymbol j])^2.
  \]

  Hoe hoger de PSNR is, hoe beter de reconstructie (over het algemeen) zal zijn.
\end{definitie}

\pagebreak
\newgeometry{left=2cm,right=2cm,top=1cm,bottom=1cm}
\begin{figure}
  \centering
  \begin{subfigure}[b]{0.24\textwidth}
    \centering
    \includegraphics[width=\textwidth]{plaatjes/gentoo_fourier_0_15.png}
  \end{subfigure}
  \begin{subfigure}[b]{0.24\textwidth}
    \centering
    \includegraphics[width=\textwidth]{plaatjes/gentoo_fourier_0_1.png}
  \end{subfigure}
  \begin{subfigure}[b]{0.24\textwidth}
    \centering
    \includegraphics[width=\textwidth]{plaatjes/gentoo_fourier_0_05.png}
  \end{subfigure}
  \begin{subfigure}[b]{0.24\textwidth}
    \centering
    \includegraphics[width=\textwidth]{plaatjes/gentoo_fourier_0_01.png}
  \end{subfigure}
  \caption{Fourier op compressieniveaus 0.15, 0.10, 0.05, 0.01.}
\end{figure}
\begin{figure}
  \centering
  \begin{subfigure}[b]{0.24\textwidth}
    \centering
    \includegraphics[width=\textwidth]{plaatjes/gentoo_haar_0_15.png}
  \end{subfigure}
  \begin{subfigure}[b]{0.24\textwidth}
    \centering
    \includegraphics[width=\textwidth]{plaatjes/gentoo_haar_0_1.png}
  \end{subfigure}
  \begin{subfigure}[b]{0.24\textwidth}
    \centering
    \includegraphics[width=\textwidth]{plaatjes/gentoo_haar_0_05.png}
  \end{subfigure}
  \begin{subfigure}[b]{0.24\textwidth}
    \centering
    \includegraphics[width=\textwidth]{plaatjes/gentoo_haar_0_01.png}
  \end{subfigure}
  \caption{Haar op compressieniveaus 0.15, 0.10, 0.05, 0.01.}
\end{figure}
\begin{figure}
  \centering
  \begin{subfigure}[b]{0.24\textwidth}
    \centering
    \includegraphics[width=\textwidth]{plaatjes/gentoo_db2_0_15.png}
  \end{subfigure}
  \begin{subfigure}[b]{0.24\textwidth}
    \centering
    \includegraphics[width=\textwidth]{plaatjes/gentoo_db2_0_1.png}
  \end{subfigure}
  \begin{subfigure}[b]{0.24\textwidth}
    \centering
    \includegraphics[width=\textwidth]{plaatjes/gentoo_db2_0_05.png}
  \end{subfigure}
  \begin{subfigure}[b]{0.24\textwidth}
    \centering
    \includegraphics[width=\textwidth]{plaatjes/gentoo_db2_0_01.png}
  \end{subfigure}
  \caption{Daubechies 2 op compressieniveaus 0.15, 0.10, 0.05, 0.01.}
\end{figure}
\begin{figure}
  \centering
  \begin{subfigure}[t]{0.48\textwidth}
    \centering
    \vspace{10pt}
    \begingroup

    \renewcommand*{\arraystretch}{1.5}
    $\begin{array}{c | c c c c}
      \text{Compr} & \text{Fourier} & \text{Haar} & \text{DB2} \\ \hline
      0.150 & -21.564987 & 1.508043 & -0.325757 \\
      0.125 & -22.220604 & -3.607729 & -4.753701 \\
      0.100 & -22.974077 & -7.518578 & -8.964313 \\
      0.075 & -23.893526 & -11.909545 & -13.080699 \\
      0.050 & -25.134504 & -17.028756 & -17.498265 \\
      0.040 & -25.786567 & -19.471026 & -19.515331 \\
      0.030 & -26.597406 & -21.936183 & -21.791275 \\
      0.020 & -27.693270 & -25.283096 & -24.377904 \\
      0.010 & -29.532364 & -28.763393 & -27.638580 \\ \hline
    \end{array}$
    \endgroup
  \end{subfigure}
  \begin{subfigure}[t]{0.48\textwidth}
    \centering
    \vspace{0pt}
    \includegraphics[height=\textwidth]{plaatjes/grafiek_gentoo_0_15-0_01.png}
  \end{subfigure}
  \caption{Grafiek en PSNR.}
\end{figure}
\restoregeometry
\pagebreak


\pagebreak
\newgeometry{left=2cm,right=2cm,top=1cm,bottom=1cm}
\begin{figure}
  \centering
  \begin{subfigure}[b]{0.24\textwidth}
    \centering
    \includegraphics[width=\textwidth]{plaatjes/Lenna_fourier_0_1.png}
  \end{subfigure}
  \begin{subfigure}[b]{0.24\textwidth}
    \centering
    \includegraphics[width=\textwidth]{plaatjes/Lenna_fourier_0_05.png}
  \end{subfigure}
  \begin{subfigure}[b]{0.24\textwidth}
    \centering
    \includegraphics[width=\textwidth]{plaatjes/Lenna_fourier_0_03.png}
  \end{subfigure}
  \begin{subfigure}[b]{0.24\textwidth}
    \centering
    \includegraphics[width=\textwidth]{plaatjes/Lenna_fourier_0_01.png}
  \end{subfigure}
  \caption{Fourier op compressieniveaus 0.10, 0.05, 0.03, 0.01.}
\end{figure}
\begin{figure}
  \centering
  \begin{subfigure}[b]{0.24\textwidth}
    \centering
    \includegraphics[width=\textwidth]{plaatjes/Lenna_haar_0_1.png}
  \end{subfigure}
  \begin{subfigure}[b]{0.24\textwidth}
    \centering
    \includegraphics[width=\textwidth]{plaatjes/Lenna_haar_0_05.png}
  \end{subfigure}
  \begin{subfigure}[b]{0.24\textwidth}
    \centering
    \includegraphics[width=\textwidth]{plaatjes/Lenna_haar_0_03.png}
  \end{subfigure}
  \begin{subfigure}[b]{0.24\textwidth}
    \centering
    \includegraphics[width=\textwidth]{plaatjes/Lenna_haar_0_01.png}
  \end{subfigure}
  \caption{Haar op compressieniveaus 0.10, 0.05, 0.03, 0.01.}
\end{figure}
\begin{figure}
  \centering
  \begin{subfigure}[b]{0.24\textwidth}
    \centering
    \includegraphics[width=\textwidth]{plaatjes/Lenna_db2_0_1.png}
  \end{subfigure}
  \begin{subfigure}[b]{0.24\textwidth}
    \centering
    \includegraphics[width=\textwidth]{plaatjes/Lenna_db2_0_05.png}
  \end{subfigure}
  \begin{subfigure}[b]{0.24\textwidth}
    \centering
    \includegraphics[width=\textwidth]{plaatjes/Lenna_db2_0_03.png}
  \end{subfigure}
  \begin{subfigure}[b]{0.24\textwidth}
    \centering
    \includegraphics[width=\textwidth]{plaatjes/Lenna_db2_0_01.png}
  \end{subfigure}
  \caption{Daubechies 2 op compressieniveaus 0.10, 0.05, 0.05, 0.01.}
\end{figure}
\begin{figure}
  \centering
  \begin{subfigure}[t]{0.48\textwidth}
    \centering
    \vspace{10pt}
    \begingroup

    \renewcommand*{\arraystretch}{1.5}
    $\begin{array}{c | c c c c}
      \text{Compr} & \text{Fourier} & \text{Haar} & \text{DB2} \\ \hline
      0.200 & -21.043701 & -16.443978 & -15.615197 \\
      0.100 & -23.867860 & -20.863769 & -20.134351 \\
      0.050 & -25.825943 & -24.400495 & -23.745103 \\
      0.040 & -26.362758 & -25.303459 & -24.684358 \\
      0.030 & -27.024421 & -26.407813 & -25.757180 \\
      0.020 & -27.928360 & -27.754692 & -27.139561 \\
      0.010 & -29.380617 & -29.651147 & -29.111878 \\ \hline
    \end{array}$
    \endgroup
  \end{subfigure}
  \begin{subfigure}[t]{0.48\textwidth}
    \centering
    \vspace{0pt}
    \includegraphics[height=\textwidth]{plaatjes/grafiek_Lenna_0_15-0_01.png}
  \end{subfigure}
  \caption{Grafiek en PSNR.}
\end{figure}
\restoregeometry
\pagebreak

\pagebreak
\newgeometry{left=2cm,right=2cm,top=1cm,bottom=1cm}
\begin{figure}
  \centering
  \begin{subfigure}[b]{0.24\textwidth}
    \centering
    \includegraphics[width=\textwidth]{plaatjes/shyguy_fourier_0_1.png}
  \end{subfigure}
  \begin{subfigure}[b]{0.24\textwidth}
    \centering
    \includegraphics[width=\textwidth]{plaatjes/shyguy_fourier_0_05.png}
  \end{subfigure}
  \begin{subfigure}[b]{0.24\textwidth}
    \centering
    \includegraphics[width=\textwidth]{plaatjes/shyguy_fourier_0_03.png}
  \end{subfigure}
  \begin{subfigure}[b]{0.24\textwidth}
    \centering
    \includegraphics[width=\textwidth]{plaatjes/shyguy_fourier_0_01.png}
  \end{subfigure}
  \caption{Fourier op compressieniveaus 0.10, 0.05, 0.03, 0.01.}
\end{figure}
\begin{figure}
  \centering
  \begin{subfigure}[b]{0.24\textwidth}
    \centering
    \includegraphics[width=\textwidth]{plaatjes/shyguy_haar_0_1.png}
  \end{subfigure}
  \begin{subfigure}[b]{0.24\textwidth}
    \centering
    \includegraphics[width=\textwidth]{plaatjes/shyguy_haar_0_05.png}
  \end{subfigure}
  \begin{subfigure}[b]{0.24\textwidth}
    \centering
    \includegraphics[width=\textwidth]{plaatjes/shyguy_haar_0_03.png}
  \end{subfigure}
  \begin{subfigure}[b]{0.24\textwidth}
    \centering
    \includegraphics[width=\textwidth]{plaatjes/shyguy_haar_0_01.png}
  \end{subfigure}
  \caption{Haar op compressieniveaus 0.10, 0.05, 0.03, 0.01.}
\end{figure}
\begin{figure}
  \centering
  \begin{subfigure}[b]{0.24\textwidth}
    \centering
    \includegraphics[width=\textwidth]{plaatjes/shyguy_db2_0_1.png}
  \end{subfigure}
  \begin{subfigure}[b]{0.24\textwidth}
    \centering
    \includegraphics[width=\textwidth]{plaatjes/shyguy_db2_0_05.png}
  \end{subfigure}
  \begin{subfigure}[b]{0.24\textwidth}
    \centering
    \includegraphics[width=\textwidth]{plaatjes/shyguy_db2_0_03.png}
  \end{subfigure}
  \begin{subfigure}[b]{0.24\textwidth}
    \centering
    \includegraphics[width=\textwidth]{plaatjes/shyguy_db2_0_01.png}
  \end{subfigure}
  \caption{Daubechies 2 op compressieniveaus 0.10, 0.05, 0.05, 0.01.}
\end{figure}
\begin{figure}
  \centering
  \begin{subfigure}[t]{0.48\textwidth}
    \centering
    \vspace{10pt}
    \begingroup

    \renewcommand*{\arraystretch}{1.5}
    $\begin{array}{c | c c c c}
      \text{Compr} & \text{Fourier} & \text{Haar} & \text{DB2} \\ \hline
      0.125 & -25.709419 & 6.188436 & -12.939796 \\
      0.100 & -26.352690 & -2.252695 & -16.684428 \\
      0.075 & -27.084570 & -13.247047 & -20.121718 \\
      0.050 & -27.987835 & -22.194021 & -23.640705 \\
      0.040 & -28.455719 & -25.861219 & -25.259163 \\
      0.030 & -29.045916 & -26.633403 & -26.750137 \\
      0.020 & -29.857696 & -28.858775 & -28.445793 \\
      0.010 & -31.289113 & -31.320495 & -30.669577 \\ \hline
    \end{array}$
    \endgroup
  \end{subfigure}
  \begin{subfigure}[t]{0.48\textwidth}
    \centering
    \vspace{0pt}
    \includegraphics[height=\textwidth]{plaatjes/grafiek_shyguy_0_15-0_01.png}
  \end{subfigure}
  \caption{Grafiek en PSNR.}
\end{figure}
\restoregeometry
\pagebreak



\chapter{Discussie van de resultaten}
\label{discH}
In dit laatste hoofdstuk zullen we een analyse maken van het praktische deel van ons project.
We beginnen met een discussie van de gebruikte methoden en bruikbaarheid van de gevonden resultaten.
Verder worden de resultaten besproken voor het comprimeren van afbeeldingen, het toepassen van het 
Tensorproduct en als laatste het gebruik van de Tensor/Mallat mengvorm op filmmateriaal. 

\section{Discussie}
\subsection{Gebruik van PSNR als maatstaf voor kwaliteit}
Omdat we te maken hebben met lossy beeldcompressie dienen we dit verlies in kaart te brengen op een 
objectieve en kwantitatieve manier. 
De PSNR is een gebruikelijke grootheid voor dit soort analyses: er wordt gezegd dat dit voor eenzelfde 
beginsignaal een goede reflectie geeft van de menselijke beleving van de kwaliteit van de reconstructie.
Uit onderzoek \cite{PSNR} blijkt dat de PSNR een goede maatstaf is zolang er in zeker zin \emph{ceteris paribus}
is aangehouden; voor hetzelfde beeldmateriaal presteert deze maat goed.

\iffalse
\subsection{Bruikbaarheid van resultaten}

De resultaten beslaan slechts een kleine set beelmateriaal: dit heeft vooral met tijdgebrek te maken gehad.
We zullen geen uitspraak doen over de specifieke beeldcategorie\"en uit hoofdstuk \ref{testjes}.
omdat hier simpelweg te weinig studie naar is verricht maar we menen dat de set beelden gevarieerd 
genoeg is om algemene uitspraken te doen over de prestatie van de verschillende algoritmes.
\fi

\subsection{Adaptieve basis}
\label{adaptief_parseval}
We hebben in onze implementatie een adaptieve basis gebruikt: de compressie-algoritme behoudt co\"effici\"enten
die in absolute waarde het grootst zijn. 
We beweren dat dit de beste reconstructie geeft, preciezer: de totale kwadratische fout is zo het kleinst.
We beroepen ons daarvoor op de Parsevalgelijkheid. Wanneer we een selectie $S$ doen van basisfuncties wordt de totale fout na reconstructie
\[
\|f-f|_{S}\|^2_{L_2} = \sum_{k\not\in S} |\inpr{f}{k}|^2 = \sum_{k\not\in S} |c_k|^2,
\]
waar $c_k$ de co\"efficienten is van de basisfunctie $k$. 
Deze sommatie is nu het kleinst wanneer alle termen in de sommatie zo klein mogelijk zijn. 

\section{Compressie van afbeeldingen}

\subsection{Prestatie van Fourier vs. Wavelets}
Uit de PSNR-grafieken blijkt dat de Fouriertransformatie een slechtere reconstructie levert voor 
dezelfde dataset dan zowel de Haar- als de Daubechies-2-wavelettransformaties, 
ongeacht de categorie waartoe het beeld behoort.

\subsection{Wavelet vs. Wavelet}
Tussen de twee wavelets zien we verder wat we theoretisch verwachten: Haar presteert goed bij harde randen
vanwege zijn kleine drager terwijl Daubechies voor vloeiend/fotografisch materiaal een goede reconstructie geeft.

\subsection{Convergentie bij kleine dataset}
Uit de grafiek blijkt dat het verschil in kwaliteit tussen de verschillende compressie algoritmes afneemt wanneer
de hoeveelheid data die opgeslagen wordt, kleiner wordt. 
Door de kleine dataset is de willekeur van de vorm van het plaatje erg van invloed op kwaliteit van de reconstructie 
voor verschillende algoritmes. 
Ook is het dubieus of de PSNR nog een goede maatstaf is voor de menselijke perceptie wanneer de algoritmes
zo'n slechte maar zeer verschillende benadering geven van het originele beeld.
We zullen daarom hier geen uitspraak over kunnen doen.

\section{Het Tensorproduct op afbeeldingen}
Voor de volledigheid hebben we er voor gekozen \'o\'ok het Tensorproduct te bekijken bij een aantal 
van onze afbeeldingen: zie figuren \ref{fig:tensor_start}-\ref{fig:tensor_end}. We hebben er voor 
gekozen om niet alle afbeeldingen in dit formaat weer te geven: deze twee afbeeldingen geven een goede representatie.

Het eerste wat op te merken valt, zijn de horizontale en verticale balken die bij het Tensorproduct 
tevoorschijn komen. Deze zijn in de Mallatdecompositie een stuk minder aanwezig. De reden 
hiervoor is simpel aan te wijzen. Bij de Mallatdecompositie bekijken we wavelets met een 
drager die in beide richtingen even groot is (een vierkant dus). Het Tensorproduct legt deze eis 
niet op met als gevolg dat er een wavelet is met een drager die heel breed is en ook weer 
heel kort: een balk. Als deze balk door de compressie-algoritme weggelaten wordt, mist de 
reconstructie hier een benadering en kan er een gekke horizontale/verticale balk in de reconstructie onstaan.

Het tweede wat we opmerkten is dat de PSNR structureel lager is bij het Tensorproduct in vergelijking 
met de Mallatdecompositie. Dit komt ook wel overeen met wat we zien in de figuren \ref{fig:tensor_start} 
en \ref{fig:tensor_end}. Merk wel op dat vooral figuur \ref{fig:tensor_start} een extreem voorbeeld 
is en dat de verschillen in figuur \ref{fig:tensor_end} al een stuk minder overheersend zijn.

Het is interessant om op te merken dat onze metingen -- dat het Tensorproduct consistent lagere 
PSNR-waardes oplevert -- in strijd zijn met de claims die gemaakt worden in \cite{tensor_vs_mallat}, 
hoewel er teveel onbekende factoren zijn om te weten of dit misschien een andere reden heeft.

\section{Gemengde decompositie op 3D signalen}
Het bekijken van 3D signalen was een interessante toevoeging aan ons project maar de analyse  
van filmmateriaal is moeilijker gebleken. Dit had twee redenen.

Ten eerste is onze implementatie, hoewel effici\"ent, niet erg snel. Het comprimeren van een 
afbeelding duurt ongeveer dertig seconden en het comprimeren van filmmateriaal duurt al snel het 
vijftigvoudige. Hierdoor wordt het comprimeren van voldoende 3D signalen een langdurige taak.

Daarnaast hebben we veel moeite gehad om passend beeldmateriaal te vinden. Uit het eerste punt volgt 
dat dit beeldmateriaal klein in afmeting moet zijn. Als de afbeelding klein is, is het echter moeilijk
om een bepaalde compressie-ratio te halen met een goede reconstructie. Dit maakte de zoektocht lastig.

\subsection{PSNR}
Hoewel de PSNR in het geval van de afbeeldingen een prima representatie geeft van de kwaliteit van de 
reconstructie, blijkt het bij het beeldmateriaal dat wij bekeken hebben minder goed te werken. 
Zie bijvoorbeeld de grafiek links in \ref{fig:cockto}, hier is te zien dat de PSNR slechts weinig verschilt
tussen de verschillende compressieniveau's zodat we geen nuttige conclusies kunnen trekken.
Daarentegen zien we in grafiek rechts in \ref{fig:cockto} een patroon dat we verwachten, daling van de PSNR
bij kleinere compressiepercentages.

Aangezien we niet erg veel beeldmateriaal bekeken hebben zullen we de PSNR als maat loslaten en ons richten 
op de optische verschillen die we kunnen aanwijzen tussen de twee reconstructies.

\begin{figure}[h]
\centering
\begin{subfigure}[t]{0.48\textwidth}
\includegraphics[width=\linewidth]{plaatjes/cockto.png}
\end{subfigure}
\begin{subfigure}[t]{0.48\textwidth}
\includegraphics[width=\linewidth]{plaatjes/croppedgif.png}
\end{subfigure}
\caption{Links: de grafiek horende bij de dansende octopus. Rechts: de grafiek horende bij de scene met de typmachine en de boom.}
\label{fig:cockto}
\end{figure}

\subsection{Mallatdecompositie versus de mengvorm}
Wat we wel direct opmerkten is het enorme verschil in detail tussen de twee compressiemethoden. 
Dit komt omdat het Tensorproduct sneller convergeert bij continue signalen 
(zie stellingen \ref{thm:foutmallat} en \ref{thm:fouttensor}). 
Omdat een bewegend beeld over het algemeen continu is in de tijdsdimensie, is het gevolg niet vreemd.

Wanneer er w\'el een grote sprong in de tijd optreedt, zoals bij een scenewisseling, is de fout van 
het Tensorproduct ineens een stuk groter dan die van de Mallatdecompositie. 
Dit is vooral duidelijk bij de typmachinescene, die voor de Mallatdecompositie en de mengvorm
weergegeven is in figuren \ref{fig:frames_tensor} en \ref{fig:frames_nontensor}.
Het gele mannetje vervaagt in de Mallatdecompositie gedurende ongeveer 5 tijdseenheden 
terwijl de randen in de mengvorm voor 7 frames zichtbaar blijven.

Dit resultaat is ten eerste te wijten aan de grote(re) drager van de Daubechieswavelet, 
waardoor harde randen slechter gereconstrueerd worden.
Het effect wordt echter versterkt in de mengvorm doordat hier veel wavelets een langwerpige
drager in de tijdsrichting hebben, wanneer de scherpe randen in frame $71$ gereconstrueerd moeten worden
zijn deze ook goed te zien in de volgende frames.

\begin{figure}[h]
\centering
\begin{subfigure}{\linewidth}
\includegraphics[width=\linewidth]{plaatjes/frames_notensor_small.png}
\caption{Mallatdecompositie}
\label{fig:frames_tensor}
\end{subfigure}
\centering
\begin{subfigure}{\linewidth}
\includegraphics[width=\linewidth]{plaatjes/frames_tensor_small.png}
\caption{Mengvorm}
\label{fig:frames_nontensor}
\end{subfigure}
\caption{\texttt{croppedgif.gif} (met verhoogd contrast) ingezoomd op de verstoring, frames 71 t/m 80 met 2\% compressie door de Daubechies 2-wavelet
met de verschillende decomposities.}
\end{figure}


\chapter{Populaire samenvatting}

\vspace{-3pt}
Vandaag de dag is men steeds meer bezig met het delen van informatie. Omdat het belangrijk is dat deze informatie ergens opgeslagen wordt, zijn we altijd op zoek naar \emph{compressie}: het `inpakken' van informatie zodat het minder ruimte in beslag neemt.

Bij deze compressie zijn twee types te onderscheiden. Bijvoorbeeld tekst moet na de compressie perfect ge\emph{reconstrueerd} kunnen worden. Dit soort signalen zullen wij niet behandelen. Wij zijn ge\"interesseerd in \emph{lossy} compressie, wat betekent dat we `oninteressante informatie' gewoon weg kunnen gooien met de hoop dat de reconstructie goed genoeg lijkt op het origineel.

In het bijzonder zullen wij ons vizier richten op beeldmateriaal, eerst stil en daarna bewegend. Foto's worden veelal opgeslagen in het zogenaamde JPEG-beeldformaat: \texttt{.jpg}. De wiskunde achter dit compressie-algoritme is niet triviaal en wij zullen hier dan ook grondig op ingaan in ons verslag. De Fouriertransformatie wordt in \emph{signaalverwerking} al honderden jaren toegepast om signalen (functies, geluid, foto's, films) te schrijven in een andere \emph{basis}, namelijk de basis van de complexe $e$-machten $e^{2 \pi i t}$. We hopen hierbij maar dat ons signaal goed te benaderen valt in deze basis: een klein aantal grote co\"effici\"enten en een groot aantal kleine.

Het lot wil dat dit helemaal niet altijd goed werkt, de Fouriertransformatie werkt namelijk niet goed wanneer een afbeelding harde randen heeft.
Daarm is er in de laatste dertig jaar een nieuwe speler op het gebied van signaalverwerking opgedoken, de \mbox{\emph{Wavelettransformatie}}. 
We spreken dan over een nieuwe versie van JPEG: JPEG-2000.

De wavelettransformatie maakt gebruik van de wavelet, een r\"eele functie waarvoor geldt dat $\int_{-\infty}^\infty \psi(t) dt = 0$;
deze heeft dus net als een golf even hoge pieken als dalen, vanwaar de naam.
Dit op zich is nog niet handig, maar er zijn wavelets met een zogenaamde \emph{compacte drager}:
de functie is identiek $0$ buiten een \emph{begrensd} en \emph{gesloten} gebied. 
Hierin ligt dan het verschil met de complexe $e$-machten van de Fouriertransformatie, waarvan de drager gelijk is aan $\R$. 
Het effect van deze compacte drager is nu dat niet elke wavelet last heeft van een harde rand in het plaatje, maar slechts een deel van de functies.
Zie figuur \ref{fig:samenv} voor twee van de wavelets die we gebruikt hebben.

\begin{figure}[h]
  \centering
  \begin{subfigure}{0.32\linewidth}
    \includegraphics[width=\linewidth]{plaatjes/db1.pdf}
  \end{subfigure}
  \begin{subfigure}{0.32\linewidth}
    \includegraphics[width=\linewidth]{plaatjes/db2_psi.pdf}
  \end{subfigure}
  \caption{Links: De Haarwavelet. Rechts: De Daubechies-2 wavelet.}
\label{fig:samenv}
\end{figure}

Een orthogonale waveletbasis is nu een verzameling $\{ \psi_{j,n}: j \in \N_0,\, n \in \Z \}$ waarbij een basisfunctie \mbox{$\psi_{j,n}(t) = 2^{j/2} \psi(2^jt - n)$} de waveletfunctie is met een verschuiving over $n$ en een dilatie met factor $2^j$. Wat dit in feite betekent is dat we gaan inzoomen met onze wavelets: het signaal wordt op verschillende (zoom)\emph{niveau's} bekeken. De orthogonaliteitseigenschap houdt tenslotte in dat voor alle basisfuncties geldt dat $\langle \psi_{a,b}, \psi_{c,d} \rangle = \delta_{a,c} \cdot \delta_{b,d}$.

Zie wederom figuur \ref{fig:samenv} voor een grafiek van $\psi_{0,0} = \psi$ en $\psi_{0,-1}$ voor twee wavelets. De Haarwavelet is de oudste (en meest simpele) wavelet in gebruik en staat centraal in ons verslag. 
Rechts is de Daubechies-2 wavelet te zien, die andere interessante eigenschappen heeft.

De Fouriertransformatie en Wavelettransformatie geven ons op deze manier een wiskundig onderbouwde manier om signalen in \'e\'en dimensie (bijvoorbeeld geluid) te comprimeren naar een fractie van haar oorsponkelijke grootte. We hopen dan dat de reconstructie nog dichtbij het origineel ligt (en het geluid goed te verstaan is). In twee dimensies gebruiken we voor de Fouriertransformatie het zogenaamde \emph{Tensorproduct}. Dit is een natuurlijke manier om meer ruimtes (in ons geval van functies) loodrecht op elkaar te zetten tot een $n$-dimensionale ruimte. In twee dimensies krijgen de basiselementen nu een tweedimensionale vorm, we bekijken in de ene richting een wavelet en in de andere richting een andere; de drager wordt hiermee een rechthoek.

Bij de Fouriertransformatie blijkt zo'n voortzetting naar meer dimensies met het Tensorproduct gewoon goed te gaan. Bij de Wavelettransformatie zijn er twee mogelijkheden die beide gebruikt worden in de praktijk. In twee dimensies kunnen we \'of in beide richtingen even snel inzoomen (dit is de zogenaamde Mallatdecompositie), \'of niet (ons `Tensorproduct').
De elementen van de Mallatdecompositie hebben over het algemeen een vierkante drager terwijl deze in het Tensorproduct een (langwerpige) rechthoek kunnen worden. Dit verschil heeft zichtbare gevolgen.

In het praktische deel van ons verslag hebben we beide decomposities bekeken met interessante resultaten. Het blijkt namelijk dat voor signalen met `redelijk weinig' harde randen het Tensorproduct een goede reconstructie geeft. Dit feit hebben we ge\"exploiteerd door een mengvorm van de Mallatdecompositie en het Tensorproduct te gebruiken in de analyse van 3D-signalen. Ook dit gaf interessante en mooie resultaten.

\begin{figure}[h!]
  \centering
  \begin{subfigure}[b]{0.20\textwidth}
    \centering
    \includegraphics[width=\textwidth]{plaatjes/vw.jpg}
  \end{subfigure}
  \begin{subfigure}[b]{0.20\textwidth}
    \centering
    \includegraphics[width=\textwidth]{plaatjes/vw_fourier_0_01.jpg}
  \end{subfigure}
  \begin{subfigure}[b]{0.20\textwidth}
    \centering
    \includegraphics[width=\textwidth]{plaatjes/vw_haar_0_01.jpg}
  \end{subfigure}
  \begin{subfigure}[b]{0.20\textwidth}
    \centering
    \includegraphics[width=\textwidth]{plaatjes/vw_db2_0_01.jpg}
  \end{subfigure}
  \caption{Logo van VW op drie manieren gecomprimeerd met 1\% van de originele data.}
\end{figure}


\begin{thebibliography}{11}
\addcontentsline{toc}{chapter}{Bibliografie} 

\bibitem{akra-bazzi}
  Mohamad Akra, Louay Bazzi,
  \emph{On the solution of linear recurrence equations}.
  Computational Optimization and Applications,
  10(2):195 - 210,
  1998.

\bibitem{fourier-fout}
  Bochner S., Chandrasekharan K.
  \emph{Fourier Transforms}.
  Princeton University Press.
  1949.

\bibitem{wavelet_filter}
  \url{http://djj.ee.ntu.edu.tw/Wavelet_Filter.pdf}

\bibitem{tammo}
  Tammo Jan Dijkstra,
  \emph{Adaptive tensor product wavelet methods for solving PDEs}, 2009
\bibitem{tensor_wavelet}
  \url{http://www.uio.no/studier/emner/matnat/math/MAT-INF2360/v12/tensorwavelet.pdf}
\bibitem{mallat}
  St\'ephane Mallat,
  \emph{A Wavelet Tour of Signal Processing}
\bibitem{jackson}
  \url{http://www.ams.org/journals/bull/1960-66-02/S0002-9904-1960-10426-0/S0002-9904-1960-10426-0.pdf}

\bibitem{daubechies}
  Ingrid Daubechies, \emph{Orthonormal Bases of Compactly Supported Wavelets}.
  AT\&T Bell Laboratories, 1988.
\bibitem{parseval}
  \url{http://www.encyclopediaofmath.org/index.php/Parseval_equality}

\bibitem{weidmann}
Joachim Weidmann, \emph{Linear operators in Hilbert spaces}, 1980.

\end{thebibliography}

\end{document}
