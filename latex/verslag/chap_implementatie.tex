\chapter{Onze implementie}
\label{implementatieH}
Een discreet signaal in 1 dimensie kan over het algemeen geschreven worden als een vector re\"ele getallen van een bepaalde lengte (de signaallengte). Wanneer we afbeeldingen willen beschrijven, bekijken we daarom vaak matrices van grijswaardes, waarbij de vier kanalen (rood, groen, blauw, doorzichtigheid) elk een eigen matrix hebben die bij elkaar de representatie geven van een kleurenafbeelding.

Omdat het menselijk oog niet heel nauwkeurig is -- en door de limitaties van computers -- slaat men de matrixco\"effici\"enten niet op als re\"ele getallen. In plaats daarvan worden vaak integerwaardes tussen 0 en 255 gebruikt. Omdat elk van de 3 kleurenkanalen zo 256 waardes krijgt, is het kleurenpallet toch groot. In elk van onze implementaties worden beelden gescheiden in haar vier kanalen en wordt op elk kanaal afzonderlijk het algoritme toegpast. De reden voor deze keuze is dat de verschillende kanalen niet veel met elkaar van doen hoeven hebben.

Bewegend beeld is zo op precies dezelfde manier te zien als een driedimensionale matrix waarbij we meerdere afbeeldingen achter elkaar plakken.

\section{Implementatie van Fouriertransformatie in Python}
Met de theoretische basis uit voorgaande secties is de implementatie van het 
\emph{Fast Fourier Transform}-algoritme nu aan bod gekomen. 
In deze sectie zullen we uitweiden over enkele genomen hordes en gemaakte keuzes. De resultaten volgen
in een latere sectie.

\subsection{Fast Fourier Transform Implementatie}
De Pythoncode uit \autoref{fft} die gebruikt is om het FFT-algoritme in te programmeren volgt precies het schema van de eerder gegeven pseudocode.

Hierbij dient uitgelegd te worden dat de notatie \texttt{a + b} voor twee lijsten de \emph{concatenatie} van de lijsten geeft.
Dit algoritme werkt -- zoals eerder beschreven -- enkel op signalen waarvan de lengte
$N$ een macht van $2$ is. We hebben daarom signaalextensie toe moeten passen in \autoref{zeropad} door middel van zero-padding
om het programma ook op andersvormige signalen te laten werken.

Vervolgens hebben we deze code toegepast in twee dimensies. We maken hier gebruik van de definitie van de
MFFT-algoritme, dat grofweg zegt dat een meerdimensionaal algoritme kan worden geconstrueerd door herhaald 
het 1-dimensionale geval toe te passen. Voor 2 dimensies in het bijzonder betekent dit dat we simpelweg 
FFT konden toepassen op rijen en kolommen: zie \autoref{fft2d}.

Hier is gebruik gemaakt van function-mapping uit \S \ref{python}.
Met het oog op duidelijkheid is de Zero-Padding-fase weggelaten en
deze algoritme verwacht nu nog een $2^n \times 2^m$ matrix maar dit is gemakkelijk te implementeren door Zero-Padding toe te passen op rijen en kolommen.

\subsection{Compressie}
Zoals in de introductie van ons verslag al aan bod gekomen is, hebben we compressie bereikt door de kleinste 
co\"effici\"enten weg te gooien en aan de slag te gaan met een kleinere lijst. 
Dit brengt twee problemen met zich mee: als eerste, welke waarde moet er als \emph{cutoff} genomen worden en ten tweede, hoe slaan we de overgebleven co\"effici\"enten op?

\subsubsection{Cutoff}
Hoewel ons eerste programma een voorgegeven cutoff verwachtte, wilden we liever op compressieratio selecteren.
Daarom hebben latere versies de routine uit \autoref{cutoff} gebruikt. 

We sorteren de co\"effici\"enten in \texttt{mat} op grootte en kiezen de cutoff als de co\"effici\"ent
die de lijst opdeelt ten opzichte van de \texttt{ratio}.

\subsubsection{Opslaan van de co\"effici\"enten}
Voor het opslaan van de co\"effici\"enten hebben we een adaptieve basis gebruikt
we moesten daarom bijgehouden \emph{welke} co\"effici\"enten er opgeslagen worden. 
Hier hebben wij een feature van Python gebruikt, de \emph{dictionary}.
Een dictionary in Python is niks meer dan een lijst van (naam, waarde)-paren. 
De naam is in ons geval de plek in de matrix en de waarde is het (complexe) getal wat in die cel hoort te staan. 
Dit geeft aanleiding tot het schrijven van twee routines \texttt{dict2mat} (\autoref{mat2dict}) en \texttt{mat2dict} (\autoref{dict2mat}).

Duidelijk moge zijn dat we, door deze twee functies na elkaar uit te voeren, een matrix terugkrijgen met nullen waar eerst co\"effici\"enten kleiner dan \texttt{cutoff} stonden.

\subsection{Het .wvg-beeldformaat}
Hoewel we een eigen beeldformaat gemaakt hebben, is het ons niet gelukt om de afbeeldingen echt in minder ruimte op te slaan dan hun ongecomprimeerde versie. Dit heeft met twee dingen te maken. Ten eerste is de manier waarop wij de dictionaries opgeslagen hebben niet optimaal geweest.\footnote{Later bedachten we nog een manier die ongeveer 50\% van de oorspronkelijke ruimte in zou nemen, door niet (index, waarde)-paren op te slaan maar met een zogenaamde bitmap.} Ten tweede is de Fouriertransformatie een complex algoritme in de zin dat zij complexe co\"effici\"enten teruggeeft. Door het isomorfisme $\C \simeq \R^2$ moesten wij zodoende twee waardes voor elke co\"effici\"ent opslaan.

\section{Wavelets in 1 dimensie}
Een orthogonale wavelet wordt discreet helemaal gekenmerkt door haar filter $h$. Zoals beschreven in de sectie over de FWT, hebben we daarmee (via vergelijkingen~\ref{highpassfilter}, \ref{approx_rec}, \ref{detail_rec}, \ref{reconstr_FWT}) feitelijk het algoritme al helemaal in handen. Definieer
\[
	\texttt{dec\_l} := \bar{h} \quad \texttt{dec\_h} := \bar{g} \quad 
        \texttt{rec\_l} := h \quad \texttt{rec\_h} := g.
\]
Dan kunnen we met twee routines (\autoref{next} en \autoref{prev}) in elke richting het algoritme simpel implementeren.

\section{Wavelets in twee dimensies}
Zoals eerder beschreven is, kunnen we in meer dimensies twee kanten op: de Mallatdecompositie en het Tensorproduct.
We zullen aan beiden wat aandacht besteden.

\subsection{Mallatdecompositie}
De Mallatdecompositie is qua implementatie vrij simpel. 
We gebruiken nog steeds de eendimensionale \texttt{next} en \texttt{prev} routines 
door deze bij elke stap op rijen en kolommen uit te voeren, wat equivalent is met een tweedimensionale convolutie.
Omdat we hier in twee richtingen werken, moet het signaal bovendien aangevuld worden tot een vierkante tweemacht.\footnote{Het is mogelijk om een rechthoek te transformeren maar daar hebben wij hier geen aandacht aan besteed.} Zie \autoref{fwt2d} en \autoref{ifwt2d}.


\subsection{Tensorproduct}
Het Tensorproduct in meer dimensies uit \autoref{tensorimpl} gaat in principe net zoals de Fouriertransformatie in meer dimensies door op rijen en kolommen de algoritme in 1 dimensie toe te passen.

\section{Hogere dimensies}
In hogere dimensies wordt het nog interessanter. In bijvoorbeeld drie dimensies kunnen we -- naast de Mallatdecompositie en het Tensorproduct -- ook de eerder genoemde mengvorm toepassen. Wij hebben er voor gekozen om in dit geval de Mallatdecompositie in het $x,y$-vlak te gebruiken, en het Tensorproduct in de $z$-richting hier loodrecht op.

\subsection{Mallatdecompositie}
De Mallatdecompositie in drie dimensies is gemakkelijk voort te zetten uit de tweedimensionale variant en
er moet nu tijdens elke stap naast een \texttt{next} of \texttt{prev}-stap in de $y$-en $z$-richting ook nog een stap 
in de $z$-richting gedaan worden. Zie \autoref{mallat3d} 

\subsection{Mengvorm}
Bij de mengvorm gaan we iets interessanters doen: zoals bij het tweedimensionale Tensorproduct
voeren we weer function-mapping uit maar dit maal voeren we de tweedimensionale Mallatdecompositie
uit op elk $(x,y)$-vlak en voeren we vervolgens de \emph{FWT} uit op elke rij in de $z$-richting. Het resultaat is te vinden in \autoref{mengertje}
