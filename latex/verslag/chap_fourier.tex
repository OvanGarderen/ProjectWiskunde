\chapter{Fourier}
\label{fourierH}

Bij het analyseren van periodieke functies is de Fouriertransformatie het gereedschap bij uitstek.
Het is een manier om een signaal te karakteriseren aan de hand van een bereik aan frequenties en dit maakt het een overtuigende manier om `nette' periodieke functies te beschrijven.
We kunnen de eis van periodiciteit ook loslaten als we naar functies op een interval kijken door dit op te vatten
als \'e\'en fase van een periodieke functie.

Iets concreter kijken we dus naar functies $f \in L_2([a,b])$. De basisfuncties waarmee we vervolgens de analyse uitvoeren worden gegeven door de complexe $e$-machten.
\begin{definitie}[Fourierbasis] Bekijk de functieruimte $L_2([a,b])$. We defini\"eren de verzameling $F_{a,b}$ 
door:
\[
  F_{a,b} := \left\{ \phi_k(x) = \tfrac{1}{\sqrt{b-a}} e^{\dpii \cdot k \frac{x-a}{b-a}} \largediv k \in \Z \right\}
\]
We noemen $F_{a,b}$ de Fourierbasis van $L_2([a,b])$.
\end{definitie}
Het woord basis is hier met recht gebruikt. Het is immers bekend dat de complexe $e$-machten loodrecht staan onder
het inproduct op $L_2$. 
Om nu een willekeurige functie te schrijven in deze basis, introduceren
we de Fouriergetransformeerde.
\begin{definitie}[Fouriertransformatie]
Zij gegeven een functie $f\in L_2([a,b])$. Schrijf dan $\hat f : \Z\to\C$ met entries gedefinieerd volgens:
\[
  \hat f [k] = \frac{1}{\sqrt{b-a}} \cdot \inpr{f}{\phi_k} = \frac{1}{b-a} \int_a^b f(x) \cdot e^{-2 \pi i \cdot k \frac{x-a}{b-a}}\d{x}.
\]
We noemen $\hat f$ de Fouriergetransformeerde van $f$.
\end{definitie}
\begin{definitie}[Inverse Fouriertransformatie]
  Gegeven een Fouriergetransformeerde $\hat f$ van een functie $f \in L_2([a,b])$, is de reconstructie $f^\circ$ van $f$ gegeven door:
  \[
    f^\circ (x) = \sum_{k=-\infty}^\infty \hat f [k] \phi_k(x).
  \]
\end{definitie}

In \cite{fourier-rec} is aangetoond dat $f^\circ$ een reconstructie geeft van $f$ die voldoet aan: 
\begin{eqnarray}
  \Ldnorm{f-f^\circ}{a,b}=0 \quad\quad \text{wanneer }& f\in L_2([a,b]) \\
  f(x) = f^\circ(x) \quad\quad \text{wanneer }& f \in C^1.
\end{eqnarray}
Merk op dat $\Ldnorm{f-f^\circ}{a,b}=0$ niet betekent dat $f - f^\circ = 0$.

%-----------------------------------------------------------------------------------------------------------------
\section{De \emph{Discrete Fourier Transform}}
Zoals we gezien hebben in het voorafgaande, kan de Fouriertransformatie gebruikt worden om continue signalen te karakteriseren door verschillende frequenties. 
Een groot gebied binnen de signaalanalyse is echter van discrete aard aangezien hier veelal digitale instrumenten worden gebruikt.

Om discrete signalen te analyseren lijkt het voor de hand te liggen om een deze als stapfuncties te zien;
een stapfunctie zit immers in $L_2$. De discontinu\"iteiten van de stapfunctie leiden echter tot ongewenste resultaten.
Elke reconstructie in termen van eindig veel continue basisfuncties is namelijk weer continu,
zodat voor een perfecte reconstructie van een stapfunctie met deze methode altijd een oneindige rij co\"effici\"enten nodig is. 
Bovendien is het moeilijk om uit deze co\"effici\"enten relevante informatie te destilleren over de aard van het signaal.

In plaats van de discrete signalen in te bedden in $L_2$ zullen we ze beschouwen als rijen in $\ell_2$ (zie sectie \ref{ruimtes}).
Hiervoor zullen we de Fourier-basis discretiseren en ons richten op de ruimte $\R^n$.
We veranderen daarvoor de co\"ordinaten naar een discrete $j$ volgens
\[
\frac{x-a}{b-a} \leftrightarrow \frac j n 
\]
Op deze manier zal de discretisatie in de limiet naar het continue geval overgaan. 

\begin{definitie}[Discrete Fourierbasis] Gegeven zij de ruimte $\C^n$. Defini\"er dan de verzameling
\[
  S_n := \left\{ s_k [j] = e^{\dpii\cdot k j/n } \largediv k,j \in \{1, \ldots, n\} \right\},
\]
als de \emph{discrete Fourierbasis} op deze ruimte met basisvectoren $s_k$.
\end{definitie}
We weten dat de basisvectoren loodrecht staan vanwege de eigenschap:
\[
  \inpr{s_k}{s_j} = 
  \sum_{m=1}^n s_k[m]\cdot \overline{s_j[m]} = 
  \sum_{m=1}^n e^{\dpii\cdot m (k-j)/n} =
  \begin{cases}
    0 \quad \text{als } k\neq j\\
    n \quad \text{als } k = j.
  \end{cases}
\]
Hierdoor kunnen we een discreet signaal $x$ schrijven in termen van $S_n$ 
een vector van inproducten te defini\"eren.
We noemen $X$ de discrete Fouriertransformatie (DFT) met entries:
\[
  X[k] = \tfrac{1}{n} \inpr{x}{s_k} \quad k\in\{1, \ldots ,n\}.
\]
Vervolgens hebben we een inverse voor deze operatie die $X$ afbeeldt op de reconstructie $x^\circ$ volgens:
\[
  x^\circ[j] = \inpr{X}{s_j^{-1}} \quad j\in \{1, \ldots, n\}.
\]
Waarbij we de notatie $s_j^{-1} = (s_j[1]^{-1},\ldots, s_j[n]^{-1})$ gebruiken.
Vanwege de orthogonaliteit van de basis $S_n$ is dit een perfecte reconstructie ($x[k] = x^\circ[k]$).

Om de claim te ondersteunen dat de DFT-methode een echte discretisatie is van de continue Fouriertransformatie,
willen we bewijzen dat deze algoritme voor een steeds fijnere selectie van waardes van een functie in de 
limiet hetzelfde resultaat geeft als de continue Fouriertransformatie. 
Zoals gebruikelijk bij het overschakelen van een discrete naar een continue setting kunnen we dit doen 
door de definitie van de Riemannintegraal toe te passen op de sommatie die voor handen ligt.

\begin{stelling}[Limiet van discrete Fouriertransformatie]
  Gegeven zij een functie $f\in L_2([a,b])$. Bekijk een discretisatie van $f$ in $n$ gelijke intervallen zodat de discrete $f$ precies 
  de randwaarde van elk interval inneemt.
  Dan geldt dat de discrete Fouriergetransformeerde limiteert naar de algemene Fouriergetransformeerde wanneer $n\to\infty$.
\end{stelling} 
\begin{proof}[Bewijs]
Gegeven een interval $[a,b]$ kunnen we een partitie $P$ maken in $n$ gelijke delen: laat $P=\{a=t_0,t_1,..,t_n=b\}$ met $t_j = a+\tfrac{j(b-a)}{n}$.
We discretiseren onze functie $f$ door uit elk interval $[t_{j-1},t_{j}]$ van de partitie de randwaarde in 
$x_j = t_j$ te selecteren, dus
\[
f[j] = f(x_j) = f(\frac{b-a}{n}j + a).
\]
De DFT van de discrete functie $f[\cdot]$ wordt dan gegeven door:
\[
F[k] = \frac1n\sum_{j=1}^n f[j] \cdot s_k^{-1}[j].
\]
We schrijven dit om in termen van onze non-discrete functie door $x_j\in[a,b]$ 
te vervangen door zijn discrete tegenhanger en krijgen zo:
\begin{eqnarray*}
  F[k] =& \frac{1}{n} \sum_{j=1}^n f[j]\cdot s_k^{-1}[n\cdot\tfrac{x_j-a}{b-a}] \\
       =& \frac{1}{n} \sum_{j=1}^n f(x_j)\cdot e^{-\dpii \cdot k \tfrac{x_j-a}{b-a}}\\
       =&  \frac{1}{\sqrt{b-a}}\sum_{j=1}^n f(x_j)\cdot \phi^*_k(x_j) \cdot\frac{b-a}{n} 
\end{eqnarray*}
We merken op dat de term $\frac{b-a}{n}$ precies de grootte is van de intervallen van de partitie 
en dat we het geheel omgeschreven hebben in termen van onze continue functies $f$ en $\phi_k$.
Omdat het product $f\cdot\phi_k$ integreerbaar is moet voor elke partitie $P$ met 
waarden in punten $x_j$ uit elk interval deze sommatie convergeren naar de integraal
\[
  \frac{1}{\sqrt{b-a}} \int_{[a,b]} f(x) \phi^*_k(x) \d{x}
\]
wanneer we de maaswijdte ($\tfrac{b-a}{n}$) naar $0$ laten gaan. \cite{shilov}
Dit is duidelijk het geval wanneer we de limiet $n\to\infty$ nemen. 
Dus is de DFT een goede discretisatie van de Fouriergetransformeerde. 
\end{proof}

%-----------------------------------------------------------------------------------------------------------------
\section{De Fast Fourier Transform}
\label{fft_sec}
De snelheid van de DFT-algoritme valt in de praktijk nogal tegen. Het nemen van $n$ inproducten over vectoren 
van lengte $n$ heeft namelijk een tijdscomplexiteit van $\O(n^2)$. Dit staat de directe implementatie van de DFT 
voor praktische toepassingen in de weg. Daarom is er een alternatieve algoritme, de \emph{Fast Fourier Transform}.

%%%%%
\begin{algo}[Fast Fourier Transform]
Gegeven een inputsignaal $x$ van lengte $n=2^m$, geeft het algoritme $\FFT$ 
een lijst terug van waardes $X$ van lengte $n=2^m$ als volgt:

Als $m=0$ dan geeft de $\FFT$ de lijst (van \'e\'en element) direct terug:
\[
X = x.
\]
Wanneer $m\neq0$ splitsen we de lijst $x$ op in lijsten $\e,o$ van zijn even en oneven indices:
\eq{
  \e[k]   =& x[2k]   &\quad \text{voor } k < n/2,\\
   o[k]   =& x[2k+1] &\quad \text{voor } k < n/2.
}
Vervolgens voeren we hierop het $\FFT$ algoritme uit om de volgende lijsten te verkrijgen:
\eq{
  E =& \FFT(\e), \\
  O =& \FFT(o).
}
Hiermee wordt de output van de algoritme geconstrueerd volgens:
\[
  X[k] = \left\{\begin{array}{llll}
    E[k]         &+& e^{-\dpii k/n}\cdot O[k] &  k< n/2, \\
    E[k-n/2] &-& e^{-\dpii (k-n/2)/n}\cdot O[k-n/2] &  k\geq n/2.
  \end{array}\right.
\]
\end{algo}
%%%%%

Dit is dus een recursief gedefinieerde algoritme dat een signaal meermaals halveert.
Het is gegarandeerd dat deze algoritme afloopt vanwege de conditie op $m=0$ samen met de halvering van de input bij elke stap. 
Een belangrijke voorwaarde voor de relevantie van de FFT is nu dat de algoritme hetzelfde resultaat geeft als de DFT-algoritme en dit zullen.
\begin{stelling}
  Het uitvoeren van de Fast Fourier Transform algoritme op een datasetgeeft
  dezelfde getransformeerde als de discrete Fouriertranformatie.
\end{stelling}
\begin{proof}[Bewijs]
We bewijzen met inductie naar $n$. Onze aanname is dat de FFT-algoritme voor $x$ van lengte $n=2^m$ gelijk is aan de DFT van $x$, ofwel
\eq{
  X[k] = \sum^{n}_{k=1} x[j] \cdot e^{-2\pi i \cdot jk/n}.
}
Dit geldt duidelijkerwijs wanneer $m=0$, onze basisstap. Hiervoor geldt namelijk:
\eq{
  X[k] = x[k] = x[1] = \sum^{2^0}_{k=1} x[j] \cdot e^{-2\pi i \cdot 1/2^0}.
}
Vervolgens passen we inductie toe naar $m$ door onze aanname voor $m-1$ te gebruiken. We vullen hier $E[k]$ en $O[k]$ in de vergelijking voor $X[k]$ in, deze hebben immers lengte $n=2^{m-1}$.
\eq{
  X[k] = \left\{\begin{array}{llll}
    \sum^{n/2}_{j=1} \e[j] 
    \cdot e^{-2\pi i \cdot kj \cdot 2/n} &+& 
    e^{-2\pi i \cdot k/n}
    \sum^{n/2}_{j=1} o[j] 
    \cdot e^{-2\pi i\cdot kj \cdot 2/n} &  k< n/2 \\
    \sum^{n/2}_{j=1} \e[j] 
    \cdot e^{-2\pi i\cdot (k-n/2) j\cdot 2/n} &-& 
    e^{-2\pi i\cdot (k-n/2)/n}
    \sum^{n/2}_{j=1} o[j] \cdot e^{-2\pi i\cdot (k-n/2)j\cdot 2/n} &  k\geq n/2 
  \end{array}\right.
}
Merk op dat we de $e$-machten in het tweede geval kunnen vereenvoudigen volgens
\eq{
  e^{-2\pi i\cdot (k-n/2)j \cdot 2/n} 
  = e^{-2\pi i\cdot kj\cdot 2/n} \quad,\quad e^{-2\pi i\cdot(k-n/2)/n} 
= -e^{-2\pi i\cdot k/n},
}
waardoor het gevalsonderscheid wegvalt, aangezien beide vergelijkingen nu identiek zijn.
Het resultaat is $X[k]$ als sommatie over de lijsten $\e$ en $o$. Vul de relatie voor $\e$ en $o$ met $x$ in en neem de factor voor de oneven indices mee in de sommatie om te krijgen
\eq{
  X[k] = \sum^{n/2}_{j=1} x[2j] \cdot e^{-2\pi i\cdot k (2j)/n} + 
    \sum^{n/2}_{j=1} x[2j+1] \cdot e^{-2\pi i\cdot k (2j+1)/n} 
    = \sum^n_{j=1} x[j] \cdot e^{-2\pi i\cdot k j/n}.
}
Dit bewijst dat de FFT hetzelfde resultaat levert als de DFT-algoritme. Het bewijs voor de gelijkheid van de iDFT en de inverse FFT is 
hetzelfde wanneer men de substitutie $-2\pi i \rightarrow 2\pi i$ uitvoert.
\end{proof}

\begin{opmerk}
We hebben hier telkens aangenomen -- en zullen deze aanname ook doorzetten -- 
dat de lengte van het ingangssignaal een macht van $2$ is. Dit is een belangrijke
eigenschap waar de variant van de FFT-algoritme dat hier gebruikt wordt, door werkt. Deze versie van FFT wordt
 \emph{Radix-2 Decimation In Time} van Cooley-Tukey genoemd. Algemenere vormen van deze algoritme
worden ook toegepast in geoptimaliseerde algoritmes maar om de implementatie te versimpelen 
is voor Radix-2 gekozen. Eventuele verschillen in afmetingen tussen een signaal en 
een 2-macht zijn opgelost met signaalextensie, zoals eerder beschreven.
\end{opmerk}

%-----------------------------------------------------------------------------------------------------------------
\subsection{Complexiteit van de Fast Fourier Transform}
We zullen de claim bewijzen dat de complexiteit van de FFT werkelijk beter lager is dan die van de DFT.
Hiervoor hebben we de volgende stelling uit de Complexiteitstheorie nodig.
\begin{stelling}[ Akra-Bazzi \cite{akra-bazzi}]
\label{akra}
    Zij $T:\N\to\R$ een recurrente betrekking van de vorm
    \[
    T(n) = \begin{cases}
      c_0 &\text{ als } n \leq d, \\
      a T(n/b) + f(n) &\text{ anders,} \\
    \end{cases}
    \]
    waarbij $a,b,d\in\N$, $c_0\in\R$ en $f$ een functie $f:\N\rightarrow\R$ die voldoet aan 
    \[
    \exists k \in \N \,:\, f(n) \in \theta(n^{\log a/\log b} \log^k n),
    \]
    dan wordt de orde van $T(n)$ gegeven door:
    \[
      T(n) \in \theta(n^{\log a / \log b} \log^{k+1}n).
    \]
\end{stelling}
De recurrente betrekking $T(n)$ kan gezien worden als het aantal stappen dat een machine nodig heeft om de algoritme uit te voeren.
Deze stelling is nu voldoende om een uitspraak te kunnen doen over de complexiteit.
\begin{stelling}[Complexiteit van de FFT]
  De Fast Fourier Transform algoritme heeft een tijdscomplexiteit van $\theta(n\log n)$ voor input van lengte $n=2^m$.
\end{stelling} 
\begin{proof}[Bewijs]
We schrijven de FFT in pseudocode.

\begin{algorithmic}
\Function{FFT}{$x$}
\State $n \gets \text{lengte}(x)$ \Comment Assumptie: $n = 2^m$ voor een $m$
\If {$n == 1$}
	\State{$X \gets x$}
\Else
	\State $E \gets FFT(x[0::2])$ \Comment{FFT op even indices}
	\State $O \gets FFT(x[1::2])$ \Comment{FFT oneven indices}
	\For{$i = 0$ to $n-1$}
		\If{$i < n/2$}
			\State $X[i] \gets E[i] + e^{-2i \pi k/n} \cdot O[i]$
		\Else
			\State $X[i] \gets E[i] - e^{-2i \pi k/n} \cdot O[i]$
		\EndIf
	\EndFor
\EndIf
\State \Return{$X$}
\EndFunction
\end{algorithmic}

Deze algoritme is recursief zodat we de complexiteit kunnen schrijven door middel van een recurrente betrekking. Laat hiervoor $T(n)$ het aantal berekeningen zijn dat de algoritme kost bij een invoersignaal van lengte $n$. We maken een gevalsonderscheid: als de lijst lengte $1$ heeft geven we deze direct terug (1 berekening). Bij een lijst van lengte groter dan $1$ splitsen we de lijst op in de even en oneven entries en voeren we op beiden weer FFT uit. Vervolgens voeren we nog $n$ maal een vast aantal (namelijk $N$) berekeningen uit om tot het eindresultaat te komen. In formulevorm geeft dit de \emph{recurrente betrekking}
\[
T(n) = \begin{cases}
    1 &\text{ als } n = 1 \\
    2\cdot T(n/2) + N\cdot n &\text{ anders}. \\
\end{cases}
\]

Gebruik nu stelling \ref{akra}. De recurrente betrekking voor de complexiteit van de FFT is inderdaad van dezelfde vorm als die van $T(n)$ in bovenstaande stelling.
Laat hiervoor namelijk $a=b=2$, $c_0=d=1$ en $f (n) = N\cdot n$, waarvoor geldt dat
\eq{
  f(n) \in \theta(n^{\log 2/\log 2} \log^0 n)=\theta(n).
}
Dit betekent dat $T(n) \in \theta(n \log n)$.
Hiermee hebben we bewezen dat de FFT en daarmee de iFFT binnen tijdscomplexiteit $\O(n\log n)$ lopen. 
\end{proof}

%-----------------------------------------------------------------------------------------------------------------
\section{Discrete Fourier Transform in meer dimensies}
Een eigenschap van de DFT-algoritme is dat het op een natuurlijke manier uit te breiden is naar hogere dimensies.
Per constructie is er dus een manier om de FFT ook te beschrijven voor hogere dimensies.
Het idee hierbij is om het Tensorproduct te nemen van meerdere bases.

\begin{definitie}[Multidimensionale Discrete Fourierbasis] 
Gegeven zij een signaalruimte van de vorm
$\R^{n_1} \times \R^{n_2} \times \ldots \times \R^{n_m}$.
We defini\"eren de multidimensionale discrete Fourierbasis behorende bij deze signaalruimte als
\[
  S_{\bold n}= S_{n_1}\otimes S_{n_2} \otimes \ldots \otimes S_{n_m} = 
  \left\{s_{\bold k}[{\bold j}]  = s_{k_1}[j_1]\cdot s_{k_2}[j_2]\cdot\ldots\cdot s_{k_m}[j_m] 
  \largediv s_{k_i} \in S_{n_i}, j_i \in \{1,\ldots, n_i\} \right\}
\]
Met basisvectoren $s_{\bold k}$, waar $\bold k$ een indexvector uit $\{1, \ldots, n_1\}\times\ldots\times\{1\ldots, n_m\}$ is.
Het feit dat dit een basis is, kan gevonden worden in \cite{topo}.
\end{definitie}

Wanneer we nu een $m$-dimensionaal signaal bekijken van lengte $n_1 \times \cdots \times n_m$, dan kunnen we
hierop een multidimensionale Fouriertransformatie op defini\"eren door het inproduct te generaliseren naar 
onze $m$-dimensionale signaalruimte.
\begin{equation}
  \label{algemeen_fourier_schema}
  X[\bold k] = \tfrac{1}{\bold n}
  \sum_{\bold j = \bold 1}^{\bold n} x[\boldsymbol j] s^{-1}_{\bold k}[\boldsymbol j] 
  =
  \left(\prod_{i=1}^m \tfrac{1}{n_i}\right) \cdot 
  \sum_{j_1=1}^{n_1} \ldots \sum_{j_m=1}^{n_m} 
  x[j_1,\ldots,j_m] \cdot 
  e^{-\dpii\cdot k_m \cdot j_m /n_m } \cdot \ldots \cdot e^{-\dpii\cdot k_1 \cdot j_1 /n_1 }.
\end{equation}

Echter, oals we voor de DFT al gezien hadden, is het uitrekenen van al deze inproducten erg tijdrovend.
We zullen daarom de Multidimentionale Discrete Fouriertransformatie (MDFT) op een andere manier defini\"eren 
zodat we beter gebruik kunnen maken van de DFT- en FFT-algoritmen die we al gevonden hebben.

\begin{algo}[Multidimensionaal DFT-algoritme]
Gegeven is een $m$-dimensionale input $x$ en een huidig level $t$. We schrijven de algoritme met output $X_t$ volgens
\begin{equation}
  \label{mdft_cases}
  X_{t}[\boldsymbol k] = \begin{cases}
  \DFT(X_{t-1}\largediv_{x_1 = k_1,\ldots,x_{t-1} = k_{t-1},x_{t+1} = k_{t+1},\ldots,x_m =k_m})[k_t] & \text{als } t>0,\\
  x[k_t] & \text{als } t=0.
  \end{cases}
\end{equation}

We beweren dat $X_m$ de multidimensionale Fouriertransformatie van een $m$-dimensionaal signaal geeft zoals
in \eqref{algemeen_fourier_schema}.
\end{algo}
\begin{proof}[Bewijs]
We voeren inductie naar de dimensie $m$. De inductiehypothese wordt dat voor een dimensie $m$ 
de vergelijking voor $X_m$ uit de algoritme gegeven wordt door \eqref{algemeen_fourier_schema}.

Als $m=1$ dan geldt:
\[
X_1[k_1] = DFT(X_0)[k_1] = DFT(x)[k_1],
\]
wat inderdaad de DFT in 1 dimensie geeft.


We gaan vervolgens door met de inductiestap. Laat $m > 1$. Schrijf kort
\[
\tilde X_{t-1}[k_t] := 
X_{t-1}\largediv_{x_1 = k_1,\ldots,x_{t-1} = k_{t-1},x_{t+1} = k_{t+1},\ldots,x_m =k_m}[k_t].
\]
Dan
\[
  X_m [\boldsymbol k] = 
  \DFT(\tilde X_{m-1})[k_m]
  = \frac 1 {n_m}\sum_{j_m=1}^{n_m} \tilde X_{m-1}[j_m] s^{-1}_{k_m}[j_m].
\]
We vatten $\tilde X$ op als een vector van lengte $n_m$ van $m-1$ dimensionale Fouriergetransformeerden.
Dit mag omdat
\[
X_{m-1}\largediv_{x_1=k_1,\ldots,x_{m-1}=k_{m-1}}[k_m] = X_{m-1}\largediv_{x_m=k_m}[k_1,\ldots,k_{m-1}]
\]
en daarmee is $X_{m-1}\largediv_{x_m=k_m}$ een $m-1$-dimensionaal object dat weer wordt gegeven door de relatie
in \eqref{mdft_cases}.

Omdat we de aanname voor $m-1$ dimensies al bewezen hebben geldt nu:
\[
X_m[\boldsymbol k]  = \frac 1 {n_m}\sum_{j_m=1}^{n_m} 
\left( \frac 1 {n_{m-1}} \sum_{j_{m-1}=1}^{n_{m-1}} \left ( \cdots 
\frac 1 {n_1} \sum_{j_1=1}^{n_1} 
x[j_1,\ldots,j_m] 
\cdot s^{-1}_{k_1}[j_1]
\cdots \right ) s^{-1}_{k_{m-1}}[j_{m-1}] \right) 
s^{-1}_{k_m}[j_m],
\]
wat precies is wat we wilden aantonen.
\end{proof}

Een belangrijk gevolg van de definitie van deze algoritme is dat de DFT-term in \eqref{mdft_cases}
gemakkelijk vervangen kan worden door een FFT-term. Beiden geven immers dezelfde output. 
Daarmee hebben we ook direct een MFFT gevonden.

\begin{opmerking}
Omdat bovenstaand algoritme niet in lineaire tijd loopt, neemt de wachttijd snel toe bij grote signalen en hogere dimensie. 
Dit wordt al snel een praktisch bezwaar en is dan ook een reden waarom we geen driedimensionale 
signalen hebben bekeken bij de implementatie van de Fouriertransformatie.
\end{opmerking}

%-----------------------------------------------------------------------------------------------------------------
\section{Compressie van een signaal onder FFT}
\label{daling_FFT}
Tot zover is de Discrete Fourieranalyse besproken met de gedachte van perfecte reconstructie,
aan de hand van een complete set co\"effici\"enten. Het doel van dit project is echter om
signalen te comprimeren: te reconstrueren aan de hand van een gelimiteerde dataset.
Om deze analyse te versimpelen, zullen in deze sectie een aantal inzichten aan bod komen die
de fout van zo'n reconstructie relateren aan de grootte van de dataset.

\iffalse
We zullen het convergentiegedrag gaan bepalen van de Fouriertransformatie voor functies die $C^k$ zijn.
We weten immers dat voor differentieerbare functies de Fourieranalyse perfect reconstrueerbaar is.
Met het convergentie gedrag van deze functies kunnen we dan kwantificeren hoe goed een functie te 
benaderen is met een eindige subset van de Fourier-getransformeerde.
We trachten daarom te bewijzen dat er een algebra\"isch verband bestaat tussen de fout en de frequentie
van de wavelets en bovendien een exponentieel verband tussen de fout en de `gladheid' van de functie.
\fi

\begin{lemm}[Riemann-Lebesque \cite{fourier-fout}]
\label{lebbie}
Wanneer $g \in L_1(\R)$, dan geldt 
\eq{
	G(z) = \left|\int_{-\infty}^\infty g(t) e^{- 2 \pi i z t} dt\right| \to 0 \text{ voor } z \to \infty.
}
\end{lemm}

\begin{stelling}[Daling van de co\"effici\"enten van de Fouriergetransformeerde]
\label{fourier_daling}
  Laat $f \in L_2([a,b])$. Stel er is een $n$ z\'o dat $f \in C^n$ en voor $0\leq j\leq n$ geldt dat $f^{(j)}(a) = f^{(j)}(b)$ -- $f$ en haar afgeleides zijn periodiek. Dan geldt dat de Fouriergetransformeerde $\hat f[k]$ in absolute waarde
  daalt met $k$ volgens $o(|k|^{-n})$.
\end{stelling}
\begin{proof}[Bewijs]
  Laat $f \in L_2([a,b])$ zodat $f$ voldoet aan de voorwaarden in de stelling voor een bepaalde $n$. 
  De Fouriergetransformeerde van $f$ wordt gegeven door:
  \eq{
    \hat f [k] = \tfrac{1}{\sqrt{b-a}} \inpr{f}{\phi_k}.
  }
  De constante in deze vergelijking heeft geen invloed op de orde, dus richten we ons op het inproduct. 
  We schrijven dit uit tot een integraal en voeren vervolgens -- $f$ is immers differentieerbaar -- parti\"ele
  integratie uit:
  \eq{
    \abso{\inpr{f}{\phi_k}} 
    =& \abso{\int_a^b f(x) e^{- 2 \pi i k \tfrac{x-a}{b-a}} dx}
    = \abso{\tfrac{b-a}{2 \pi i k}\left[ f(x) \cdot  e^{- 2 \pi i k \tfrac{x-a}{b-a}} \right]_a^b} 
    + \abso{\frac{b-a}{2 \pi i k}\right| \left|\int_0^1 f'(x) e^{-2 \pi i k \tfrac{x-a}{b-a}} dx} \\
    =& \abso{ f(b) \cdot 1 - f(a) \cdot 1} + \frac{b-a}{2 \pi \abso{k}} 
    \abso{ \int_a^b f'(x) e^{-2 \pi i k \tfrac{x-a}{b-a}} dx }
    = \tfrac{b-a}{2 \pi |k|} \left| \int_a^b f'(x) e^{-2 \pi i k \tfrac{x-a}{b-a}} dx \right|.
  }
  Hier hebben we gebruikt dat alle afgeleiden van $0$ tot $n$ periodiek zijn.
  Omdat $f$ van de klasse $C^n$ is en de afgeleiden weer periodiek zijn kunnen we dit herhaald toepassen en we verkrijgen daarmee
  \[
  \left| \int_a^b f(x) e^{-2 \pi i k \tfrac{x-a}{b-a}} dx \right| 
  = (\tfrac{2 \pi}{b-a} |k|)^{-n}\left| \int_a^b f^{(n)}(x) e^{- 2 \pi i k \tfrac{x-a}{b-a}} dx \right|.
  \]
  
  Vermenigvuldig beide kanten met $(\tfrac{2 \pi}{b-a} |k|)^n$ om te vinden dat
  \[
  (\tfrac{2 \pi}{b-a} |k|)^n \abso{\inpr{f}{\phi_k}} 
  = \abso{\int_a^b f^{(n)}(x) e^{- 2 \pi i k \tfrac{x-a}{b-a}} dx}.
  \]
  We willen graag lemma \ref{lebbie} toepassen op de rechterkant. 
  Merk daartoe op dat $f$ continu is op een gesloten interval, met als gevolg dat zij uniform continu is 
  en zij haar maximum en minimum hier dus aanneemt.

  Daarmee is de integraal van $|f|$ begrensd en dus is $f\in L_1([a,b])$. 
  Maar een functie die integreerbaar is op $[a,b]$ en daarbuiten nul, is integreerbaar op $\R$. 
  Dus we mogen Riemann-Lebesgue gebruiken om te zien dat
  \[
  (\tfrac{2 \pi}{b-a} |k|)^n \abso{\inpr{f}{\phi_k}} \to 0 \text{ als } |k| \to \infty.
  \]
  Maar dit betekent precies dat $\abso{\inpr{f}{\phi_k}} \in  o((\tfrac{2 \pi}{b-a} |k|)^{-n}) = o(|k|^{-n})$.
\end{proof}

De analyse van de co\"effici\"enten die we hier gegeven hebben vertaalt direct naar de fout die we krijgen
bij reconstructie van een Fouriergetransformeerde functie aan de hand van een kleinere set co\"effici\"enten.

\begin{gevolg}
Schrijf $f|_{N}$ voor de reconstructie met de eerste $N$ basisfuncties. Dan hebben we de relatie
\[
  || f - f|_N||^2_{L_2([a,b])} = \sum_{k=1}^\infty |\inpr{f-f|_N}{\phi_k}|^2 = \sum_{k=1}^\infty |\inpr{f}{\phi_k} 
  - \inpr{f|_N}{\phi_k}|^2 = \sum_{k=N+1}^\infty |\inpr{f}{\phi_k}|^2,
\]
vanwege de Parsevalgelijkheid \eqref{parseval}. Door stelling \ref{fourier_daling} zijn er $k_0$ en $c$ zodat $\abso{\inpr{f}{\phi_k}} < c\cdot k^{-n}$ voor $k > k_0$. Dus
\[
	\sum_{k=N+1}^\infty | \langle f, \phi_{k} \rangle |^2 \leq \sum_{k=N+1}^\infty c \cdot k^{-2n} < c \cdot \int_{N}^\infty x^{-2n} dx = c \cdot \left[ \frac{x^{1-2n}}{1-2n} \right]^\infty_N = c \cdot \frac{N^{1-2n}}{2n-1}.
\]
Dan
\[
N^{2(n-1)} \cdot \sum_{N+1}^\infty |\inpr{f}{\phi_k}|^2 < N^{2(n-1)} \cdot c \cdot \frac{N^{1-2n}}{2n-1} = \frac{c}{2n-1} N^{-1} \to 0 \quad \text{als } N\to\infty,
\]
wat precies betekent dat
 \[
\|f-f|_{N}\|^2_{L_2([a,b])} \in o\left ( N^{-2(n-1)} \right) 
\implies ||f - f|_{N}||_{L_2([a,b])} \in o\left(N^{-(n-1)}\right).
\]
\end{gevolg}

\subsection{De fout in het discrete geval}
Wanneer we een discrete functie bekijken zullen we het criterium van \emph{gladheid} niet kunnen gebruiken.
We beroepen ons daarom op de analogie tussen het discrete en non-discrete geval.

De gladheid van een discrete functie $f:A\to\R$ wordt bepaald door de verschillen \mbox{$f[x]-f[x+1]$},
wanneer de verschillen klein zijn en de verzameling $A$ groot is komt dit overeen met de afgeleide van een 
continue functie en zeggen we dat de functie glad is. Zo kunnen we analoga geven voor hogere afgeleiden.

We zullen in onze resultaten de discrete fout in kaart brengen door de zogenaamde PSNR-functie te gebruiken,
een logaritmische schaal die werkt over discrete signalen, welke ook als elementen uit $\ell_2$ kunnen worden beschouwd. Zie hiervoor hoofdstuk \ref{sectie_psnr}.

