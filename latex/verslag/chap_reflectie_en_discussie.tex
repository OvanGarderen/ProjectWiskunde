\chapter{Reflectie en discussie}
In dit laatste hoofdstuk zullen we een analyse maken van het praktische deel van ons project.

\section{Discussie}
\subsection{Gebruik van PSNR als maatstaf voor kwaliteit}
Omdat we te maken hebben met `lossy' beeldcompressie dienen we dit verlies in kaart te brengen op een 
objectieve en kwantitatieve manier. 
De PSNR is een gebruikelijke grootheid voor dit soort analyses, er wordt gezegd dat dit voor eenzelfde 
beginsignaal een goede reflectie geeft van de menselijke beleving van de kwaliteit van de reconstructie.
Uit onderzoek\cite{PSNR} blijkt dat de PSNR een goede maatstaf is zolang er in zeker zin \emph{ceteris paribus}
is aangehouden; voor hetzelfde beeldmateriaal presteert het goed.

\subsection{Bruikbaarheid van resultaten}

De resultaten beslaan slechts een kleine set beelmateriaal, dit heeft vooral met tijdgebrek te maken gehad.
We zullen geen uitspraak doen over een bepaalde beelcategorie (zoals boven beschreven) omdat hier simpelweg te
weinig studie naar is verricht, echter menen we dat de set beelden gevari\"eerd genoeg is om algemene uitspraken te doen
over de prestatie van de verschillende algoritmes.

\section{Verklaring resultaten}

\subsection{Prestatie van Fourier vs. Wavelets}
Uit de PSNR grafieken blijkt dat de Fouriertransformatie een slechtere reconstructie levert voor dezelfde dataset
dan zowel de Haar als de Daubechies(2/4) wavelettransformaties, ongeacht de catagorie waartoe het beeld behoort.

\subsection{Wavelet vs. Wavelet}
Tussen de twee wavelets zien we verder wat we theoretisch verwachten, Haar presteert goed bij harde randen en grote kleurvlakken
vanwege zijn kleine drager, terwijl Daubechies voor fotografisch materiaal een goede reconstructie geeft door zijn grote drager.
We kunnen dit echter niet hard maken doordat we te weinig beeldmateriaal hebben bekeken in de verschillende catagori\"een. 

\subsection{Convergentie bij kleine dataset}
Uit de grafiek blijkt dat het verschil in kwaliteit tussen de verschillende compressie algoritmes afneemt wanneer
de hoeveelheid data die opgeslagen wordt klein wordt. 
Door de kleine dataset is de willekeur van de vorm van het plaatje erg van invloed op kwaliteit van de reconstructie 
voor verschillende algoritmes. 
Ook is het dubieus of de PSNR nog een goede maatstaf is voor de menselijke perceptie wanneer de algoritmes
zo'n slechte maar zeer verschillende benadering geven van het originele beeld.
We zullen daarom hier geen uitspraak over kunnen doen.

