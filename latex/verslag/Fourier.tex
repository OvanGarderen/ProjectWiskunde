%We moeten even beslissen welke van de twee we willen gebruiken
\documentclass[11pt]{report}
%\documentclass[11pt]{amsart} 
\usepackage[demo]{graphicx}
\usepackage{subfigure}

\usepackage{amssymb}
\usepackage{amsmath}
\usepackage{amsthm}

\usepackage{parcolumns}

% PYTHON CODE
\usepackage{listings}
\usepackage{color}

\usepackage{caption}

\DeclareCaptionFont{black}{ \color{black} }
\DeclareCaptionFormat{listing}{
  \parbox{\textwidth}{\hfill#3}
}
\captionsetup[lstlisting]{ format=listing, textfont=black, singlelinecheck=false, margin=0pt, font={footnotesize} }

\definecolor{dkgreen}{rgb}{0,0.6,0}
\definecolor{gray}{rgb}{0.5,0.5,0.5}
\definecolor{mauve}{rgb}{0.58,0,0.82}

\lstset{
  language=Python,
  aboveskip=3mm,
  belowskip=3mm,
  frame=single,
  showstringspaces=false,
  columns=flexible,
  basicstyle=\small,
%  numbers=none,
  numberstyle=\small\color{gray},
  keywordstyle=\small\color{blue},
  commentstyle=\small\color{dkgreen},
  stringstyle=\small\color{mauve},
  breaklines=true,
  breakatwhitespace=true,
  captionpos=b,
  tabsize=3
}
%PYTHON CODE

\usepackage[dutch]{babel}
\usepackage{a4wide}
\usepackage{algpseudocode}
\usepackage{algorithmicx}
\usepackage[square,super]{natbib}

\makeatletter
%replaces : \def\@endtheorem{\endtrivlist\@endpefalse }
% with:
%\def\@endtheorem{\endtrivlist}
\makeatother

\newcommand{\R}{\mathbb{R}}
\newcommand{\N}{\mathbb{N}}
\newcommand{\Z}{\mathbb{Z}}
\newcommand{\C}{\mathbb{C}}
\newcommand{\A}{\mathbb{A}}
\newcommand{\Q}{\mathbb{Q}}
\newcommand{\F}{\mathbb{F}}
\newcommand{\e}{\epsilon}
\renewcommand{\O}{\mathcal{O}}

\newcommand{\FFT}{\text{FFT}}

\newtheorem*{lemm}{Lemma}
\newtheorem*{stelling}{Stelling}
\newtheorem*{algo}{Algoritme}
\newtheorem*{definitie}{Definitie}
\newtheorem*{gevolg}{Gevolg}

\theoremstyle{remark}
\newtheorem*{opmerk}{Opmerking}

\newcommand{\eq}[1]{\begin{eqnarray*} #1 \end{eqnarray*}}
\newcommand{\mogelijkheden}[1]{\begin{cases} #1 \end{cases}}
\newcommand{\repr}[1]{{#1}^{\!\!-1}}

\newcommand{\coefficient}{co\"effici\"ent}
\newcommand{\coefficienten}{co\"effici\"enten}

\newcommand{\dx}{\text{d}x}
\newcommand{\dy}{\text{d}y}
\newcommand{\dz}{\text{d}z}
\newcommand{\largediv}{\,\big|\,}
\newcommand{\Ldnorm}[1]{{||#1||_{L_2}}}
\newcommand{\inpr}[2]{\langle #1 , #2 \rangle}
\newcommand{\DFT}{\text{DFT}}
\newcommand{\dpii}{{2\pi i}}
\newcommand{\abso}[1]{{\left| #1 \right|}}

\setlength\parindent{0pt}
\parskip = \baselineskip
\setcounter{tocdepth}{2}
\setcounter{secnumdepth}{1}

\begin{document}
\tableofcontents 
\newpage
\chapter{Fourier}

Bij het analyseren van periodieke functies is de Fourier-transformatie het gereedschap bij uitstek.
Het is een manier om een signaal te karakteriseren aan de hand van een bereik aan frequenties.
Dit maakt het een overtuigende manier om `nette' periodieke functies te beschrijven.
We kunnen de eis van periodiciteit ook loslaten als we naar functies op een interval kijken door dit op te vatten
als \'e\'en fase van een periodieke functie.

De basisfuncties waarmee we deze analyse uitvoeren worden gegeven door de complexe $e$-machten.
Hiermee zijn de functies te transformeren naar een vector op $\C$.
\begin{definitie}[FourierBasis] Zij gegeven een interval $[a,b]$, en bekijk de functieruimte $L_2([a,b])$. 
Laat de verzameling $F_{[a,b]}$ gedefinieerd zijn door:
\eq{
  F_{[a,b]} := \left\{ \phi_k(x) = \tfrac{1}{\sqrt{b-a}} e^{2 \pi i \cdot k \frac{x-a}{b-a}} \largediv k \in \Z \right\}
}
Dan noemen we $F_{[a,b]}$ de Fourier`basis' van $L_2([a,b])$
\end{definitie}
Merk op dat hier het woord \emph{basis} hier tussen aanhalingstekens staat aangezien de elementen van $F_{[a,b]}$ 
complexe functies zijn, ze spannen een grotere ruimte op dat nodig voor deze re\"ele analyse.
We kunnen echter wel degelijk een functie ontbinden in deze Fourierbasis en deze vervolgens bijna-overal reconstrueren.
We zullen hier een aantal handige definities voor introduceren.

\begin{definitie}[Fouriergetransformeerde]
Zij gegeven een functie $f\in L_2({[a,b]})$, schrijf dan een vector $\hat f : \Z\to\C$ met entries gedefinieerd volgens (hier aangegeven met blokhaken):
\eq{
  \hat f [n] = \frac{1}{\sqrt{b-a}} \cdot \inpr{f}{\phi_n} = \frac{1}{b-a} \int_{[a,b]} f(x) \cdot e^{-2 \pi i \cdot k \frac{x-a}{b-a}}\dx.
}
We noemen $\hat f$ de Fouriergetransformeerde van $f$.
\end{definitie}
\begin{definitie}[Inverse Fouriertransformatie]
  Gegeven een Fouriergetransformeerde $\hat f$ van een functie $f \in L_2([a,b])$ is de reconstructie van $f$ mogelijk volgens:
  \eq{
    f(x) = \sum_{k=-\infty}^\infty \hat f [k] \phi_k(x)
  }
  In alle $x \in [a,b]$ met uitzondering van een aantal geisoleerde punten. We noemen dit de inverse Fouriertransformatie.
\end{definitie}

Het kan worden aangetoond dat deze inverse Fouriertransformatie perfecte reconstructie geeft wanneer de functie
$f$ continu differentieerbaar is.\cite{fourier-rec} We zullen 

%-----------------------------------------------------------------------------------------------------------------
\section{De \emph{Discrete Fourier Transform}}
Zoals we gezien hebben in het vorige hoofdstuk kan de Fouriertransformatie gebruikt worden om continue signalen te karakteriseren voor verschillende frequenties. Een groot gebied binnen de signaalanalyse is echter van discrete aard aangezien hier toch veelal digitale instrumenten voor worden gebruikt. De toepassingen waar wij op zoek zijn liggen ook in dit digitale domein (JPEG is immers een beeldcompressie-algoritme) en dus zal ons verslag zich verder afspelen in deze discrete setting.

Om discrete signalen te analyseren lijkt het in eerste instantie voor de hand te liggen om het discrete signaal als stapfunctie te bekijken, deze is immers ook kwadratisch integreerbaar. Dit leidt echter tot ongewenste resultaten, zoals is te zien aan de reconstructie van een blokgolf door middel van Fouriertransformatie. Erger nog, elke eindige som van continue functies is weer continu dus voor een perfecte reconstructie van een discreet signaal is met deze methode altijd een oneindige rij \coefficient en nodig. Bovendien is het moeilijk om uit deze \coefficient en relevante informatie te destilleren over de aard van het signaal.

In plaats van de discrete signalen in te bedden in $L_2$ zullen we een discreet analogon gebruiken om de Fouriertransformatie van praktisch nut te laten zijn in de analyse van discrete signalen. 
Hiervoor zullen we de Fourier-basis moeten discretiseren en ons moeten richten op de ruimte $\R^n$.
We doen dit op zo'n manier dat deze discretisatie in de limiet naar het continue geval overgaat. 
Zo zullen we uiteindelijk tot de \emph{Discrete Fourier Transform} komen.

\begin{definitie}[Discrete Fourierbasis] Gegeven de ruimte $\R^n$, definieer dan de verzameling
\eq{
  S_n := \left\{ s_k [j] = e^{\dpii k j/n } \largediv k,j \in \{0, \ldots, n-1\} \right\}
}
Als zijnde de discrete Fourierbasis met basisvectoren $s_k$.
\end{definitie}
Met deze basis in de hand kunnen we vervolgens de discrete Fouriergetransformeerde (DFT) van een signaal schrijven als de vector van (complexe) inproducten van dit signaal met de DFT-basisfuncties. Laat voor een signaal $x$ van lengte $n$ de DFT (aangeduid met $X$) gegeven zijn door:
\eq{
  X[k] = \tfrac{1}{n} \inpr{x}{s_k} \quad k\in\{0, \ldots ,n-1\}
}
Ook defini\"eren we een operatie op zulke vectoren $X$ en beweren dat deze een inverse DFT -- ook wel iDFT -- voorstelt (dit bewijs volgt weldra in een volgende sectie). De iDFT wordt gegeven door
\eq{
  x[j] = \inpr{X}{s_j^{-1}} \quad j\in \{0, \ldots, n-1\}
}
We zullen nu enkele bewijzen geven om de term `Discrete Fourierbasis' te ondersteunen.

\begin{stelling}[Fourierbasis]
  De vectoren $s_k$ zoals gedefinieerd voor de verzameling $S_k$ staan onderling loodrecht onder het complexe
  inproduct en $S_k$ is dus een complexe basis voor $\C^n$.
\end{stelling}
\begin{proof}[Bewijs]
  Bekijk het complexe inproduct tussen twee willekeurig gekozen elementen van deze verzameling:
  \eq{
    \inpr{s_k}{s_j} = \sum_{m\leq n} s_k[m]\cdot \overline{s_j[m]} = \sum_{m\leq n} e^{\dpii\cdot m (k-j)/n} =
    \begin{cases}
      0 \quad \text{als } k\neq j\\
      n \quad \text{als } k = j
    \end{cases}
  }
  Dus staan de vectoren van $S_n$ onderling loodrecht, zodat deze een complexe ruimte opspannen van dimensie $n$.
\end{proof}

Om de claim te ondersteunen dat de DFT-methode een echte discretisatie is van de continue Fouriertransformatie willen we bewijzen dat dit algoritme voor een steeds fijnere selectie van waardes van een functie in de limiet hetzelfde resultaat geeft als de continue Fouriertransformatie. 

Zoals gebruikelijk bij het overschakelen van een discrete naar een continue setting, kunnen we dit doen 
door de definitie van de Riemann integraal toe te passen op de sommatie die voor handen ligt.

\begin{stelling}[Limiet van discrete Fourier-transformatie]
  Gegeven een interval $[a,b]$ en een functie $f\in L_2([a,b])$. 
  Voor een discretisatie van $f$ in $n$ gelijke
  intervallen zodat de discrete $f$ precies \'e\'en waarde uit elk interval inneemt geldt dat de discrete
  Fouriergetransformeerde limiteert naar de algemene Fouriergetransformeerde wanneer $n\to\infty$.
\end{stelling} 
\begin{proof}[Bewijs]
Gegeven een interval $[a,b]$ kunnen we een partitie $P$ maken in $n$ gelijke delen, ofwel laat $P=\{a=t_0,t_1,..,t_n=b\}$ met $t_j = \tfrac{na + j(b-a)}{n}$.
We discretiseren onze functie $f$ door uit elk interval $[t_{j-1},t_{j}]$ van de partitie de waarde in $x_j = t_j - c_j(\tfrac{b-a}{n})$ te nemen met $c_j \in [0,1]$, ofwel
$f[j] = f(\frac{b-a}{n}(j-c_j) + a)$.
De DFT van de discrete functie $f[\cdot]$ wordt dan gegeven door het inproduct te nemen met de DFT-basisfuncties:
\[
F[k] = \frac1n\sum_{j\leq n} f[j] \cdot \overline{s_k}[j].
\]
We schrijven dit om in termen van onze partitie, met de identificatie $n\cdot\tfrac{x_j-a}{b-a} + c_j= j$ 
waarmee we de continue variable $x_j\in[a,b]$ omschrijven naar zijn discrete tegenhanger en krijgen zo:
\eq{
  F[k] =& \frac{1}{n} \sum_{j \leq n} f[j]\cdot \overline{s_k}[n\cdot\tfrac{x_j-a}{b-a} + c_j] \\
       =& \frac{1}{n} \sum_{j \leq n} f(x_j)\cdot 
       e^{-\dpii \cdot k \tfrac{x-a}{b-a}} \cdot e^{-\dpii \cdot k c_j/n} \\
       =&  \frac{1}{\sqrt{b-a}}\sum_{j \leq n} f(x_j)\cdot \overline{\phi}_k(x_j) \cdot\frac{b-a}{n} 
\cdot e^{-\dpii \cdot k c_j/n}
}
We merken op dat $c_j$ begrensd dus we weten dat de limiet van 
$e^{-\dpii\cdot k c_j/n}$ voor $n\to\infty$ gelijk is aan $1$ voor elke $j$. 
Dan is de limiet voor $F[k]$ gelijk aan de limiet van de overige factoren in de sommatie.
We merken op dat de term $\frac{b-a}{n}$ precies de grootte is van ons interval en dat we het geheel
omgeschreven hebben in termen van onze continue functies $f$ en $\phi_k$.
Omdat het product $f\cdot\phi_k$ integreerbaar is moet voor elke partitie $P$ met 
waarden in punten $x_j$ uit elk interval deze sommatie convergeren naar de integraal
\eq{
  \frac{1}{\sqrt{b-a}} \int_{[a,b]} f(x) \overline{\phi}_k(x) dx
}
wanneer we de maaswijdte ($\tfrac{b-a}{n}$) naar $0$ laten gaan. Dit is duidelijk het geval wanneer we de limiet $n\to\infty$ nemen. Daarmee is de DFT een goede discretisatie van de Fouriergetransformeerde. 
\end{proof}
 
Voor de werking van het DFT algoritme als signaalcompressie-algoritme is het van belang dat er een 
inverse algoritme bestaat dat het getransformeerde inputsignaal weer terugtransformeert 
zonder verlies van informatie. We zullen nu bewijzen dat dit mogelijk is door de DFT en iDFT te gebruiken.

\begin{stelling}[Inverteerbaarheid van de DFT]
  De samenstelling van de inverse Discrete Fouriertransformatie met de Discrete Fouriertransformatie is
  gelijk aan de identiteit.
\end{stelling}
\begin{proof}[Bewijs]
We bekijken daarvoor een signaal $x\in\R^n$ dat met de DFT methode getransformeerd wordt in een vector $X\in\C^n$.
Hierbij hoort de discrete Fourier-basis $S_n$ met basisvectoren $s_k$. 
De discrete Fourier transformatie en zijn inverse zijn gedefini\"eerd door:
\eq{
  X[k]      =& \tfrac 1 n \langle x, s_k \rangle \quad\quad \text{(DFT)}\\
  \hat x[j] =&  \langle X, \repr{s_j} \rangle \quad\quad\text{(iDFT)}
}
Om te bewijzen dat de iDFT de DFT inverteert voldoet het om de iDFT uit te schrijven in termen van $x$ 
door $X$ in te vullen volgens de formule van de DFT. Uitschrijven hiervan geeft:
\eq{
  \hat x[j] =&  \langle X , \repr{s_j} \rangle \\
      =&  \sum_{i=1}^n X[i]\cdot \overline{\repr{s_j}}[i] \\
      =&  \sum_{i=1}^n \tfrac1n\inpr{x}{s_i} \overline{\repr{s_j}}[i] \\
      =& \langle x , \frac1n \sum_{i=1}^n s_i \repr{s_j}[i] \rangle
}
We schrijven nu de sommatie in de tweede term van het inproduct (die een vector van lengte $n$ representeert) uit in zijn componenten. 
Merk op dat de discrete Fourier-basisvectoren zo gedefinieerd zijn dat $s_k[m] = s_m[k]$. 
We verwisselen de indices in blokhaken met die in het subscript van $s$ en verkrijgen daarmee
\eq{
  \frac1n \sum_{i=1}^n s_i[m] \repr{s_j}[i] = \frac1n \sum_{i=1}^n s_m[i] \overline{s_j}[i] 
=\tfrac1n \inpr{s_m}{s_j} = \begin{cases}0\quad\text{als }m\neq j\\1\quad\text{als }m=j\end{cases}.\\
}
Hier gebruiken we dus de eerdere vergelijking voor het inproduct van twee basisfuncties.
We gebruiken deze eigenschap van de sommatie om ons eerder inproduct uit te schrijven. 
We hebben nu een gelijkheid
\eq{
  \hat x_j = \inpr{x}{e_j} = x_j
}
Door de iDFT toe te passen op het resultaat van de DFT zullen we altijd het origineel terug krijgen,
dus de iDFT samengesteld met de DFT geeft een identiteit.
\end{proof}

%-----------------------------------------------------------------------------------------------------------------
\section{De Fast Fourier Transform}
De snelheid van het DFT algoritme valt in de praktijk nogal tegen, het nemen van $n$ inproducten over vectoren 
van lengte $n$ heeft namelijk een tijdscomplexiteit van $\O(n^2)$. Dit staat de directe implementatie van de DFT 
voor praktische toepassingen in de weg. Daarom is er een alternatief algoritme, de \emph{Fast Fourier Transform}.  \bigskip

%%%%%
\begin{algo}[Fast Fourier Transform]
Gegeven zij een inputsignaal $x$ van lengte $n=2^m$, dan geeft het algoritme $\FFT$ 
een lijst terug van waardes $X$ van lengte $n=2^m$ als volgt:

Als $m=0$ dan geeft de $\FFT$ de lijst (van \'e\'en element) direct terug:
\eq{
X = x.
}
Wanneer $m\neq0$ splitsen we de lijst $x$ op in lijsten $\e,o$ van zijn even en oneven indices:
\eq{
  \e[k]   =& x[2k]   &\quad \text{voor } k < n/2\\
   o[k]   =& x[2k+1] &\quad \text{voor } k < n/2
}
Vervolgens voeren we hierop het $\FFT$ algoritme uit om de volgende lijsten te verkrijgen:
\eq{
  E =& \FFT(\e) \\
  O =& \FFT(o)
}
Hiermee wordt de output van het algoritme geconstrueerd als volgt:
\eq{
  X[k] = \left\{\begin{array}{llll}
    E[k]         &+& e^{-\dpii k/n}\cdot O[k] &  k< n/2 \\
    E[k-n/2] &-& e^{-\dpii (k-n/2)/n}\cdot O[k-n/2] &  k\geq n/2 
  \end{array}\right.
}
\end{algo}
%%%%%

Dit is dus een recursief gedefinieerd algoritme dat een signaal meermaals halveert en in zichzelf terugvoert.
Het is gegarandeerd dat dit algoritme afloopt vanwege de conditie op $n=0$ samen met de halvering van de input bij elke stap. Een belangrijke voorwaarde voor de relevantie van de FFT is nu dat het algoritme hetzelfde resultaat geeft als het DFT algoritme en dit zullen we nu dan ook bewijzen. 

\begin{stelling}[]
  Het uitvoeren van het Fast Fourier Transform algoritme op een dataset van lengte $n=2^m$ geeft
  dezelfde getransformeerde als de discrete Fouriertranformatie.
\end{stelling}
\begin{proof}[Bewijs]
We gebruiken hier een inductief bewijs met inductie naar $n$. Onze aanname is dat het FFT-algoritme voor $x$ van lengte $n=2^m$ gelijk is aan de DFT van $x$, ofwel
\eq{
  X[k] = \sum^{N}_{k=1} x[j] \cdot e^{-2\pi i \cdot jk/n}
}
Dit geldt duidelijkerwijs wanneer $m=0$, onze basistap. Hiervoor geldt namelijk:
\eq{
  X[k] = x[k] = x[1] = \sum^{2^0}_{k=1} x[j] \cdot e^{-2\pi i \cdot 1/2^0}
}
Vervolgens passen we inductie toe naar $m$ door onze aanname voor $m-1$ te gebruiken,
we vullen hiermee $E[k]$ en $O[k]$ in de vergelijking voor $X[k]$ in, deze hebben immers lengte $n=2^{m-1}$.
\eq{
  X[k] = \left\{\begin{array}{llll}
    \sum^{n/2}_{j=1} \e[j] 
    \cdot e^{-2\pi i \cdot kj \cdot 2/n} &+& 
    e^{-2\pi i \cdot k/n}
    \sum^{n/2}_{j=1} o[j] 
    \cdot e^{-2\pi i\cdot kj \cdot 2/n} &  k< n/2 \\
    \sum^{n/2}_{j=1} \e[j] 
    \cdot e^{-2\pi i\cdot (k-n/2) j\cdot 2/n} &-& 
    e^{-2\pi i\cdot (k-n/2)/n}
    \sum^{n/2}_{j=1} o[j] \cdot e^{-2\pi i\cdot (k-n/2)j\cdot 2/n} &  k\geq n/2 
  \end{array}\right.
}
We merken op dat we de e-machten in het tweede geval kunnen vereenvoudigen volgens
\eq{
  e^{-2\pi i\cdot (k-n/2)j \cdot 2/n} 
  = - e^{-2\pi i\cdot kj\cdot 2/n} \quad,\quad e^{-2\pi i\cdot(k-n/2)/n} 
= -e^{-2\pi i\cdot k/n},
}
waardoor het gevalsonderscheid wegvalt, aangezien beide vergelijkingen nu identiek zijn.
We verkrijgen $X[k]$ als sommatie over de lijsten $e$ en $o$, we vullen de relatie voor $e,o$ met $x$ in, en nemen de factor voor de oneven indices mee in de sommatie.
\eq{
  X[k] = \sum^{n/2}_{j=1} x[2j] \cdot e^{-2\pi i\cdot k (2j)/n} + 
    \sum^{n/2}_{j=1} x[2j+1] \cdot e^{-2\pi i\cdot k (2j+1)/n} 
    = \sum^n_{j=1} x[j] \cdot e^{-2\pi i\cdot k j/n}
}
Dit bewijst dat de FFT hetzelfde resultaat levert als het DFT algoritme, het bewijs voor de gelijkheid van de iDFT en de inverse FFT is 
hetzelfde wanneer men de substitutie $-2\pi i \rightarrow 2\pi i$ uitvoert.
\end{proof}

\begin{opmerk}
We hebben hier telkens aangenomen, en zullen deze aanname ook doorzetten, 
dat de lengte van het ingangssignaal een macht van $2$ is. Dit is een belangrijke
eigenschap waar de variant van het FFT-algoritme dat hier gebruikt wordt door werkt. Deze versie van FFT wordt
de Radix-2 Decimation In Time van het Cooley-Tukey FFT algoritme genoemd. Algemenere vormen van het algoritme
worden ook toegepast in geoptimaliseerde algoritmes maar om de implementatie te versimpelen 
is voor Radix-2 gekozen. Eventuele verschillen in afmetingen tussen een signaal en 
een 2-macht zijn opgelost met signaalextensie, zoals eerder beschreven.
\end{opmerk}

%-----------------------------------------------------------------------------------------------------------------
\subsection{Complexiteit van de Fast Fourier Transform}
Hoewel het vanuit een puur wiskundig oogpunt wellicht minder relevant is, is de complexiteit van een algoritme 
als een factor van belangrijk praktisch nut. De complexiteit vertaalt namelijk direct naar 
de looptijd van het algoritme. We hebben hier de volgende stelling uit de complexiteitstheorie nodig. 
\footnote{Dit is een speciaal geval van de stelling, voor de volledige stelleing en het bewijs zie \cite{akra-bazzi}.}

\begin{stelling}[Akra-Bazzi]
    Zij $T(n)$ een recurrente betrekking van de vorm
    \[
    T(n) = \begin{cases}
      c &\text{ als } n \leq d \\
      a T(n/b) + f(n) &\text{ anders}, \\
    \end{cases}
    \]
    waarbij $a,b,d\in\N$, $c\in\R$ en $f$ een functie $f:\N\rightarrow\R$ die voldoet aan 
    \eq{
  \exists k \in \N \,:\, f(n) \in \theta(n^{\log a/\log b} \log^k n).
    }
    Dan wordt de orde van $T(n)$ gegeven door:
    \eq{
      T(n) \in \theta(n^{\log a / \log b} \log^{k+1}n).
    }
\end{stelling}
We beschouwen hier $T(n)$ als het aantal stappen dat een machine nodig heeft om het algoritme uit te voeren.
Deze stelling is voldoende om een uitspraak te kunnen doen over de complexiteit 
\begin{stelling}[Complexiteit van de FFT]
  Het Fast Fourier Transform algoritme heeft een tijds-complexiteit $\O(n\log n)$ voor input van lengte $n=2^m$ 
\end{stelling} 
\begin{proof}[Bewijs]
We schrijven het FFT algoritme in pseudocode.

\begin{algorithmic}
\Function{FFT}{$x$}
\State $n \gets \text{lengte}(x)$ \Comment Assumptie: $n = 2^m$ voor een $m$
\If {$n == 1$}
	\State{$X \gets x$}
\Else
	\State $E \gets FFT(x[0::2])$ \Comment{FFT op even indices}
	\State $O \gets FFT(x[1::2])$ \Comment{FFT oneven indices}
	\For{$i = 0$ to $n-1$}
		\If{$i < n/2$}
			\State $X[i] \gets E[i] + e^{-2i \pi k/n} \cdot O[i]$
		\Else
			\State $X[i] \gets E[i] - e^{-2i \pi k/n} \cdot O[i]$
		\EndIf
	\EndFor
\EndIf
\State \Return{$X$}
\EndFunction
\end{algorithmic}

Dit algoritme is recursief, dus kunnen we de complexiteit schrijven door middel van een recurrente betrekking. Laat hiervoor $T(n)$ het aantal berekeningen zijn dat het algoritme kost bij een invoersignaal van lengte $n$. We maken een gevalsonderscheid: als de lijst lengte $1$ heeft geven we deze direct terug (1 berekening). Bij een lijst van lengte $>1$ splitsen we de lijst op in de even en oneven entries en voeren we op beiden weer het FFT algoritme uit, vervolgens voeren we nog $n$ maal een vast aantal ($c$) berekeningen uit om tot het eindresultaat te komen. In formulevorm geeft dit de \emph{recurrente betrekking}
\[
T(n) = \begin{cases}
    1 &\text{ als } n = 1 \\
      2\cdot T(n/2) + c\cdot n &\text{ anders}. \\
\end{cases}
\]
We zullen nu bovenstaande (vereenvoudigde) stelling van Akra-Bazzi gebruiken. We rekenen hierbij niet de $\O$ van de FFT uit maar de strictere $\theta$ die gedefinieerd is volgens:
\[
f \in \theta(g) \Leftrightarrow \exists c \in \R_+: \lim_{n \to \infty} \frac{f(n)}{g(n)} = c
\]
Zonder verder al te veel in te gaan op de implicaties die $\theta$ heeft op het gedrag van de FFT gebruiken we dan dat in ieder geval geldt dat $f \in \theta(g)$ impliceert dat $f \in \O(g)$. 

De recurrente betrekking voor de complexiteit van de FFT is inderdaad van dezelfde vorm als die van $T(n)$ in bovenstaande stelling.
Laat hiervoor namelijk $a=b=2$, $c=d=1$ en $f$ de functie die $n\mapsto n$ dan voldoet $f$ aan:
\eq{
  f(n) \in \theta(n^{\log 2/\log 2} \log^0 n)=\theta(n).
}
Dit betekent dat $T(n) \in \theta(n \log n)$ en dus zeker $T(n) \in \O(n \log n)$.
Hiermee hebben we bewezen dat de FFT en daarmee de iFFT binnen tijdscomplexiteit $\O(n\log n)$ lopen. 
\end{proof}

%-----------------------------------------------------------------------------------------------------------------
\section{Discrete Fourier Transform in meer dimensies}
Een eigenschap van het DFT-algoritme is dat het op een natuurlijke manier uit te breiden is naar hogere dimensies.
Per extensie daarvan is er een manier om het FFT-algoritme mee te laten schalen, die we ook zullen behandelen.
Het idee hierbij is om het Tensorproduct te nemen van meerdere bases.

\begin{definitie}[Multidimensionale Discrete Fourierbasis] 
Gegeven zij een signaalruimte van de vorm
$\R^{n_1} \times \R^{n_2} \times \ldots \times \R^{n_m}$ welke we opvatten als
een $m$-dimensionaal discreet signaal waarbij de $i$-de orthogonale richting een lengte $n_i$ heeft.
Dan defini\"eren we de multidimensionale discrete Fourier`basis' behorende bij deze signaalruimte als
\eq{
  S_{\bold n}= S_{n_1}\otimes S_{n_2} \otimes \ldots \otimes S_{n_m} = 
  \left\{s_{\bold k}[{\bold j}]  = s_{k_1}[j_1]\cdot s_{k_2}[j_2]\cdot\ldots\cdot s_{k_m}[j_m] 
  \largediv s_{k_i} \in S_{n_i}, j_i \in \{1,\ldots, n_i\} \right\}
}
Met basisvectoren $s_{\bold k}$, waar $\bold k$ een indexvector uit $\{0, \ldots, n_1\}\times\ldots\times\{0\ldots, n_m\}$ is.
\end{definitie}

Wanneer we nu een $m$-dimensionaal signaal bekijken van lengte $n_1$ bij $n_2$ etc., dan kunnen we
hierop een multidimensionale Fouriertransformatie op defini\"eren door het inproduct te generaliseren naar 
onze $m$-dimensionale signaalruimte.
\eq{
  X[\bold k] = \tfrac{1}{\bold n}\sum_{\bold j = \bold 1}^{\bold n} x[\bold j] \overline{s_{\bold k}}[{\bold j}] =
  \left(\prod_{i\leq m} \tfrac{1}{n_i}\right) \cdot \sum_{j_1=1}^{n_1} \ldots \sum_{j_m=1}^{n_m} x[j_1,\ldots,j_m] \cdot 
  e^{-\dpii\cdot k_m \cdot j_m /n_m } \cdot \ldots \cdot e^{-\dpii\cdot k_1 \cdot j_1 /n_1 } 
}
We zullen echter een andere definitie gebruiken die equivalent is aan deze method, maar recursief gedefinieerd.
We noteren deze transformatie als $\DFT_m$, een functie die een $m$-dimensionaal ingangssignaal transformeert.
\begin{definitie}[Multidimensionale Discrete Fouriertransformatie]
Definieer $\DFT_0$ als de identiteit op $\R^{n_1}$.
We defini\"eren dan een DFT van orde $m$ (notatie $\DFT_m$) als een transformatie 
$\R^{n_1}\times\ldots\times\R^{n_m} \to \C^{n_1}\times\ldots\times\C^{n_m}$ aan de hand van het beeld $X$ van een $m$-dimensionaal ingangssignaal $x$:
\eq{
  X[k_1,\ldots,k_m] = \tfrac{1}{n_m}\sum_{j_m=1}^{n_m} \DFT_{m-1}(x[j_m])[k_1,\ldots,k_{m-1}] \cdot s_{k_m}[j_m]  
}
Hier staat de notatie $x[j_m]$ voor een $(m-1)$-dimensionaal signaal dat we verkrijgen door de co\"ordinaat
in de $m$-de richting vast te zetten.
\end{definitie}
Dit is duidelijk dezelfde definitie als eerder gegeven in termen van herhaalde sommatie. Voor $m=1$ staat er immers de gewone DFT en voor hogere machten schaalt dit met een factor $\tfrac{1}{n}$ en een extra sommatie.
Het voordeel van deze definitie is dat het makkelijk om te schrijven is naar een algoritme dat 
herhaald $\DFT$ toepast.
\begin{algo}[Multidimensionaal DFT-algoritme]
Gegeven is een $m$-dimensionale input $x$ en een huidig level $t$, 
we schrijven het algoritme met output $X_t$ als:
\eq{
  X_{t}[k_1,\ldots,k_{t-1},k_{t+1},\ldots,k_m] = \DFT(X_{t-1}[k_1,\ldots,k_{t-1},k_{t+1},\ldots,k_m]) \\
  \text{voor } k_1\leq n_1, \ldots ,k_{t-1}\leq n_{t-1}, k_{t+1}\leq n_{t+1}, \ldots k_m \leq n_m
}
Waarbij we de randvoorwaarde $X_0 = x$ gebruiken. We beweren dat dit algoritme de multidimensionale Fouriertransformatie van een $m$-dimensionaal signaal geeft.
\end{algo}
\begin{proof}[Bewijs]
Om te bewijzen dat dit de MDFT geeft volstaat het om aan te tonen dat
\eq{
  X_t [k_1,\ldots,k_m] = 
  \sum_{j_t = 1}^{n_t}\DFT_{t-1}(x[j_t,k_{t+1},\ldots,k_m])[k_1,\ldots,k_{t-1}] \cdot s_{k_t}[j_t]
}
Wanneer dit geldt voor elke $t$ dan geldt dit namelijk in het bijzonder voor het geval $t=m$, 
zodat $X_m$ gegeven wordt voor de formule voor de MDFT.

Laat $t=1$ dan geldt:
\eq{
  X_1[k_1,\ldots,k_m] =& \DFT(X_0[k_2,\ldots,k_m])[k_1] \\
 =&  \sum_{j_1=1}^{n_1} x[j_1,k_2,\ldots,k_m]\cdot s_{k_1}[j_1] \\
 =& \sum_{j_1=1}^{n_1} \DFT_0(x[j_1,k_2,\ldots,k_m])\cdot s_{k_1}[j_1]
}
Nu geldt verder voor $t>1$:
\eq{
  X_t[k_1,\ldots,k_m] =& DFT(X_{t-1}[k_1,\ldots,k_{t-1},k_{t+1},\ldots,k_m])[k_t] \\
           =& \sum_{j_t =1}^{n_t} X_{t-1}[k_1,\ldots,k_{t-1},j_t,k_{t+1},\ldots,k_m] \cdot s_{k_t}[j_t] \\
           =& \sum_{j_t =1}^{n_t} 
           \sum_{j_{t-1} = 1}^{n_{t-1}}
           \DFT_{t-2}(x[j_{t-1},k_t,\ldots,k_m])[k_1,\ldots,k_{t-2}] \cdot s_{k_{t-1}}[j_{t-1}]
                         \cdot s_{k_t}[j_t] \\
=&  \sum_{j_t = 1}^{n_t}\DFT_{t-1}(x[j_t,k_{t+1},\ldots,k_m])[k_1,\ldots,k_{t-1}] \cdot s_{k_t}[j_t]
}
Waarbij we opeenvolgend onze de inductieaanname gebruiken en vervolgens de definitie van de MDFT.
Hiermee geldt de relatie dus ook voor $X_m$, zodat dit algoritme de MDFT geeft.
\end{proof}

Een belangrijk gevolg is nu dat het algoritme in termen van de DFT direct vertaalt naar een multidimentionaal algoritme voor de FFT: beiden geven immers dezelfde output. 
Wanneer we overal de functie DFT vervangen door FFT dan geeft dit de Multidimentionale Fast Fourier Transform.
De complexiteit van zo'n algoritme is $\O( \sum_{i=1}^m \log(n_i) \prod_{j=1}^m n_j)$ t.o.v.
$\O( \sum_{i=1}^m n_i \prod_{j=1}^m n_j)$ voor de MDFT.

%-----------------------------------------------------------------------------------------------------------------
\section{Compressie van signaal onder FFT}
Tot zover is de Discrete Fourieranalyse besproken aan de hand van perfecte reconstructie, 
aan de hand van een  complete set co\"efficienten. Het doel van dit project is echter om
signalen te comprimeren; te reconstrueren aan de hand van een gelimiteerde dataset.
Om deze analyse te vegemakkelijken, zullen in deze sectie een aantal bewijzen aan bod komen die
de geinduceerde fout van zo een reconstructie relateren aan de grootte van de dataset.

We zullen het convergentie-gedrag gaan bepalen van de Fourier-transformatie voor functies die $C^k$ zijn.
We weten immers dat voor differentieerbare functies de Fourieranalyse perfect reconstrueerbaar is.
Met het convergentie gedrag van deze functies kunnen we dan kwantificeren hoe goed een functie te 
benaderen is met een eindige subset van de Fourier-getransformeerde.
We trachten daarom te bewijzen dat er een algebra\"isch verband bestaat tussen de fout en de frequentie
van de wavelets en bovendien een exponentieel verband tussen de fout en de `gladheid' van de functie.

Voor dit bewijs zullen we het Lemma van Riemann-Lebesgue gebruiken: 
\begin{lemm}[Riemann-Lebesque{\cite{fourier-fout}}]
Wanneer $g \in L_1(\R)$, dan geldt 
\eq{
	G(z) = \left|\int_{-\infty}^\infty g(t) e^{- 2 \pi i z t} dt\right| \to 0 \text{ voor } z \to \infty.
}
\end{lemm}

\begin{stelling}[Daling van de co\"effici\"enten van de Fouriergetransformeerde]
  Laat $f \in L_2([a,b])$. Stel er is een $n$ z\'o dat $f \in C^n$ en voor $0\leq j\leq n$ geldt dat $f^{(j)}(a) = f^{(j)}(b)$. Dan geldt dat de $k$-de entry van de Fouriergetransformeerde $\hat f[k]$ in absolute waarde
  daalt met $k$ volgens $\O(|k|^{-n})$.
\end{stelling}
\begin{proof}[Bewijs]
  Laat $f \in L_2([a,b])$ zodat $f$ voldoet aan de voorwaarden in de stelling voor een bepaalde $n$. 
  De Fourier-getransformeerde van $f$ wordt gegeven door:
  \eq{
    \hat f [k] = \tfrac{1}{\sqrt{b-a}} \inpr{f}{\phi_k}.
  }
  De constante in deze vergelijking heeft geen invloed op de orde, dus richten we ons op het inproduct. 
  We schrijven dit uit tot een integraal en voeren vervolgens, omdat $f$ differentieerbaar is, partie\"ele
  integratie uit.
  \eq{
    \abso{\inpr{f}{\phi_k}} 
    =& \abso{\int_a^b f(x) e^{- 2 \pi i k \tfrac{x-a}{b-a}} dx}
    = \abso{\tfrac{b-a}{2 \pi i k}\left[ f(x) \cdot  e^{- 2 \pi i k \tfrac{x-a}{b-a}} \right]_a^b} 
    + \abso{\frac{b-a}{2 \pi i k}\right| \left|\int_0^1 f'(x) e^{-2 \pi i k \tfrac{x-a}{b-a}} dx} \\
    =& \abso{ f(b) \cdot 1 - f(a) \cdot 1} + \frac{b-a}{2 \pi \abso{k}} 
    \abso{ \int_a^b f'(x) e^{-2 \pi i k \tfrac{x-a}{b-a}} dx }
    = \tfrac{b-a}{2 \pi |k|} \left| \int_a^b f'(x) e^{-2 \pi i k \tfrac{x-a}{b-a}} dx \right|.
  }
  Hier hebben we gebruikt dat alle afgeleiden van $0$ tot $n$ periodiek zijn.
  Omdat $f$ nu $C^n$ is en de afgeleiden weer periodiek zijn kunnen we dit herhaald toepassen,
  we verkrijgen daarmee
  \[
  \left| \int_a^b f(x) e^{-2 \pi i k \tfrac{x-a}{b-a}} dx \right| 
  = (\tfrac{2 \pi}{b-a} |k|)^{-n}\left| \int_a^b f^{(n)}(x) e^{- 2 \pi i k \tfrac{x-a}{b-a}} dx \right|.
  \]
  
  Vermenigvuldig beide kanten met $(\tfrac{2 \pi}{b-a} |k|)^n$ om te vinden dat
  \[
  (\tfrac{2 \pi}{b-a} |k|)^n \abso{\inpr{f}{\phi_k}} 
  = \abso{\int_a^b f^{(n)}(x) e^{- 2 \pi i k \tfrac{x-a}{b-a}} dx}.
  \]
  We willen graag het lemma van Riemann-Lebesgue toepassen op de rechterkant. 
  Merk daartoe op dat omdat $f \in C^n$, $f^{(n)}$ continu op $[a,b]$ 
  dus neemt hier een maximum en een minimum aan. 
  Daarmee is de integraal van $|f|$ begrensd en dus is $f\in L_1([a,b])$. 
  Maar een functie die integreerbaar is op $[a,b]$ en daarbuiten nul, is integreerbaar op $\R$. 
  Dus we mogen Riemann-Lebesgue gebruiken om te zien dat
  \[
  (\tfrac{2 \pi}{b-a} |k|)^n \abso{\inpr{f}{\phi_k}} \to 0 \text{ als } |k| \to \infty.
  \]
  Maar dit betekent precies dat $\abso{\inpr{f}{\phi_k}} \in  O((\tfrac{2 \pi}{b-a} |k|)^{-n}) = O(|k|^{-n})$
\end{proof}

De analyse van de co\"effici\"enten die we hier gegeven hebben vertaalt direct naar de fout die we krijgen
bij reconstructie van een Fourier-getransformeerde functie wanneer we een gelimiteerde dataset gebruiken.
Schrijf namelijk $F$ voor de terug getransformeerde met een dataset waarbij $k$ gelimiteerd is, en $f$ voor
de perfecte reconstructie. Dan hebben we de relatie:
\eq{
  \Ldnorm{ f - F }^2 =& 
  \abso{\int_a^b \left (\sum_{k=1}^\infty \hat f [k] \phi_k(x) 
    - \sum_{k=1}^{m-1} \hat f[k] \phi_k(x) \right)^2\dx } \\
  =& \abso{\int_a^b \left( \sum_{k=m}^\infty \hat f [k] \phi_k(x) \right)^2\dx } \\
  \leq& \abso{\sum_{k=m}^\infty \hat f^2 [k] \cdot \int_a^b \phi_k^2(x) \dx}
  \leq \sum_{k=m}^\infty \abso{\hat f[k]}^2.
}
Omdat de orde van deze vergelijking in termen van $m$ bepaald wordt door de grootste term kunnen we zeggen 
dat de orde van de fout gelijk is aan de orde van het grootste co\"effici\"ent:
\eq{
  ||f-F||_{L_2([a,b])} \in \O\left(\abso{\hat f[m]}\right) = \O(m^{-n})
}
\subsection{De fout in discrete geval}

TODO: discreet $\to$ continu voor grote dataset, maak een definitie van smooth op discrete functies

%-----------------------------------------------------------------------------------------------------------------
\section{Implementatie van Fourier Transformatie in Python}
Met de theoretische basis uit voorgaande secties is de implementatie van het \emph{Fast Fourier Transform} algoritme
nu aan bod gekomen. In deze sectie zullen we uitwijden over enkele genomen hordes en keuzes, de resultaten volgen
in een latere sectie.

Voor de implementatie van de Fourier-transformatie is de taal Python gebruikt,
enkele redenen hiervoor waren:
\begin{itemize}
\item Voorzieningen om beeld en geluid te laden m.b.v. het \emph{scipy} pakket.
  Dit pakket liet ons bijvoorbeeld toe om beelden te laden als matrix van kleurwaarden en deze weer op te slaan
  nadat we deze gereconstrueerd hadden. Zo hoefden we ons niet te bekommeren om andere beeldcompressie implementaties.
\item Duidelijke relatie van wiskunde naar code. Python heeft een syntax die een wiskundige manier van denken en werken
  ondersteunt, in tegenstelling tot meer declaratieve talen zoals \emph{C} of object-georienteerde talen zoals \emph{Java}.
\item Goede abstractie van onderliggende processen. Python is een geinterpreteerde taal, dit betekent in concreto
  dat onze programma's draaien op een platform dat de allocatie van geheugen en rekenkracht. Dit vergemakkelijkt
  de implementatie en zorgt dat er makkelijk veranderingen zijn aan tee brengen in de code zonder het programma te crashen.
\end{itemize}
Het enige nadeel dat deze keuze teweeg heeft gebracht is dat de snelheid van al onze algoritmes tegenvalt
omdat we ons niet bezighouden met kleine optimalisaties en vertrouwen op de Python \emph{interpreter}.

\subsection{Fast Fourier Transform Implementatie}
TODO: even kijken of we dit zo moeten doen of dat we beter alles kunnen referen

De python code die gebruikt is om het FFT algoritme in te programmeren volgt precies het schema van Pseudocode
zoals beschreven in een vorige sectie.

\begin{lstlisting}[caption={FFT algoritme in Python, voert de pseudocode uit zoals in sectie (TODO)}]
def FFT( xs ):
    N = len(xs)
    if N <= 1:                  # randconditie
        return xs
    else:
        even = FFT(xs[0::2])    # voer FFT uit op even indices
        odd  = FFT(xs[1::2])    # voer FFT uit op oneven indices

        return [0.5*(even[k] + exp(-2j*pi*k/N)*odd[k]) for k in range(N/2)] + 
               [0.5*(even[k] - exp(-2j*pi*k/N)*odd[k]) for k in range(N/2)]
\end{lstlisting}

Hierbij dient uitgelegd te worden dat de $[x \text{ for } y] + [z \text{ for } w]$ notatie twee lijsten construeert en achter elkaar zet. Dit algoritme werkt, zoals eerder beschreven enkel op signalen waarvan de lengte
$N$ een macht van $2$ is. We hebben daarom signaalextensie toe moeten passen door middel van zero-padding
om het programma ook op andersvormige signalen te laten werken.

\begin{lstlisting}[caption={Zero-Padding algoritme in Python, voegt nullen toe tot een tweemacht is bereikt}]
def Zero_Padding( xs ):
    N_old = len(xs)
    N_new = 2**ceil(log(N_old,2))         # rond logaritme af voor kleinste tweemacht
    return [xs[k] if (k < N_old) else 0 for k in range(N_new)]
\end{lstlisting}

Vervolgens hebben we deze code toegepast in twee dimensies. We maken hier gebruik van de definitie van het
MFFT algoritme, dat grofweg zegt dat een meerdimensionaal algoritme kan worden geconstrueerd door herhaald 
het 1-dimensionale geval toe te passen. Voor 2 dimensies in het bijzonder betekent dit dat we simpelweg 
FFT konden toepassen op rijen en kolommen.

\begin{lstlisting}[caption=2-Dimensionaal FFT algoritme]
def FFT_2D( xss ):
    xss = map(FFT, xss)        # voer FFT uit op rijen

    xss_t = transpose(xss)     # verwissel rijen met kolommen
    xss_t = map(FFT, xss_t)    # voer FFT uit op kolommen
    xss   = transpose(xss_t)   # maak verwisseling ongedaan 

    return xss
\end{lstlisting}
Hier is de python-functie \emph{map} gebruikt die ruwweg gedefinieerd is als 
$\text{map}(f,x) = [f(y) \text{ for } y \text{ in } x]$
Met het oog op duidelijkheid is hier de Zero-Padding fase weggelaten, 
dit algoritme verwacht nu nog een $2^n \times 2^m$ matrix.
Dit is echter gemakkelijk te implementeren door Zero\_Padding toe te passen op rijen en kolommen.

TODO :: 
Dingen die nog moeten.
\begin{itemize}
\item Iets over de conversie van matrix naar dictionary en de compressiestap 
\item Iets over de manier waarop beelden geconstrueerd zijn (0 - 255 en dan 4 kanaals)
\item Niet uit te breiden naar 3D $\to$ te langzaam.
\end{itemize}

\begin{thebibliography}{11}

\bibitem{akra-bazzi}
  Akra M., Bazzi L.
  \emph{On the solution of linear recurrence equations}.
  Computational Optimization and Applications, 
  10(2):195 - 210,
  1998.

\bibitem{fourier-fout}
  Bochner S., Chandrasekharan K. 
  \emph{Fourier Transforms}. 
  Princeton University Press.
  1949.

\bibitem{fourier-rec}
  Carleson L.
  \emph{On convergence and growth of partial sums of Fourier series}. 
  Acta Mathematica,
  116 (1): 135 – 157,
  1966.

\end{thebibliography}

\end{document}
