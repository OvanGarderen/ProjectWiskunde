\chapter{Basisbegrippen}
In dit verslag zullen een aantal termen worden ge\"introduceerd die wellicht niet bekend zijn. De eerste en meest belangrijke notie is dat wij \emph{signalen} analyseren. Een signaal is door \cite{signal} gedefinieerd als
\begin{quote}
Een signaal een functie die informatie over de kenmerken of het gedrag van een of ander fenomeen overdraagt.
\end{quote}

Enkele voorbeelden van een signaal zijn:
\begin{description}
	\item[Geluid] Geluid is de vibratie van een medium (zoals lucht) en een geluidssignaal associeert een bepaalde druk met elk moment in de tijd (en mogelijk op elk punt in de ruimte). Geluid wordt door een microfoon omgezet naar een elektrisch signaal welke weer ge\emph{sampled} kan worden tot een discrete lijst waardes in de tijd. Hiermee is geluid een eendimensionaal signaal.
	\item[Afbeeldingen] Een afbeelding bestaat uit een helderheidssignaal als functie van twee dimensies. Een afbeelding kan bestaan uit een continu domein, zoals bij een schilderij of een analoge foto, of een discreet raster zoals op de computer. Een kleurenafbeelding bestaat meestal uit drie kanalen, een voor de helderheid van elke primaire kleur.
	\item[Video] Een video is een lijst van afbeeldingen en een punt in een video wordt op deze manier gekarakteriseerd door een punt in de tijd samen met een punt in het vlak. Bewegend beeld is hiermee een driedimensionaal signaal met een al dan niet discreet domein.
\end{description}

In het vervolg zullen we de termen `signaal' en `functie' door elkaar gebruiken, tenzij anders aangegeven. Tijdens het project hebben we elk van deze drie voorbeelden bekeken, hoewel de twee- en driedimensionale signalen in ons verslag onze focus zullen krijgen.

\section{Signaaluitbreiding}
\label{signaal}
Beide algoritmes die we zullen behandelen kunnen enkel omgaan met signalen die een tweemacht lang zijn. Om te zorgen dat een willekeurig signaal ook getransformeerd kan worden, moet het dus uitgebreid worden voorbij zijn definitiegebied. De meeste bronnen onderscheiden de volgende manieren om het signaal uit te breiden.\cite{mallat,pywt} Laat $x_1, x_2, \ldots x_n$ het signaal.

Er zijn twee types die grote sprongen in het signaal kunnen veroorzaken (de discrete variant van een \emph{discontinu\"iteit}).
\begin{description}
\item[Zero-padding] $x' = 0, \ldots, 0| x_1, x_2, \ldots, x_n| 0, \ldots, 0$;
\item[Periodic padding] $x' = x_1, \ldots, x_n| x_1, x_2, \ldots, x_n| x_1, \ldots, x_n$.
\end{description}

Daarnaast zijn er nog twee types die een niet-vloeiende overgang kunnen veroorzaken (de discrete variant van een \emph{discontinue afgeleide}).
\begin{description}
\item[Constant padding] $x' = x_1, \ldots, x_1| x_1, x_2, \ldots, x_n| x_n, \ldots, x_n$;
\item[Symmetric padding] $x' = x_n, \ldots, x_1| x_1, x_2, \ldots, x_n| x_n, \ldots, x_1$.
\end{description}

De keuze van de \emph{signal extension mode} kan gevolgen hebben voor de mogelijkheid tot compressie op de rand. Verder in het verslag (zie stelling \ref{daling_fourier} en sectie \ref{daling_wavelets}) wordt duidelijk dat de functie $f$ moet voldoen aan bepaalde continu\"iteitseisen. Wanneer hier niet aan wordt voldaan, is het gevolg meestal dat compressie slecht mogelijk is.

\section{Notatie}
TODO: ik snap dit niet hoor :s -- jan
De blokhaaknotatie $f[x]\in K$ duidt op een discrete functie $A\subset \Z\to K$ met $K$ een willekeurige ruimte.
Als $A$ bovendien eindig is, dan is $f$ ook te karakteriseren aan de hand van zijn beeld, een vector in $K^{N}$.
In het bijzonder is deze notatie dus een manier om een vector aan te duiden.
Evenzo kunnen we de notatie uitbreiden naar meer dimensies door te schrijven
$f[x_1,\ldots,x_n] : A_1\times \cdots \times A_n\subset \Z^n \to K$, waarbij de $A_i$'s meestal eindige intervallen zijn. 

\section{Complexiteit}
Aangezien er enkele algoritmes behandeld worden in het verslag willen we hiervan de tijdscomplexiteit bepalen.
Deze eigenschap bepaalt namelijk hoe de tijd die het een machine kost oploopt met de grootte van de input.
We voeren daarom $o$, $\O$ en $\theta$ als notatie in.
\begin{eqnarray*}
  f \in o(g) \Leftrightarrow \forall \epsilon > 0 \quad  \exists x_0(\epsilon)\,: & \forall x > x_0(\epsilon) \quad |f(x)| \leq \epsilon |g(x)|  \\
  f \in \O(g)     \Leftrightarrow \exists x_0,c \,:&\,\forall x>x_0 \quad|f(x)| \leq c|g(x)|  \\
  f \in \theta(g) \Leftrightarrow \exists x_0,k_1,k_2 \,:&\,\forall x>x_0 \quad k_1|g(x)| \leq |f(x)|\leq k_2|g(x)|.
\end{eqnarray*}
Merk op dat $o$ een \emph{sterker} begrip is dan $\O$: als $f \in o(g)$ dan $f \in \O(g)$. Verder betekent $f \in \theta(g)$ zo iets als `$f$ gaat, op een constante na, precies even snel omhoog als $g$'.

We zullen in de praktijk algoritmes $f$ bekijken die een 1,2 of 3 dimensionale lijst data als input hebben.
We zijn dan ge\"interesseerd in de tijdscomplexiteit in termen van de grootte van bijvoorbeeld een $N\times N$ 
afbeelding, dus is $f$ een functie van $N$.
