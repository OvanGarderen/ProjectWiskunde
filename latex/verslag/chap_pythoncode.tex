\chapter{Pythoncode}
\label{muhappendix}

\begin{lstlisting}[label=fft,float=h!,caption={FFT algoritme in Python, voert de pseudocode uit zoals in sectie~\ref{fft_sec}}]
def FFT( xs ):
    N = len(xs)
    if N <= 1:                  # randconditie
        return xs
    else:
        even = FFT(xs[0::2])    # voer FFT uit op even indices
        odd  = FFT(xs[1::2])    # voer FFT uit op oneven indices

        return [0.5*(even[k] + exp(-2j*pi*k/N)*odd[k]) for k in range(N/2)] + 
               [0.5*(even[k] - exp(-2j*pi*k/N)*odd[k]) for k in range(N/2)]
\end{lstlisting}

\begin{lstlisting}[label=zeropad,float=h!,caption={Zero-padding algoritme in Python, voegt nullen toe tot een tweemacht is bereikt}]
def zero_pad( xs ):
    N_old = len(xs)
    N_new = 2**ceil(log(N_old,2))         # rond logaritme af voor kleinste tweemacht
    return [xs[k] if (k < N_old) else 0 for k in range(N_new)]
\end{lstlisting}

\begin{lstlisting}[label=fft2d,float=h!,caption=2-Dimensionaal FFT algoritme]
def FFT_2D( xss ):
    xss = map(FFT, xss)        # voer FFT uit op rijen

    xss_t = transpose(xss)     # verwissel rijen met kolommen
    xss_t = map(FFT, xss_t)    # voer FFT uit op kolommen
    xss   = transpose(xss_t)   # maak verwisseling ongedaan 

    return xss
\end{lstlisting}

\begin{lstlisting}[label=cutoff,float=h!,caption=Het vinden van een goede cutoff-waarde gegeven een gewenst compressieniveau]
def find_cutoff( mat, ratio ):
	if ratio < 0 or ratio > 1: return
	length = reduce( lambda x, y: x*y, array.shape ) #aantal cellen in de matrix 
	vector = reshape(array, length) 				 #reshape naar 1 lange vector
	sort(modulus(vector)) 							 #sorteer de vector
	return vector[ int( ratio*len(vector) ) ]
\end{lstlisting}

\begin{lstlisting}[label=mat2dict,float=h!,caption=Matrix naar dictionary conversie]
def mat2dict( mat, cutoff ):
	N, M = mat.shape
	dict = {}
	for y in range(N):
		for x in range(M):
			if modulus( mat[y][x] ) >= cutoff:
				dict[(y,x)] = mat[y][x]
	return (dict, N, M)
\end{lstlisting}

\begin{lstlisting}[label=dict2mat,float=h!,caption=Dictionary naar matrix conversie]
def dict2mat( dict, N, M ):
	mat = []
	for y in range( N ):
		row = []
		for x in range( M ):
			i = (y, x)
			row[x] = dict[i] if i in dict else 0.0
		mat[y] = row
	return mat
\end{lstlisting}

\begin{lstlisting}[label=next,float=h!,caption=De FWT]
def next(signal):
	low = downsampling_convolution( signal, dec_l )
	hi = downsampling_convolution( signal, dec_h )
	return concatenate(low, hi)
	
def fwt(signal):
	signal = zero_pad(signal)
	steps = int( log( len(signal), 2 ) ) #neem de 2-log van de lengte
	
	for i in range( steps ):
		k = len(output) / 2**i #pak eerst alles, dan de helft, dan een kwart etc.
		signal[0:k] = next( signal[0:k] )
	return signal
	
\end{lstlisting}

\begin{lstlisting}[label=prev,float=h!,caption=De iFWT]
def prev(signal):
	N = len(signal)
	low = upsampling_convolution( signal[:N//2], rec_l )
	hi = upsampling_convolution( signal[N//2:], rec_h )
	return  sum(low, hi) 	#elementsgewijze sommatie
	
def ifwt( wavelet, signal, original_length ):
	steps = int( log( len(signal), 2) ) #neem de 2-log van de lengte
	
	for i in range( steps ):
		j = steps - i - 1 #we gaan andersom
		k = len(signal) / 2**j
		signal[0:k] = prev( signal[0:k])
	return signal[0:original_length] #het signaal was aangevuld tot een tweemacht
\end{lstlisting}

\begin{lstlisting}[label=fwt2d,float=h!,caption=De Mallatdecompositie in 2 dimensies]
def next_2d( signal ):
	x, y = signal.shape
	for j in range(x): #1d-fwt op de rijen
		signal[j, :] = next( signal[j, :] )
	for i in range(y): #1d-fwt op de kolommen
		signal[:, i] = next( signal[:, i] )
	return signal

def fwt2d(wavelet, signal):
	signal = zero_pad_2d(signal)
	steps = int( log( len(signal), 2 ) ) #neem de 2-log van de lengte
	
	for i in range( steps ):
		k = len(output) / 2**i #pak eerst alles, dan de helft, dan een kwart etc.
		signal[0:k,0:k] = next_2d( signal[0:k,0:k] )
	return signal
\end{lstlisting}

\begin{lstlisting}[label=ifwt2d,float=h!,caption=De inverse Mallatdecompositie in 2 dimensies]	
def prev_2d( signal ):
	x, y = signal.shape
	for j in range(x): #1d-ifwt op de rijen
		signal[j, :] = prev( signal[j, :] )
	for i in range(y): #1d-ifwt op de kolommen
		signal[:, i] = prev( signal[:, i] )
	return signal

def ifwt2d( signal, N, M ): #N en M zijn originele lengte en breedte
	steps = int( log( len(signal), 2) ) #neem de 2-log van de lengte
	
	for i in range( steps ):
		j = steps - i - 1 #we gaan andersom
		k = len(signal) / 2**j
		signal[0:k,0:k] = prev_2d( signal[0:k,0:k] )
	return signal[0:N,0:M]
\end{lstlisting}

\begin{lstlisting}[label=tensorimpl,float=h!,caption=De 2D FWT in het Tensorgeval]
def fwt2d_tensor(signal):
	signal = zero_pad(signal)
	steps = int( log( len(signal), 2 ) ) #neem de 2-log van de lengte
	
	for i in range( steps ):
		k = len(output) / 2**i #pak eerst alles, dan de helft, dan een kwart etc.
		for p in range( len( signal ) ):
			signal[p, 0:k] = next( signal[p, 0:k] )
		for q in range( len( signal ) ):
			signal[0:k, q] = next( signal[0:k, q] )
	return signal

def ifwt2d_tensor( signal, N, M ):
	steps = int( log( len(signal), 2) ) #neem de 2-log van de lengte
	
	for i in range( steps ):
		j = steps - i - 1 #we gaan andersom
		k = len(signal) / 2**j
		for q in range( len( signal ) ):
			signal[0:k, q] = prev( signal[0:k, q] )
		for p in range( len( signal ) ):
			signal[p, 0:k] = prev( signal[p, 0:k] )
	return signal[0:N,0:M] #het signaal was aangevuld tot een tweemacht
\end{lstlisting}

\begin{lstlisting}[label=mallat3d,float=h!,caption=De Mallatdecompositie in 3 dimensies]
def next_3d( signal ):
	x, y, z = signal.shape
	for j in range(x): #2d-fwt op de rijen
		signal[j, :, :] = next_2d( signal[j, :, :] )
	for k in range(y):
		for l in range(z): #1d-fwt op de kolommen
			signal[:,k,l] = next( signal[:,k,l] )
	return signal

def fwt3d(wavelet, signal):
	signal = zero_pad_3d(signal)
	steps = int( log( len(signal), 2 ) ) #neem de 2-log van de lengte in 1 richting
	
	for i in range( steps ):
		k = len(output) / 2**i #pak eerst alles, dan de helft, dan een kwart etc.
		signal[0:k,0:k,0:k] = next_3d( signal[0:k,0:k,0:k] )
	return signal

def prev_3d( signal ):
	x, y, z = signal.shape
	for j in range(x): #2d-ifwt op de rijen
		signal[j, :, :] = prev_2d( signal[j, :, :] )
	for k in range(y):
		for l in range(z): #1d-ifwt op de kolommen
			signal[:,k,l] = prev( signal[:,k,l] )
	return signal

def ifwt3d( wavelet, signal, N, M, O ): #N, M en O zijn originele lengte, breedte en diepte
	steps = int( log( len(signal), 2) )
	
	for i in range( steps ):
		j = steps - i - 1 #we gaan andersom
		k = len(signal) / 2**j
		signal[0:k,0:k,0:k] = prev_3d( signal[0:k,0:k,0:k] )
	return signal[0:N,0:M,0:O]
\end{lstlisting}

\begin{lstlisting}[label=mengertje,float=h!,caption=De mengvorm in 3 dimensies]
def fwt3d_mix(signal):
	signal = zero_pad_3d(signal)
	steps = int( log( len(signal), 2 ) ) #neem de 2-log van de lengte in 1 richting
	
	for i in range( steps ):
		m = len(output) / 2**i #pak eerst alles, dan de helft, dan een kwart etc.
		for j in range(len( signal )): #2d-fwt op de rijen
			signal[j, 0:k, 0:k] = next_2d( signal[j, 0:k, 0:k] )
		for k in range(len( signal )):
			for l in range(len( signal )): #1d-fwt op de kolommen
				signal[0:k,m,l] = next( signal[0:k,m,l] )

	return signal

def ifwt3d_mix( signal, N, M, O ): #N, M en O zijn originele lengte, breedte en diepte
	steps = int( log( len(signal), 2) )
	
	for i in range( steps ):
		j = steps - i - 1 #we gaan andersom
		m = len(output) / 2**j #pak eerst alles, dan de helft, dan een kwart etc.
		for n in range(len( signal )): #2d-fwt op de rijen
			signal[n, 0:k, 0:k] = prev_2d( signal[n, 0:k, 0:k] )
		for k in range(len( signal )):
			for l in range(len( signal )): #1d-fwt op de kolommen
				signal[0:k,m,l] = prev( signal[0:k,m,l] )
	return signal[0:N,0:M,0:O]
\end{lstlisting}
