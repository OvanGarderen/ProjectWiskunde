\documentclass[11pt]{amsart}

\usepackage[dutch]{babel}
\usepackage{a4wide}
%\setlength{\parindent}{0pt}

\newtheorem*{vraag}{Vraag}
\theoremstyle{definition}
\newtheorem*{uitwerking}{Uitwerking}

\newcommand{\R}{\mathbb{R}}
\newcommand{\N}{\mathbb{N}}
\newcommand{\Z}{\mathbb{Z}}
\newcommand{\C}{\mathbb{C}}
\newcommand{\A}{\mathbb{A}}
\newcommand{\Q}{\mathbb{Q}}
\newcommand{\F}{\mathbb{F}}
\newcommand{\f}{\varphi}
\newcommand{\e}{\varepsilon}
\renewcommand{\d}{\delta}

\newtheorem*{thm}{Lemma van Riemann-Lebesgue}

\begin{document}

\title{Fourier fout}
\author{Jan Westerdiep}
\maketitle

Laat $f \in L_2[0,1]$. Stel $f$ is $C^n$ en periodiek (voor alle afgeleides $f^{(j)}$ geldt $f^{(j)}(1) = f^{(j)}(0)$). Dan kunnen we iets zeggen over de mate van daling in de co\"effici\"enten.

Voor dit bewijs zullen we het Lemma van Riemann-Lebesgue gebruiken: TODO-bewijs of referentie.
\begin{thm}
Wanneer $g \in L^1(\R)$, dan geldt
\[
	G(z) = \left|\int_{-\infty}^\infty g(t) e^{- 2 \pi i z t} dt\right| \to 0 \text{ voor } z \to \infty.
\]
\end{thm}

Wij willen graag iets zeggen over $\langle f, \psi_k\rangle$, de $k$-de co\"effici\"ent van onze Fourierbasis. Dan:
\[
	|\langle f, \psi_k \rangle| = \left|\int_0^1 f(x) e^{- 2 \pi i k x} dx\right| = \left|\left[ f(x) \cdot \frac{-1}{2 \pi i k} e^{- 2 \pi i k x} \right]_0^1\right| + \left|\frac{1}{2 \pi i k}\right| \left|\int_0^1 f'(x) e^{-2 \pi i k x} dx \right|
\]
\[
	= | f(1) \cdot 1 - f(0) \cdot 1| + \frac{1}{2 \pi k} \left| \int_0^1 f'(x) e^{-2 \pi i k x} dx \right| = \frac{1}{2 \pi |k|} \left| \int_0^1 f'(x) e^{-2 \pi i k x} dx \right|.
\]
Omdat $f$ nu $C^n$ is, herhaal totdat:
\[
	\frac{1}{2 \pi k} \left| \int_0^1 f'(x) e^{-2 \pi i k x} dx \right| = (2 \pi |k|)^{-n}\left| \int_0^1 f^{(n)}(x) e^{- 2 \pi i k x} dx \right|.
\]
Vermenigvuldig beide kanten met $(2 \pi |k|)^n$ om te vinden dat
\[
	(2 \pi |k|)^n |\langle f, \psi_k \rangle| = \left| \int_0^1 f^{(n)}(x) e^{- 2 \pi i k x} dx \right|.
\]
We willen graag het lemma van Riemann-Lebesgue toepassen op de rechterkant. Merk daartoe op dat omdat $f \in C^n$, $f^{(n)}$ continu op $[0,1]$ dus neemt hier een maximum en een minimum aan. Dus is de integraal van $|f|$ begrensd en dus is $f$, $L_1[0,1]$. Maar een functie die integreerbaar is op $[0,1]$ en daarbuiten nul, is integreerbaar op $\R$. Dus we mogen Riemann-Lebesgue gebruiken om te zien dat
\[
	(2 \pi |k|)^n |\langle f, \psi_k \rangle| \to 0 \text{ als } |k| \to \infty \Rightarrow \langle f, \psi_k \rangle \in o((2 \pi |k|)^{-n}) \subset O((2 \pi |k|)^{-n}) = O(|k|^{-n}),
\]
oftewel een algebraische afname.

\end{document}
