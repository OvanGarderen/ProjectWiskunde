\documentclass[11pt]{article}

\usepackage{amssymb, amsmath}
\usepackage{a4wide}

\newcommand{\R}{\mathb1b{R}}
\newcommand{\N}{\mathbb{N}}
\newcommand{\Z}{\mathbb{Z}}
\newcommand{\C}{\mathbb{C}}
\newcommand{\A}{\mathbb{A}}
\newcommand{\Q}{\mathbb{Q}}
\newcommand{\F}{\mathbb{F}}
\newcommand{\e}{\epsilon}

\newcommand{\eq}[1]{\begin{eqnarray*} #1 \end{eqnarray*}}
\newcommand{\mogelijkheden}[1]{\begin{cases} #1 \end{cases}}

\setlength\parindent{0pt}

\begin{document}

Voor een lijst van lengte N, laat $s_k$ een vector zodat $(s_k)_n = e^{i2\pi k n /N}$.
(Merk op dat dus $(s_k)_n = (s_n)_k$))

Dan zijn de Discrete Fourier Transformatie en zijn inverse gedefini\"eerd door:
\eq{
  X_k = \langle x, s_k \rangle \\
  x_n = \frac 1 N \langle X, s_{-n} \rangle
}
Waar de functie $\langle . , . \rangle$ het complexe inproduct op $\C^N$ aangeeft

We gaan bewijzen dat toepassen van DFT en zijn inverse de identiteit oplevert.
Dit zou betekenen dat:
\eq{
  x_n &= \frac 1 N \langle X , s_{-n} \rangle = \frac 1 N \sum_{k=1}^N X_k\cdot s_{-n}(k) \\
      &= \frac 1 N \sum_{k=1}^N \langle x , s_k \rangle s_{-n}(k) \\
      &= \langle x , \frac1N \sum_{k=1}^N s_k s_{-n}(k) \rangle
}
Er is hier gelijkheid zodra:
\eq{
  \frac1N \sum_{k=1}^N s_k s_{-n}(k) =& e_n \\
  \frac1N \sum_{k=1}^N s_k(j) s_{-n}(k) =& \delta_{jn}\\
  \frac1N \sum_{k=1}^N e^{i2\pi k j /N} \cdot e^{-i2\pi k n/N} =& \delta_{jn} \\
  \frac1N \sum_{k=1}^N e^{i2\pi k (j-n) /N} =& \delta_{jn}
}
Hier is $\delta$ de kronicker delta functie.
Maar dit laatse feit volgt precies uit de orthogonaliteit van de complexe e-machten!

\end{document}
