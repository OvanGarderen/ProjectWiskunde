\documentclass[11pt]{amsart}

\usepackage{bbold}
\usepackage{a4wide}

\newcommand{\R}{\mathbb{R}}
\newcommand{\N}{\mathbb{N}}
\newcommand{\Z}{\mathbb{Z}}
\newcommand{\C}{\mathbb{C}}
\newcommand{\A}{\mathbb{A}}
\newcommand{\Q}{\mathbb{Q}}
\newcommand{\F}{\mathbb{F}}
\newcommand{\e}{\epsilon}
\renewcommand{\O}{\mathcal{O}}
\newcommand{\eq}[1]{\begin{eqnarray*} #1 \end{eqnarray*}}
\newcommand{\mogelijkheden}[1]{\begin{cases} #1 \end{cases}}
\newcommand{\abso}[1]{{|#1|}}

\setlength\parindent{0pt}

\title{}
\author{Okke van Garderen}

\begin{document}
\maketitle
\section{Afschatting Tensor co\"efficienten}
Doordat de wavelet functie op een niveau $\abso{\lambda}$ een element is van $W_{\abso{\lambda}-1}$
staat deze loodrecht op elk element uit $V_{\abso{\lambda}-1}$.
Schrijf dus $P_{V_j} : L_2([0,1]) \rightarrow V_j$ voor de projectie-afbeelding die een
functie $f$ afbeeldt op zijn orthogonale projectie in de ruimte $V_j$.
Dan geldt dat het inproduct tussen de wavelet functie en $f$ ook in andere vorm is te schrijven:
\eq{
  \langle f , \phi_\abso{\lambda} \rangle = \langle f - P_{V_j}(f) , \phi_\abso{\lambda} \rangle.
}
We zullen dit gebruiken om de inproducten tussen f en een tensor product af te schatten. \bigskip

Bekijk nu een wavelet in $2$ dimensies die is verkregen door het tensor product te nemen
van twee wavelet functies. We schatten het de co\"effiecient af van deze wavelet voor een 
functie $f$ in twee veranderlijken door het inproduct te nemen:
$c_{\abso{\lambda_1},\abso{\lambda_2}} = \langle f , \psi_{\abso{\lambda_1}} \otimes \psi_{\abso{\lambda_2}} \rangle$ 
Welnu, met de identiteit uit de vorige paragraaf schrijven we dit om tot 
$\langle (I - Q_{\abso{\lambda_1}-1})\otimes(I - Q_{\abso{\lambda_2}-1})(f),\psi_{\abso{\lambda_1}} \otimes \psi_{\abso{\lambda_2}} \rangle^2$
wat een inproduct over tensorproducten is, zodat we met Cauchy-Zwarz dit kunnen afschatten als zijnde kleiner dan
$||I-Q_{\abso{\lambda_1} -1}||_{H^d\rightarrow L_2} ||I-Q_{\abso{\lambda_2} -1}||_{H^d\rightarrow L_2} ||\psi_{\abso{\lambda_1}}\otimes\psi_{\abso{\lambda_2}}||$
\\{\Large TODO: Kijken hoe dit werkt enzo} \\
De orde van de fout wordt daarmee $2^{-\abso{\lambda_1} d} \cdot 2^{-\abso{\lambda_2} d} = 2^{-(\abso{\lambda_1}+\abso{\lambda_2})d}$
Mits hierbij geldt dat alle parti\"ele afgeleiden tot en met orde $d$ van $f$ continu zijn. \bigskip

Het aantal wavelets valt nu als volgt af te schatten. Stel we beschouwen alle wavelets tot een bepaald niveau $M$, namelijke alle wavelets zodat
$\abso{\lambda_1}+\abso{\lambda_2} \leq M$. Voor elke niveau $m$ zijn er $(m+1) \cdot 2^m$ wavelets ($2m$ wavelets voor elke $(\abso{\lambda_1},\abso{\lambda_2})$ met $m+1$ van zulke punten), zodat we de orde af kunnen schatten op het aantal wavelets
op niveau $M$, zodat het aantal wavelets 
\\{\Large TODO: kijk hier even naar}\\ 
%$N \in \O((M+1)\cdot2^M )$
$N \in \O(2^M)$

Daarmee is de orde van de fout in termen van het aantal gebruikte wavelets van orde $\O(N^{-d})$.
\end{document}
