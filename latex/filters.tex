\documentclass[11pt]{article}

\usepackage{bbold}
\usepackage{amssymb, amsmath}
\usepackage{a4wide}

\newcommand{\R}{\mathbb{R}}
\newcommand{\N}{\mathbb{N}}
\newcommand{\Z}{\mathbb{Z}}
\newcommand{\C}{\mathbb{C}}
\newcommand{\A}{\mathbb{A}}
\newcommand{\Q}{\mathbb{Q}}
\newcommand{\F}{\mathbb{F}}
\newcommand{\e}{\epsilon}

\newcommand{\eq}[1]{\begin{eqnarray*} #1 \end{eqnarray*}}
\newcommand{\mogelijkheden}[1]{\begin{cases} #1 \end{cases}}

\setlength\parindent{0pt}

\begin{document}

Wavelets zijn zo geconstrueerd dat de volgende relaties gelden:
\eq{
  \exists F \in \N :& \phi(x) = \sum_{k=-F}^F c_k \cdot \phi(2\cdot x + k) \\
 &\psi(x) = \sum_{k=-F}^F (-1)^k c_{1-k} \cdot \phi(2\cdot x + k)
}
Hier hoeven niet alle co\"efficienten $c_k$ ongelijk nul te zijn

We noemen deze lijst van $\{c_k | k\in[-F,F]\}$'s de highpass filter co\"efficienten en de lijst van $\{(-1)^k \cdot c_{1-k} | k\in[-F,F]\}$ de lowpass filter co\"efficienten.

Door gebruik te maken van de orthonormaliteit van de wavelets onder verschuiving kunnen
we deze co\"efficienten dan als volgt berekenen:
\eq{
  <\phi(x),\phi(2x+k)> =& \sum_{n=-F}^F c_n \cdot <\phi(2\cdot x + n),\phi(2x+k)> \\
                       =& \sum_{n=-F}^F c_n \cdot \mathbb{1}_{n=k} \\
                       =& c_k
}
\end{document}
