\documentclass[11pt]{amsbook}

\usepackage[dutch]{babel}
\usepackage{a4wide}
%\setlength{\parindent}{0pt}

\newtheorem*{vraag}{Vraag}
\theoremstyle{definition}
\newtheorem*{uitwerking}{Uitwerking}

\newcommand{\R}{\mathbb{R}}
\newcommand{\N}{\mathbb{N}}
\newcommand{\Z}{\mathbb{Z}}
\newcommand{\C}{\mathbb{C}}
\newcommand{\A}{\mathbb{A}}
\newcommand{\Q}{\mathbb{Q}}
\newcommand{\F}{\mathbb{F}}
\newcommand{\f}{\varphi}
\newcommand{\e}{\varepsilon}
\renewcommand{\d}{\delta}

\begin{document}

\title{Wavelet}
\maketitle

\chapter{fourier}
bladie

\chapter{wavelets}

\section{Intro}
Fourier, te grote support, nadeel -> wavelets.

Haar, Daubechies.

\subsection{Continuous Wavelets}
Wavelet function, scaling function. Orthogonaliteit

\subsection{Discrete case}
discontinuiteiten -> matplotlib.

Filter

\section{Discrete Wavelet Transformation en Fast}
Recursief toepassen (plaatje), approximation & detail coeffs.

\section{Compression}
plaatje met matplotlib

\section{More dimensions}
Image compression; JPEG-2000
\subsection{Mallat versus Tensor}
Plaatje

\section{Results}
Terminology: PSNR, implementatie (snippets)

\subsection{pt 1: plaatjes}
\subsection{pt 2: filmpjes}

-----

\chapter{laatste chappie}
\section{discussion}

BRONNEN

\end{document}
