\documentclass[11pt]{amsart}
\usepackage[demo]{graphicx}
\usepackage{subfigure}
\usepackage{amssymb}
\usepackage{parcolumns}

\usepackage[dutch]{babel}
\usepackage{a4wide}
%\setlength{\parindent}{0pt}

\newtheorem*{vraag}{Vraag}
\newtheorem*{uitwerking}{Uitwerking}
\newtheorem*{lemma}{Lemma}
\newtheorem*{stelling}{Stelling}
\newtheorem*{algoritme}{Algoritme}

\newcommand{\R}{\mathbb{R}}
\newcommand{\N}{\mathbb{N}}
\newcommand{\Z}{\mathbb{Z}}
\newcommand{\C}{\mathbb{C}}
\newcommand{\A}{\mathbb{A}}
\newcommand{\Q}{\mathbb{Q}}
\newcommand{\F}{\mathbb{F}}
\newcommand{\f}{\varphi}
\newcommand{\e}{\varepsilon}
\renewcommand{\d}{\delta}

\begin{document}

\section{Intro}
De Fouriertransformatie bestaat al honderden jaren en is een grote speler geworden in de \emph{signal processing}. Een groot nadeel van deze transformatie is dat zij slecht reageert op discontinue signalen.

In de loop van de vorige eeuw is een nieuwe transformatie ontstaan met een eigenschap die de Fouriertransformatie nooit kende. Deze noemt men nu ook wel de Wavelettransformatie.

Een wavelet is simpelweg een functie die voldoet aan
\[
  \int_{-\infty}^{\infty} \psi(t) dt = 0.
\]
Met deze functie $\psi$ kunnen we een familie functies $\psi_{u,s}$ bouwen door middel van schaling en translatie:
\[
  \psi_{u,s}(t) := \frac{1}{\sqrt{s}} \psi\left(\frac{t-u}{s}\right).
\]

Deze familie geeft aanleiding tot een Wavelettransformatie $W_f$ van $f$:
\[
  W_f(u,s) = \int_{-\infty}^\infty f(t) \psi^*_{u,s}(t) dt.
\]

Het is nu mogelijk om wavelets te construeren die met deze schaling en translatie een basis voor de $L_2(\R)$ vormen. Over het algemeen kijken we dan naar
\[
  \Psi := \left\{ \psi_{j,n}(t) = \frac{1}{\sqrt{2^j}} \psi\left( \frac{t - 2^jn}{2^j}\right) : (j,n) \in \Z^2 \right\}.
\]
De kunst is nu om de basiselementen loodrecht op elkaar te laten staan, zodat er een orthogonale (en dus een orthonormale) basis gevormd wordt. 

Figuurtje van voorbeeld TODODOO.

Het gevolg is nu dat we elke voldoend nette\footnote{Deze functie moet wel in $L_2(\R)$ zitten natuurlijk.} functie kunnen schrijven in deze basis:
\[
  f(t) = \sum_{j=-\infty}^{\infty} \sum_{n=-\infty}^{\infty} \langle f, \psi_{j,n} \rangle \psi_{j,n}(t),
\]
waarbij $\langle \cdot, \cdot \rangle$ het standaardinproduct op de $L_2(\R)$ aangeeft.

Het grote nadeel van de Fouriertransformatie maakt compressie van discrete signalen moeilijk. Veel van deze wavelets worden nu z\'o geconstrueerd dat dit probleem (deels) verholpen wordt. We zijn namelijk op zoek naar een wavelet die een eindige drager heeft. Het blijkt dat deze bestaat en dat er zelfs een hele grote verzameling wavelets is, elk met eigen gewilde eigenschappen.

Omdat wij naar de toepassing van wavelets binnen de beeldcompressie bekijken, zijn we natuurlijk vooral ge\"interesseerd in het discrete geval. We kijken dus naar de benadering van $f$ op een TODO. Dit geeft aanleiding tot een rij geneste ruimtes die uiteindelijk naar de $L_2(\R)$ toe gaat
\begin{equation}
  \label{multires}
  \{0\} \ldots \subset V_0 \subset V_1 \subset \ldots \subset L_2(\R)
\end{equation}
genaamd een multiresolutie.
\subsection{Multiresolutie}
Een rij geneste ruimtes $\{ V_j: j \in \Z \}$ zoals in \ref{multires} heet een multiresolutie wanneer voldaan wordt aan de volgende eigenschappen:
\[
\begin{array}{c}
  \forall j, k: f(t) \in V_j \implies f(t - 2^j k) \in V_j; \\
  \forall j: V_{j+1} \subset V_j; \\
  \forall j: f(t) \in V_j \iff f(t/2) \in V_{j+1}; \\
  \cap_{j=-\infty}^{\infty} V_{j} = \lim_{j\to\infty} V_j = \{0\}; \\
  \cup_{j=-\infty}^{\infty} V_j = \lim_{j\to-\infty} V_j = L_2(\R). \\
  TODO: \text{ Er is $\theta$ zo dat $\{ \theta(t-n): n \in \Z \}$ een Rieszbasis voor $V_0$ is.}
\end{array}
\]

Als voorbeeld bekijken we een multiresolutie van stuksgewijs constante functies. De ruimte $V_j$ wordt hiermee de verzameling van alle $g(t) \in L_2(\R)$ die constant zijn voor $t \in [n 2^j, (n+1)2^j)$ met $n \in \Z$. De basisfunctie $\theta$ voor $V_0$ wordt in dit geval $\theta(t) = 1_{[0,1]}$.

\section{Schalingsfuncties}
Gegeven zo'n Rieszbasis voor $V_0$ willen we graag een orthonormale basis voor $V_j$ construeren. 
TODO: gaan we dit bewijzen? -> stelling 7.1 Mallat
Deze $\theta$ geeft een $\phi$ volgens Mallat. Wanneer we nu defini\"eren
\[
  \phi_{j,n}(t) := \frac{1}{\sqrt{2^j}} \phi\left( \frac{t-n}{2^j} \right),
\]
is $\{ \phi_{j,n}: n \in \Z \}$ een orthonormale basis voor $V_j$.

\subsection{Benadering} De orthogonale projectie van $f$ op $V_j$ is, zoals we weten, de beste benadering van $f$ in $V_j$. Deze is nu te vinden door
\[
	P_{V_j} f = \sum_{n=-\infty}^\infty \langle f, \phi_{j,n} \rangle \phi_{j,n}.
\]
De co\"effici\"enten $a_j[n] = \langle f, \phi_{j,n} \rangle$ geven ons op deze manier een discrete benadering van $f$ op resolutie $2^{-j}$. TODO: convolutie hierbij?

Met de multiresolutie uit het eerdere voorbeeld kunnen we vinden dat $\{\theta(t-n): n \in \Z \}$ al een orthonormale basis is voor $V_0$, met als gevolg dat $\phi = \theta$.

\section{Filters}
TODO intro.

Per definitie van de multiresolutie weten we dat $V_j \subset V_{j-1}$. In het bijzonder geldt dat $2^{-1/2}\phi(t/2) \in V_1 \subset V_0$ en omdat $\{ \phi(t-n): n \in \Z\}$ een orthonormale basis voor $V_0$ is, kunnen we $2^{-1/2} \phi(t/2)$ nu schrijven als
\[
  \frac{1}{\sqrt{2}} \phi\left(\frac{t}{2}\right) = \sum_{n=-\infty}^{\infty} \left\langle \frac{1}{\sqrt{2}} \phi\left(\frac{t}{2}\right), \phi(t-n) \right\rangle \phi(t-n).
\]
Deze inproducten hebben een speciale naam, want de rij $\{h[n]: n \in \Z\}$ met
\[
  h[n] := \left\langle \frac{1}{\sqrt{2}} \phi\left(\frac{t}{2}\right), \phi(t-n) \right\rangle
\]
wordt nu ook wel de \emph{filter} van $\phi$ genoemd. TODO: low-pass filter intro? Andersom, gegeven een $h[n]$ die voldoet aan een aantal voorwaarden, is er een schalingsfunctie $\phi \in L^2(\R)$ waarvoor $h[n]$ de filter is. THEOREM 7.2 TODO?

Om weer terug te komen bij het doorlopende voorbeeld $\phi(t) = 1_{[0,1)}$, vinden we in dit geval dat
\[
  h[n] = \left\langle \frac{1}{\sqrt{2}} \phi\left(\frac{t}{2}\right), \phi(t-n) \right\rangle = \begin{cases} \frac{1}{\sqrt{2}} & \text{ als } n \in \{0,1\} \\ 0 & \text{ anders.} \end{cases}
\]

\section{Eindelijk: wavelets}
We weten dat $V_j$ bevat is in $V_{j-1}$. Laat nu $W_j$ het orthogonale complement van $V_j$ in $V_{j-1}$:
\[
	W_j \oplus V_j = V_{j-1}.
\]
De projectie van $f$ op $V_{j-1}$ kan dus geschreven worden als som van projecties:
\[
	P_{V_{j-1}} f = P_{V_j} f + P_{W_j} f.
\]
Omdat $V_j \subset V_{j-1}$ is alle informatie over $f$ die beschikbaar is in $V_j$, ook beschikbaar in $V_{j-1}$. Ook is het goed mogelijk dat door deze grovere benadering, informatie zoek gaat. Deze `details' worden op die manier zichtbaar in $P_{W_j} f$.

Het kan bewezen worden \cite{mallat}[T7.3] dat, gegeven een schalingsfunctie $\phi$ (en daarmee een filter $h$) er een functie $\psi$ bestaat zo dat 
\[
	\left\{ \psi_{j,n}(t) := \frac{1}{\sqrt{2}} \psi\left(\frac{t-2^jn}{2^j}\right) : n \in \Z \right\}
\] een orthonormale basis is voor $W_j$ en $\{ \psi_{j,n}: (j,n) \in \Z^2 \}$ een basis voor $L_2(\R)$. Deze functie is zo een orthogonale wavelet, omdat $W_j \perp V_j$.
Omdat nu $W_j \subset V_{j-1}$ en dus in het bijzonder $2^{-1/2} \psi(t/2) \in W_1 \subset V_0$ en omdat $\{ \phi(t-n): n \in \Z \}$ een orthonormale basis is voor $V_0$, kunnen we ook $2^{-1/2}\psi(t/2)$ in termen schrijven als:
\[
	\frac{1}{\sqrt{2}} \psi\left(\frac{t}{2}\right) = \sum_{n=-\infty}^{\infty} \left\langle \frac{1}{\sqrt{2}} \psi\left(\frac{t}{2}\right), \phi(t-n) \right\rangle \phi(t-n).
\]
Ook deze inproducten hebben een speciale naam: de rij $g[n]$ met
\[
	g[n] := \left\langle \frac{1}{\sqrt{2}} \psi\left(\frac{t}{2}\right), \phi(t-n) \right\rangle
\]
wordt nu ook wel de filter van $\psi$ genoemd. TODO: high-pass filter. In het bijzonder geldt dat TODO:bewijzen?
\[
	g[n] = (-1)^{1-n}h[1-n].
\]

Zoals nu wel duidelijk geworden is, wordt met een filter $h$ (die voldoet aan bepaalde eigenschappen die we hier verder niet zullen behandelen) een schalingsfunctie $\phi$ en een filter $g$ met waveletfunctie $\psi$ geconstrueerd.

Als laatste keren we nog een enkele keer terug naar het voorbeeld waarin $\phi(t) = 1_{[0,1)}$. We vinden met de gelijkheden uit voorgaande paragrafen dat
\[
\frac{1}{\sqrt{2}} \psi\left(\frac{t}{2}\right) = \sum_{n=-\infty}^{\infty} (-1)^{1-n}h[1-n] \phi(t-n),
\]
en omdat $h = \{ 0 \to \frac{1}{\sqrt{2}}; 1 \to \frac{1}{\sqrt{2}} \}$ zoals we eerder vonden, herschrijft dit tot
\[
\frac{1}{\sqrt{2}} \psi\left(\frac{t}{2}\right) = \frac{1}{\sqrt{2}}\left(\phi(t-1) - \phi(t)\right)
\]
met als gevolg dat
\[
	\psi(t) = \begin{cases} -1 & \text{ als } t \in [0,1/2) \\ 1 & \text{ als } t \in [1/2,1) \\ 0 & \text{ anders.} \end{cases}
\]

Deze wavelet $\psi$ wordt ook wel de Haarwavelet genoemd en is uitgevonden voor Alfred Haar in 1909, hoewel het onderzoeksgebied van de wavelets toen nog niet bestond. In het vervolg zullen we nog verdere aandacht aan deze wavelet besteden.

\section{In de praktijk}
Bij het kiezen of vinden van een wavelet is men over het algemeen op zoek naar bepaalde eigenschappen. Voor compressie zijn we op zoek naar een wavelet die voor bepaalde klassen functies een klein aantal grote co\"effici\"enten en een groot aantal kleine teweeg brengt: een soort concentratie van de co\"effici\"enten. Dit wordt vooral bepaald door drie factoren: gladheid van $f$ (waar we niks aan kunnen doen), de grootte van de drager (welke hierna aan bod komt) en de zogenaamde orde van de wavelet.

Wanneer de waveletfunctie loodrecht staat ($\langle \psi, q\rangle = 0$) op alle polynomen van graad $p-1$ of lager, spreken we van een wavelet van orde $p$. Dit komt overeen met te zeggen dat
\[
	\int_{-\infty}^\infty x^k \psi(x) dx = 0 \text{ voor } k \in \{ 0, \ldots p-1 \}.
\]

Gevolg van deze eigenschap is dat we van de functie $f$ elk polynoom van graad $p-1$ af mogen trekken zonder een verschil in inproduct:
\[
	\langle f, \psi_{j,n} \rangle = \langle f - q, \psi_{j,n} \rangle \text{ voor $q$ een polynoom van graad $p-1$}. 
\]
Intu\"itief is deze eigenschap natuurlijk gewild: we winnen immers een heel stel keuzevrijheden. We zullen dit argument in een volgende sectie formaliseren.

In de vorige sectie spraken we het verlangen uit om een wavelet met eindige drager te vinden zodat discontinu\"iteiten alleen lokaal zichtbaar zijn. We zullen hier de dragers van $h[n], \psi$ en $\phi$ aan elkaar relateren.


\subsection{Compacte drager} De schalingsfunctie $\phi$ heeft een compacte drager dan en slechts dan als $h[n]$ een compacte drager heeft, en deze zijn hetzelfde. Als de drager van $\phi$ gelijk is aan $[N_1,N_2]$ dan is de drager van $\psi$ gelijk aan $[(N_1 - N_2 + 1)/2, (N_2 - N_1 + 1)/2]$.
\begin{proof}[Bewijs 1] Als $\phi$ een compacte drager heeft dan $h[n]$ ook: we weten dat
\[
	h[n] = \left\langle \frac{1}{\sqrt{2}} \phi\left(\frac{t}{2}\right), \phi(t-n) \right\rangle,
\]
er kunnen maar eindig veel $n$ ongelijk nul zijn. Omgekeerd, als $h[n] \not= 0$ voor eindig veel $n$, dan zien we
\begin{equation}
\label{phi_t2}
	\frac{1}{\sqrt{2}} \phi\left(\frac{t}{2}\right) = \sum_{n=-\infty}^\infty h[n] \phi(t-n)
\end{equation}
 TODO: pag 966 van Daubechies "orthonormal bases of compactly supported wavelets".

Om deze dragers gelijk te krijgen, stel dat de drager van $h[n]$ gelijk $[N_1,N_2]$ is, en die van $\phi$ $[K_1, K_2]$. De drager van $\phi(t/2)$ is $[2K_1, 2K_2]$ en de drager van de rechterzijde van $\ref{phi_t2}$ is $[N_1 + K_1, N_2 + K_2]$. We concluderen dat $K_1 = N_1$ en $K_2 = N_2$.
\end{proof}
\begin{proof}[Bewijs 2]
Kijk nu naar
\[
\frac{1}{\sqrt{2}} \psi\left(\frac{t}{2}\right) = \sum_{n=-\infty}^{\infty} g[n] \phi(t-n) = \sum_{n=-\infty}^{\infty} (-1)^{1-n}h[-1-n] \phi(t-n).
\]
Met de informatie uit het begin van de stelling kunnen we de drager van de rechterkant vinden: $[N_1 - N_2 + 1, N_2 - N_1 + 1]$. De functie $\psi(t/2)$ is nu precies een dilatie met factor twee dus de drager van $\psi(t)$ moet wel gelijk zijn aan $[(N_1 - N_2 + 1)/2, (N_2 - N_1 + 1)/2]$.
\end{proof}

\subsection{Daubechieswavelets}
Hoewel de constructie van de Daubechieswavelet buiten het spectrum van dit artikel valt,\footnote{Voor een goede beschrijving van deze constructie, zie \cite{mallat}.} willen we toch een kort licht schijnen op deze speciale familie van wavelets. Deze worden gemaakt met de noties eerder, namelijk dat we de drager willen minimaliseren maar de orde maximaliseren. Daubechies heeft bewezen [TODO ref] dat een filter $h$ met orde $p$, minimaal een drager van lengte $2p$ moet hebben. 

De Daubechieswavelet van orde $p$ nu, heeft precies een filter van lengte $2p$. In het bijzonder is de Haarwavelet de eerste in de familie van Daubechieswavelets.

TODO picca van Daubechies wavelets

\section{Fast Wavelet Transform}
TODO schrijven dit
probeer dit er in te stoppen (als dat niet al duidelijk wordt)?
\[
	V_{-J} = V_0 \oplus W_0 \oplus \cdots \oplus W_{1-J}
\]
$\boldsymbol{\N oem\, ook\, \C onvolutie}$

\begin{equation}
\label{approx}
	a_{j+1}[n] = \sum_{p=-\infty}^\infty h[p-2n] a_j[p] = a_j \star \bar{h}[2n]
\end{equation}
\begin{equation}
\label{detail}
	d_{j+1}[n] = \sum_{p=-\infty}^\infty g[p-2n] a_j[p] = a_j \star \bar{g}[2n]
\end{equation}

\section{Analyse van de Fast Wavelet Transform}
Met de theoretische beschouwing van wavelets en de Fast Wavelet Transform achter de rug, kunnen we wat verder kijken naar practische obstakels.

\subsection{Eindige signalen} 
Een van de eerste aannames die we tot nu toe steeds maakten is die van de oneindige signalen. Wanneer echter de functie $f$ een compacte drager heeft, worden een aantal zaken wat lastiger. Neem als eerste aan dat de drager van $f$ gewoon $[0,1]$ is.\footnote{Door translatie en dilatie kan elk signaal met compacte basis omgevormd worden tot een signaal met drager in $[0,1]$. We verliezen hier dus geen algemeenheid.}

In dit geval kunnen we alle $V_j$ met $j > 0$ buiten beschouwing laten, omdat de resolutie in deze ruimte te laag is om nog interessante informatie over $f$ te bieden. We hebben op deze manier een rij geneste ruimtes
\[
V_0 \subset V_{-1} \subset \ldots.
\]
Wanneer we nu een benadering van $f$ maken op resolutie $2^{-J}$ (door bijvoorbeeld een Fast Wavelet Transform), bekijken we
\[
	V_0 \subset V_{-1} \subset \ldots \subset V_{-J}.
\]
Een discreet signaal van lengte $2^{-J}$ kan zo perfect `benaderd' worden in een waveletbasis op resolutie $2^{-J}$ en dit is precies waarom de Fast Wavelet Transform zo veel gebruikt wordt bij het analyseren van discrete signalen.

\subsection{Signaaluitbreiding}
Een probleem waar we in het geval van eindige signalen nog meer mee te maken krijgen is dat het algoritme niet goed omgaat met de randen. De convolutie moet nu ineens \emph{buiten het definitiegebied} van het signaal `kijken'. Eerder in sectie [TODO] hebben we al gezien hoe signalen naar een tweemacht uitgebreid kunnen worden. Precies dezelfde methoden kunnen gebruikt worden om het signaal nog verder uit te breiden. 

Om niet te veel tijd te verliezen met het ondersteunen van meerdere mogelijkheden hebben wij er voor gekozen om periodic padding op alle signalen toe te passen. Dit omdat de zogenaamde \emph{circulaire convolutie} ingebouwd zit in de bibliotheek die wij gebruikt hebben.

\subsection{Complexiteit van het algoritme}
Als de lengte van de filter $h$ gelijk is aan $K$, en de lengte van het originele signaal $a_L$ gelijk is aan $N = 2^{-L}$, kunnen we voor $j \in \{L, \ldots, 0\}$ zien dat $a_j$ en $d_j$ beide $2^{-j}$ elementen bevatten. Nu kunnen $a_{j+1}$ en $d_{j+1}$ gemaakt worden door $2^{-j}K$ operaties zodat elke stap van het algoritme $2^{-j} \cdot K$ operaties kost. Dan kost het hele algoritme
\[
	\sum_{j=L}^0 2^{-j} \cdot K = K \sum_{j=L}^0 2^{-j} = K \cdot (2^{1-L} - 1) < 2 \cdot K 2^{-L} = 2KN
\]
operaties. Dus deze DWT is een $\mathcal{O}(KN)$ algoritme. Ook de complexiteit van de inverse wordt op dezelfde manier van orde $KN$. 

\section{Meer dimensies}
Met een orthonormale waveletbasis $\{ \psi_{j,n}: (j,n) \in \Z^2\}$ van $L_2(\R)$ volgt een natuurlijke voortzetting naar twee dimensies door
\[
  \{ \psi_{j_1,n_1}(x_1) \psi_{j_2,n_2}(x_2): (j_i,n_i) \in \Z^4 \},
\]
welke een orthonormale basis voor $L_2(\R^2)$ is. We zien direct dat we op de $x_1$-as met resolutie $2^{-j_1}$ kijken terwijl de $x_2$-as resolutie $2^{-j_2}$ kent. Mallat vond dit iets om te vermijden \cite{mallat}[\S 7.7] en legt in zijn analyse dan ook de eis $j_1 = j_2 =: j$ op. Wij zullen zowel de Mallatdecompositie in meer dimensies bekijken als het zogenaamde Tensorproduct, wat de eis $j_1 = j_2$ \emph{niet} oplegt.

\subsection{Mallat} Zoals in 1 dimensie is de notie van `resolutie' geformaliseerd in het begrip multiresolutie. De definitie van deze multiresolutie is wederom een natuurlijke voortzetting van het eendimensionale geval. Op deze manier bekijken we een ruimte $V_j^2 = V_j \otimes V_j$. In \cite[\S 7.7]{mallat} wordt nu bewezen dat, gegeven een orthonormale basis $\{ \phi_{j,m}: m \in \Z \}$ voor $V_j$, de verzameling
\[
	\{ \phi^2_{j,n} := \phi_{j,n_1} \otimes \phi_{j,n_2}: n \in \Z^2 \}
\]
een basis voor $V_j^2$ is.

Neem als voorbeeld weer $V_j$ de ruimte van stuksgewijs constante functies op het interval $[2^j m, 2^j(m+1) )$ voor $m \in \Z$. We vinden voor $V_j^2$ nu de ruimte van vierkantsgewijs constante functies op vierkanten $[2^jn_1, 2^j(n_1+1)) \times [2^jn_2, 2^j(n_2+1))$. De tweedimensionale schalingsfunctie wordt op die manier
\[
	\phi^2(x_1,x_2) = \phi(x_1)\phi(x_2) = \begin{cases} 1 & \text{ als } x_1 \in [0,1)\text{ en }x_2 \in [0,1) \\ 0 & \text{ anders.} \end{cases}
\]

\subsection{Tweedimensionale Waveletfuncties} We weten dat $V_j^2$ bevat is in $V_{j-1}^2$. We bekijken het orthogonale complement $W_j^2 \perp V_j^2$:
\[
	V_j^2 \oplus W_j^2 = V_{j-1}^2.
\]
Om nu een orthogonale waveletbasis voor $W_j$ en (dus) $L^2(\R^2)$ te vinden, gaan we als volgt te werk.
\begin{stelling}[\cite[T7.24]{TODOmallat}]
Laat $\phi$ een schalingsfunctie en $\psi$ de bijhorende wavelet die en basis voor de $L^2(\R)$ voortbrengt. Maak drie wavelets
\[
	\psi^1(x) = \phi(x_1)\psi(x_2) \quad \psi^2(x) = \psi(x_1) \phi(x_2) \quad \psi^3(x) = \psi(x_1)\psi(x_2)
\]
en laat voor $k \in \{1,2,3\}$ nu
\[
	\psi^k_{j,n}(x) = \frac{1}{2^j} \psi^k\left( \frac{x_1 - 2^j n_1}{2^j}, \frac{x_2 - 2^j n_2}{2^j} \right).
\]

Dan is $\{ \psi^1_{j,n}, \psi^2_{j,n}, \psi^3_{j,n}: n \in \Z^2 \}$ een basis voor $W_j^2$ en $\{ \psi^1_{j,n}, \psi^2_{j,n}, \psi^3_{j,n}: n \in \Z^2, j \in \Z \}$ een basis voor $L^2(\R^2)$.
\end{stelling}
\begin{proof}[Bewijs]
We weten
\[
	V_{j-1}^2 = V_j^2 \oplus W_j^2 \implies V_{j-1} \otimes V_{j-1} = ( V_j \otimes V_j ) \oplus W_j^2.
\]
Vul nu $V_{j-1} = V_j \oplus W_j$ in om te vinden
\[
	( V_j \oplus W_j ) \otimes (V_j \oplus W_j ) = (V_j \otimes V_j) \oplus (W_j \otimes W_j) 
\]
\[
	\implies (V_j \otimes V_j) \oplus (V_j \otimes W_j) \oplus (W_j \otimes V_j) \oplus (W_j \otimes W_j) = (V_j \otimes V_j) \oplus W_j^2
\]
\[
	\implies (V_j \otimes W_j) \oplus (W_j \otimes V_j) \oplus (W_j \otimes W_j) = W_j^2.
\]
Nu is het duidelijk dat $\{ \psi^1_{j,n}, \psi^2_{j,n}, \psi^3_{j,n}: n \in \Z^2 \}$ een basis is voor $W_j^2$. Omdat nu
\[
	L^2(\R^2) = \bigoplus_{j=-\infty}^\infty W_j
\]
is $\{ \psi^1_{j,n}, \psi^2_{j,n}, \psi^3_{j,n}: n \in \Z^2, j \in \Z \}$ een basis voor $L^2(\R^2)$.
\end{proof}

Met deze kennis in handen zullen we de tweedimensionale tegenhanger van de Fast Wavelet Transform formuleren.
\begin{algoritme}[Tweedimensionale Fast Wavelet Transform]
Zoals in het eendimensionale geval, laat $j$ vast. Dan:
\[
	a_j[n] := \langle f, \phi^2_{j,n} \rangle \quad d^k_j[n] := \langle f, \psi^k_{j,n} \rangle , k \in \{1,2,3\}.
\]

Vervolgens, gebruikmakend van vergelijkingen $\ref{approx}$ en $\ref{detail}$, zien we dat (TODO):
\[
	a_{j+1}[n] = a_j \star (\bar{h} \otimes \bar{h})[2n];
\]
\[
	d^1_{j+1}[n] = a_j \star (\bar{h} \otimes \bar{g})[2n];
\]
\[
	d^2_{j+1}[n] = a_j \star (\bar{g} \otimes \bar{h})[2n];
\]
\[
	d^3_{j+1}[n] = a_j \star (\bar{g} \otimes \bar{g})[2n],
\]
waarbij
\[
	g \star (x \otimes y)[n_1,n_2] := \sum_{p_1=-\infty}^\infty x[n_1 - p_1] \sum_{p_2 = -\infty}^\infty y[n_2 - p_2] g[p_1, p_2]
\]
een tweedimensionale convolutie defini\"eert.

Andersom geldt dat
\[
a_j[n] = \breve{a}_{j+1} \star (h \otimes h)[n] + \breve{d}_{j+1}^1 \star (h \otimes g)[n] + \breve{d}_{j+1}^2 \star (g \otimes h)[n] + \breve{d}_{j+1}^3 \star (g \otimes g)[n]
\]
\end{algoritme}

TODO: ogier schrijf jij dit af?

Wat we steeds doen is de ruimte $V_j^2$ schrijven als $V_{j+1}^2$ met daarbij een aantal orthogonale complementen. Wanneer dit algoritme recursief toegepast wordt, schrijven we $V_{j+1}^2$ ook weer als $V_{j+2}^2$ met orthogonale complementen en zo voort. Dit levert dat
\[
	\boldsymbol\Phi = \left\{ \psi^k_{j+1,\cdot}, \psi^k_{j+2,\cdot}, \ldots \psi^k_{L,\cdot}, \phi^2_{L,\cdot}: k \in \{1,2,3\} \right\}
\]
een basis is voor $V_j^2$. TODO plaatje. Dit is echter niet de enige basis die we kunnen bouwen voor $V_j^2$ wanneer we recursief tot een niveau $L$ kijken.

\section{Tensorproduct}
Herinner de gelijkheid
\[
	V_{j} = V_{j+1} \oplus W_{j+1}.
\]
Dit kan ook herhaaldelijk toegepast worden:
\[
	V_j = V_{j+1} \oplus W_{j+1} = (V_{j+2} \oplus W_{j+2} ) \oplus W_{j+1} = \ldots = V_{L} \oplus W_L \oplus \cdots \oplus W_{j+1}.
\]
Dus we kunnen nu $V_j^2$ ook anders schrijven:
\[
	V_j^2 = V_j \otimes V_j = (V_{L} \oplus W_L \oplus \cdots \oplus W_{j+1}) \otimes (V_{L} \oplus W_L \oplus \cdots \oplus W_{j+1}).
\]

Voor $V_L$ was $\{ \phi_{L,k}: k \in \Z \}$ een basis. Voor $W_i$ was $\{ \psi_{i,k}: k \in \Z \}$ een basis. Zo krijgen we dat
\[
	\Psi := \{ \phi_{L,\cdot}, \psi_{L,\cdot}, \ldots, \psi_{j+1,\cdot} \}
\]
een basis moet zijn voor $V_j$. Uit \cite[T8.5]{tensor_wavelet} vinden we dat $\boldsymbol\Psi := \{ \psi_1 \otimes \psi_2: \psi_1, \psi_2 \in \Psi \}$ een basis is voor $V_j^2$. Dit kunnen we uitschrijven tot
\[
	\boldsymbol\Psi = \left\{
	\begin{array}{cccc}
		\phi_{L, \cdot} \otimes \phi_{L, \cdot}, & \phi_{L, \cdot} \otimes \psi_{L, \cdot}, & \cdots, & \phi_{L, \cdot} \otimes \psi_{j+1,\cdot}, \\
		\psi_{L, \cdot} \otimes \phi_{L, \cdot}, & \psi_{L, \cdot} \otimes \psi_{L, \cdot}, & \cdots, & \psi_{L, \cdot} \otimes \psi_{j+1,\cdot} ,\\
		\vdots \\
		\psi_{j+1, \cdot} \otimes \psi_{j+1, \cdot}, & \phi_{L, \cdot} \otimes \psi_{L, \cdot}, & \cdots, & \psi_{j+1, \cdot} \otimes \psi_{j+1,\cdot} \\
	\end{array}\right\}.
\]
In onze definitie\footnote{In de literatuur wordt de Mallatdecompositie ook vaak genoeg een Tensorproduct genoemd. Dit is natuurlijk geen verkeerde (hoogstens een verwarrende) naamgeving want de Mallatdecompositie \emph{is} gewoon een Tensorproduct.} is het Tensorproduct nu het schrijven van het (tweedimensionale) signaal in termen van deze basis $\boldsymbol\Psi$. 

\section{Meer dimensies}
Het moge duidelijk zijn hoe het Tensorproduct schaalt naar meer dimensies. De $\boldsymbol\Psi$ moet hier gewoon niet langer twee basisfuncties maar $n$ basisfuncties pakken. Het $n$-dimensionale geval van de Mallatdecompositie is nog TODOOO

\section{Concreet}
Concreet zullen we echter niet gebruik maken van signalen die leven in $L^2(\R^n)$. We zullen eerder op zoek zijn naar de Wavelettransformatie van een signaal met compacte drager. Neem dus aan dat $f$ leeft in $L^2([0,1]^n)$. Dit $n$-dimensionale eenheidsinterval wordt ook wel met $\Box$ aangegeven. Omdat $\Box \subset \R^n$, kunnen we een waveletbasis voor $L^2(\R^n)$ ook als basis nemen voor $L^2(\Box)$. Maar eigenlijk is dit iets te veel (gezien het feit dat veel basisfuncties hun drager geheel buiten het interval zullen hebben). Daarom wordt de verzameling van indices $\lambda := (j,n)$ van wavelets die drager doorsnijden met $[0,1]$ ook wel $\nabla$ genoemd. Dus $\Psi = \{ \psi_\lambda: \lambda \in \nabla \}$ wordt nu een basis voor $L^2([0,1])$. Defini\"eer $|\lambda| = |(j,n)| = j$ als het niveau.
Verder zullen we vanaf nu aannemen dat $\boldsymbol\psi$ een compacte drager heeft (zoals we in de praktijk altijd willen) en van orde $p$ is.


\section{Analyse van de fout van een lineaire benadering}
Bij compressie is men op zoek naar een manier om stukjes data weg te kunnen gooien of te schrijven op zo'n manier dat het minder ruimte inneemt. Wij zijn in het bijzonder ge\"interesseerd naar \emph{hoe dichtbij} we bij perfecte reconstructie zitten wanneer we een vooraf bepaald aantal data opslaan. Er is al uitgebreid onderzoek gedaan naar hoe dit werkt bij de waveletbasis en een aantal resultaten hiervan zullen we opnemen in ons verslag.

Laat $f$ een functie in $L^2(\Box)$ met aftelbare basis $\mathcal{B} = \{ g_m \}$. Dan valt $f$ in deze basis te schrijven als
\[
	f = \sum_{m = 0}^\infty \langle f, g_m \rangle g_m.
\]

\begin{lemma}
Als nu een functie $f$ in $L^2(\Box)$ geschreven wordt in $\mathcal{B}$ dan geldt
\[
	\|f\|^2 = \sum_{m=0}^\infty | \langle f, g_m \rangle |^2.
\]
\end{lemma}
\begin{proof}[Bewijs]
We hebben te maken met een Hilbertruimte dus
\[
\|f\|^2 = \langle f, f \rangle = \left\langle \sum_{m=0}^\infty \langle f, g_m \rangle g_m, \sum_{n=0}^\infty \langle f, g_n \rangle g_n \right\rangle = \sum_{m=0}^\infty \sum_{n=0}^\infty \left\langle \langle f, g_m \rangle g_m, \langle f, g_n \rangle g_n \right \rangle
\]
\[
 = \sum_{m=0}^\infty \sum_{n=0}^\infty \langle f, g_m \rangle \overline{\langle f, g_n \rangle}\langle g_m, g_n \rangle = \sum_{m=0}^\infty \sum_{n=0}^\infty \langle f, g_m \rangle \overline{\langle f, g_n \rangle} \delta_{m,n} 
\]
\[ = \sum_{m=0}^\infty \langle f, g_m \rangle \overline{\langle f, g_m \rangle} = \sum_{m=0}^\infty |\langle f, g_m \rangle |^2.
\]
\end{proof}

Wanneer we nu niet de hele basis, maar zeg alleen de eerste $N$ elementen pakken, krijgen we een verzameling $\mathcal{B}_N \subset \mathcal{B}$ zodat
\[
	f_{\mathcal{B}_N} := \sum_{m = 0}^{N-1} \langle f, g_m \rangle g_m.
\]

In het bijzonder zijn we nu op zoek naar de \emph{fout} $\| f - f_{\mathcal{B}_N} \|$:
\[
	\| f - f_{\mathcal{B}_N} \|^2 = \left\| \sum_{m=0}^\infty\langle f, g_m \rangle g_m - \sum_{m=0}^{N-1}\langle f, g_m \rangle g_m \right\|^2 = \left\| \sum_{m=N}^\infty\langle f, g_m \rangle g_m \right\|^2 = \sum_{m=N}^\infty | \langle f, g_m \rangle |^2.
\]
Duidelijk moge zijn dat voor $N \to \infty$, $\| f - f_{\mathcal{B}_N} \|^2 \to 0$.

\subsection{Mallat}
Wij zijn op het moment ge\"interesseerd in de Mallat-waveletbasis $\boldsymbol\Phi$. Deze is ook duidelijk aftelbaar dus bovenstaande eigenschap geldt.

Defini\"eer $J_M := \{ l \in \N^n: \| l \|_\infty \leq M \}$. Laat $\boldsymbol\Phi_M := \{ \boldsymbol{\psi}_{\boldsymbol{\lambda}} \in \boldsymbol\Phi: |\boldsymbol\lambda| \in J_M \}$ met $|\boldsymbol\lambda| = (|\lambda_1, \ldots, |\lambda_n|)$ de verzameling basisfuncties tot een niveau $M$.

\begin{stelling}[Fout van Mallatdecompositie]
Wanneer $f \in H^p(\Box)$, zal de fout $\| f - f_{\mathcal{\boldsymbol\Phi}_M} \|$ bij een Mallatdecompositie met de basisfuncties tot niveau $M$ hoogstens proportioneel met $(\# \boldsymbol\Phi_M)^{-p/n}$ zijn.
\end{stelling}
\begin{proof}

We maken gebruik van de zogenaamde Jacksonongelijkheid \cite{jackson} die zegt dat 
\[
  \inf_{q \in \mathbb{P}_{p-1}} ||f - q||_{L_2(\Box)} \simeq 2^{-jp} ||f||_{H^p(\Box)}
\]
wanneer $f \in H^p(\Box)$.

Voor elke dimensie zitten er $\mathcal{O}(2^M)$ basisfuncties in $\{ \psi_\lambda: |\lambda| \leq M \}$. Er zijn $n$ dimensies dus $2^{Mn} = N (= \# \boldsymbol\Phi)$ basisfuncties in totaal.

Bekijk de fout:
\[
	\left\| f - f_{\boldsymbol\Phi_M} \right\|^2_{L_2(\Box)} = \sum_{{\boldsymbol\psi} \in \boldsymbol\Phi \setminus \boldsymbol\Phi_M} | \langle f, \boldsymbol\psi \rangle |^2 \simeq \sum_{|\boldsymbol\lambda| > M} 2^{-|\boldsymbol\lambda|p} \simeq 2^{-Mp},
\]
waarbij het laatste isteken voortkomt uit
\[
	\sum_{k=M+1}^\infty 2^{- kp} = \frac{2^{-Mp}}{2^p-1}
\]
en de notie dat $p$ constant is voor een keuze van de wavelet. De fout is nu $2^{-Mp/2} \simeq 2^{-Mp}$. Omschrijven geeft dat dit overeenkomt met een fout van $N^{-p/n}$.
\end{proof}

\subsection{Tensorproduct}
Volgens \cite[L3.1.7]{tammo} is 
\[ 
  \boldsymbol\Psi = \Psi \otimes \cdots \otimes \Psi = \{ \boldsymbol{\psi_\lambda} := \psi_{\lambda_1} \otimes \cdots \otimes \psi_{\lambda_n}: \lambda_i \in \nabla \}
\]
met $\boldsymbol\lambda := (\lambda_1, \ldots, \lambda_n) \in \boldsymbol{\nabla} := \nabla^n$ nu een orthogonale basis voor $L^2(\Box)$.

Laat vervolgens $I_M := \{ l \in \N^n_0: ||l||_1 \leq M \}$ en maak de \emph{sparse grid index set} $\boldsymbol{\nabla}_M := \{ \boldsymbol{(j,n)} \in \boldsymbol{\nabla}: \boldsymbol{j} \in I_M \}$.

\begin{lemma}{\cite[P3.2.3]{tammo}}
  Voor $f \in H^p(\Box)$ geldt dat de fout van de benadering op basis van de sparse grid index set $\boldsymbol{\nabla}_M$ hoogstens voldoet aan
\[
  \left\| f - f_{\boldsymbol\nabla_M} \right\|_{L_2(\Box)} \lesssim 2^{-pM} M^{(n-1)/2} \| f \|_{H^p(\Box)}
\]
\end{lemma}

Het aantal elementen in deze verzameling $\boldsymbol{\nabla}_M$ nu, kunnen we vinden.
\begin{lemma}{\cite[L3.3.1]{tammo}}
  Het aantal elementen in $\boldsymbol{\nabla}_M$ is proportioneel met $2^M M^{n-1}$.
\end{lemma}

\begin{lemma}
  Wanneer er voor twee functies $f, g$ geldt dat $f(J) \lesssim J^{-p}\log_2(J)^\mu$ en $g(J) = \log_2(J)^\nu J =: N$, dan
  \[
    (f \circ g^{-1})(N) \lesssim N^{-p} \log_2{N}^{\mu + \nu p}.
  \]
  TODO: bewijs klopt (nog) niet
\end{lemma}
\iffalse
\begin{proof}[Bewijs]
  We weten dat $g(J) = N$ dus $g^{-1}(N) = J$. Dan
  \[
    (f \circ g^{-1})(N) = f(J) \simeq J^{-p}\log_2(J)^{\mu}.
  \]
  Omdat verder $J \log_2(J)^\nu = N$, geldt $J^{-1} = N^{-1}\log_2(J)^\nu$. Vul dit in in bovenstaande om te krijgen
  \[
    f(J) \simeq N^{-s} \log_2(J)^{\nu s} \log_2(J)^\mu = N^{-s} \log_2(J)^{\nu s + \mu}????
  \]
\end{proof}
\fi

Met bovenstaande drie lemma's is het nu mogelijk een goede afschatting te maken. 
\begin{stelling}[Fout van het Tensorproduct]
  Laat $f \in H^p(\Box)$ en $N = \#\boldsymbol{\nabla}_M$. Dan:
  \[
    \left\| f - f_{\boldsymbol\nabla_M} \right\|_{L_2(\Box)} \lesssim N^{-p} \log_2(N)^{(n-1)(1/2 + p)} \| f \|_{H^p(\Box)}.
  \]
\end{stelling}
\begin{proof}
  Gebruik het tweede lemma om te vinden dat $N \simeq M^{n-1}2^M$. Nu vinden we via het eerste lemma dat
  \[
    \left\| f - f_{\boldsymbol\nabla_M} \right\|_{L_2(\Box)} \lesssim M^{(n-1)/2}2^{-Mp}\| f \|_{H^p(\Box)}
  \]
  zodat
  \[
    \frac{\left\| f - f_{\boldsymbol\nabla_M}  \right\|_{L_2(\Box)}}{\| f \|_{H^p(\Box)}} \lesssim M^{(n-1)/2}2^{-Mp}.
  \]

  We willen graag lemma drie toepassen. Door
  \[
    J^{-p}\log_2(J)^\mu = M^{(n-1)/2} 2^{-Mp}
  \]
  volgt $J = 2^m$ en $\mu = (n-1)/2$. Door
  \[
    N = M^{n-1}2^M = \log_2(J)^\nu J = 2^M M^\nu
  \] volgt $\nu = n-1$. TODO: wat is $f$ en $g$?
\end{proof}

We vinden dus dat voor een voldoend gladde $f$ dat (gebruikend een beperkte hoeveelheid basisfuncties) de Mallatdecompositie hoogstens een convergentiesnelheid $N^{-p/n}$ bereikt, terwijl het Tensorproduct theoretisch een snelheid $N^{-p} \log_2(N)^{(n-1)(p+1/2)}$ bereikt. Let op dat voor $n=1$ de twee convergentiesnelheden beide $N^{-p}$ zijn (omdat beide bases dan gewoon hetzelfde zijn).

In de praktijk hebben we echter nooit te maken met compleet gladde functies. Gevolg is dat deze stellingen niet helemaal opgaan. Niet \emph{helemaal}, omdat per constructie van onze wavelet, de basis compact is en dus `zoomen we in' op de functie. Lokaal is de mogelijkheid dat $f$ glad genoeg is ineens een stuk meer in zicht. Het gevolg is wel dat je op zo'n moment waarschijnlijk lokaal een hoger niveau wil gebruiken. De bewijzen van hierboven zijn op basis van een \emph{niet-adaptieve} deelverzameling, dat wil zeggen dat ze geen rekening houden met lokaal een hoger niveau.

Hoewel de vorige zin misschien klinkt alsof er roet in ons eten gegooid wordt, is het in de praktijk toch goed mogelijk om de gevolgen te zien. Dit zullen we zien wanneer we de twee decomposities zullen gaan vergelijken.

\begin{thebibliography}{9}

\bibitem{tammo}
Tammo Jan Dijkstra,
\emph{Adaptive tensor product wavelet methods for solving PDEs}, 2009
\bibitem{tensor_wavelet}
http://www.uio.no/studier/emner/matnat/math/MAT-INF2360/v12/tensorwavelet.pdf
\bibitem{mallat}
St\'ephane Mallat,
\emph{A Wavelet Tour of Signal Processing}
\bibitem{jackson}
http://www.ams.org/journals/bull/1960-66-02/S0002-9904-1960-10426-0/S0002-9904-1960-10426-0.pdf
\end{thebibliography}
\end{document}


  \iffalse
Bekijk $f$ even als een functie in 1 variabele. Het geval van meer variabelen is een stuk technischer maar niet interessanter.
Als gevolg van de orde $p$ van de wavelet vinden we dat we $|\langle f, \psi_{j,n}\rangle |$ als volgt om kunnen schrijven:
\[
	|\langle f, \psi_{j,n} \rangle | = |\langle f-q, \psi_{j,n} \rangle | \text{ voor elk polynoom $q$ van graad $p-1$ of lager,} 
\]
zodat
\[
	|\langle f-q, \psi_{j,n} \rangle | \leq ||f-q|| \cdot ||\psi_{j,n}||.
\]

De norm van deze functies $f-q$ en $\psi_{j,n}$ is de $L_2$-norm, welke bepaald wordt door een integraal over $[0,1]$. Omdat $\psi_{j,n}$ een compacte drager $S_{j,n}$ heeft, is het laatste product net zo goed te beschouwen als integraal over $S_{j,n}$:
\[
	||f-q|| \cdot ||\psi_{j,n}|| = ||f-q||_{S_{j,n}} \cdot ||\psi_{j,n}||_{S_{j,n}} = ||f-q||_{S_{j,n}},
\]
omdat $\psi_{j,n}$ per constructie norm 1 heeft.

Lokaal kunnen we $f$ echter redelijk goed benaderen door middel van een polynoom van graad $p-1$, namelijk de eerste $p$ termen van de Taylorontwikkeling. Per definitie van deze Taylorontwikkeling weten we nu dat de restterm, $f(x)-q(x) \simeq h^p||f||_{S_{j,n}}$ met $h$ de diameter van $S$.\cite[\S 31.\{3,4\}]{TODOross} Met andere woorden, $||f-q||_{S_{j,n}} \simeq h^p ||f||_{S_{j,n}}$. Omdat deze drager gewoon een lijnstukje is met lengte $2^{-j}$ is, wordt 

\[
	||f-q||_{S_{j,n}} \simeq 2^{-jp} ||f||_{S_{j,n}}.
\]

Het bewijs tot nu toe is simpel voort te zetten naar een of andere dimensie $n$. Het enige wat echt verandert is de diameter. De drager $S_{j,n}$ wordt nu geen lijnstukje maar een kubus met zijdes $2^{-j}$ zodat geldt voor de diameter $h$:
\[
	h = \sqrt{\sum_{i=1}^n (2^{-j})^2} = \sqrt{n 2^{-2j}} = \sqrt{n} 2^{-j}.
\]


Uiteindelijk krijgen we dat $||f-q||_{S_{j,n}} \simeq n^{p/2} 2^{-jp} ||f||_{S{j,n}}$.

Men krijgt nu de beste benadering door alle ${j,n}$ met $|{j,n}|$ tot een bepaald niveau, zeg $M$, te pakken. Bekijk de fout:
\[
	\left\| f - \sum_{|{j,n}| \leq M} \langle f, \psi_{j,n} \rangle \psi_{j,n} \right\|_\Box = \sum_{|{j,n}| > M} | \langle f, \psi_{j,n} \rangle |^2 \in \mathcal{O}\left(\sum_{|{j,n}| > M} n^{p/2} 2^{-|{j,n}|p} \right) = \mathcal{O}(n^{p/2}2^{-Mp}),
\]
waarbij het laatste isteken voortkomt uit
\[
	\sum_{k=M+1}^\infty n^{p/2} \, 2^{- kp} = n^{p/2} \frac{2^{-Mp}}{2^p-1}
\]
en de notie dat $p$ constant is voor een keuze van de wavelet.

Voor elke dimensie van de ruimte zitten er $\mathcal{O}(2^M)$ basisfuncties in ``alle ${j,n}$ met $|{j,n}| \leq M$''. We hebben $n$ dimensies, dus er zijn $2^{Mn} = K $ wavelets die zorgen voor een fout van $\mathcal{O}(n^{p/2} 2^{-Mp})$. Omschrijven geeft dat dit overeenkomt met een fout van $\mathcal{O}(n^{p/2} K^{-p/n})$.
\fi
