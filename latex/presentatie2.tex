\documentclass{beamer}
\usepackage{beamerthemeAmsterdam}
\usepackage{amsmath}
\usepackage{graphicx}

\newcommand{\lijstje}[1]{\begin{itemize} #1 \end{itemize}}

\title{Jpeg Compressie}
\subtitle{De Wondere Wereld van Wavelets}
\author{Jan Westerdiep \and Okke van Garderen}
\date{\today}
\institute{Universiteit van Amsterdam}

\renewcommand{\figurename}{}

\begin{document}

\frame{\titlepage}

\section{Recap}

\frame{\frametitle{Welk project bij wie?}
  \lijstje{
  \item robstevensson.jpg
  \item JPEG compressie
  \item Fouriertransformatie
  }
}

\section{Theoretische beschouwing}
\frame{\frametitle{DFT}
  \lijstje{
    \item uitleg van het algo
  }
}
\frame{\frametitle{Uitleg}
  Keuze uit:
  \lijstje{
    \item is de inverse DFT ook echt een inverse
    \item voorbeeldje geven
  }
}
\frame{\frametitle{FFT}
  \lijstje{
    \item uitleg van dit algo
  }
}
\frame{\frametitle{snelheid/bewijs}
  Keuze uit:
  \lijstje{
    \item Werkt FFT nou echt sneller?
    \item Werkt FFT uberhaupt als snelle DFT?
  }
}

\section{Resultaten}
\frame{\frametitle{1 Kanaals - 1D Geluid}
  \lijstje{ 
  \item[?] spongebob.wav
  \item[?] pokemon.wav
  }
}
\frame{\frametitle{1 Kanaals - 2D plaatjes}
  \lijstje{
  \item[?] even een soort van plaatje maken
  }
}
\frame{\frametitle{4 Kanaals - 2D plaatjes}
  \lijstje{
  \item installgentoo.jpg
  }
}

\section{Toekomst}
\frame{\frametitle{Toekomst}
  \lijstje{
  \item Overschakelen naar Wavelets
  \item 3D / Filmpjes
  }
}

\section{Slides voor Chris}
\frame{
  \lijstje{
  \item Volgende slides geven antwoord op Chris zn vragen
  }
}

\end{document}
