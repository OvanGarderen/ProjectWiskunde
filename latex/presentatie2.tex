\documentclass{beamer}
\usepackage{beamerthemeAmsterdam}
\usepackage{amsmath}
\usepackage{graphicx}
\usepackage{algpseudocode}
\usepackage{algorithmicx}

\iffalse
idee: alle slides uitwerken/kennen, en dan gewoon aan de klas vragen wat ze willen?
\fi

\newcommand{\lijstje}[1]{\begin{itemize} #1 \end{itemize}}

\title{Jpeg Compressie}
\subtitle{De Wondere Wereld van Wavelets}
\author{Jan Westerdiep \and Okke van Garderen}
\date{\today}
\institute{Universiteit van Amsterdam}

\renewcommand{\figurename}{}

\begin{document}

\frame{\titlepage}

\section{Recap}

\frame{\frametitle{Welk project bij wie?}
  \lijstje{
  \item robstevensson.jpg
  \item JPEG compressie
  \item Fouriertransformatie
  }
}

\section{Theoretische beschouwing}
\frame{\frametitle{DFT}
  \lijstje{
  \item $x_n$ is een lijst meetwaarden van lengte $N$
  \item De DFT hiervan geeft een nieuwe lijst $X_k$:
    \[
      X_k = \sum_{n=0}^{N-1} x_n \cdot e^{-i 2 \pi k n / N}
    \]
  \item $X_k$ is nu te beschouwen als de co\"effici\"ent van $k$'de basiselement
  \item inverse DFT:
    \[
      x_n = \frac{1}{N}\sum_{k=0}^{N-1} X_k \cdot e^{i 2 \pi k n / N}
    \]
  }
}
\iffalse
\frame{\frametitle{Uitleg}
  Keuze uit:
  \lijstje{
    \item is de inverse DFT ook echt een inverse
    \item voorbeeldje geven
  }
}
\fi
\frame{\frametitle{Voorbeeld}
  \lijstje{
  \item Stel $x = (0, 1, 0, -1)$
  \begin{align*}
    X_0 &= \sum x_n e^{-2 \pi i\cdot 0 \cdot n/4} = \sum x_n \cdot 1 = \sum x_n = 0 \\
    X_1 &= \sum x_n e^{-2 \pi i\cdot 1 \cdot n/4} = x_0 - i x_1 - x_2 + i x_3 = -\frac{1}{2}i \\
    X_2 &= \sum x_n e^{-2 \pi i\cdot 2 \cdot n/4} = x_0 - x_1 + x_2 - x_3 = 0 \\
    X_3 &= \sum x_n e^{-2 \pi i\cdot 3 \cdot n/4} = x_0 + i x_1 - x_2 - i x_3 = \frac{1}{2} i
  \end{align*}
  }
}
\frame{\frametitle{Voorbeeld}

  \lijstje{
  \item $x = (0,1,0,-1) \Rightarrow X = (0,-\frac{1}{2} i, 0, \frac{1}{2}i)$
  \item Hadden we dit ook anders kunnen zien?
  \[
    \sin(\pi x /2) \text{ op de punten } (0, 1, 2, 3)
  \]
  \[
    \sin( \pi x /2) = \frac{ e^{i \frac{1}{2} \pi x} - e^{- i \frac{1}{2} \pi x} }{2 i} = -\frac{1}{2}i e^{\frac{1}{2} i \pi x} + \frac{1}{2} i e^{-\frac{1}{2} i \pi x}
  \]
  \[
    = 0 + X_1 e^{\frac{1}{2} i \pi x} + 0 + X_3 e^{\frac{3}{2} \pi x} = \sum_{k=0}^{3} X_k e^{2 \pi i k x/4}!
  \]
  }
}
\frame{\frametitle{FFT}
  \small{
  \begin{algorithmic}
  \Function{FFT}{$x$}
  \State $n \gets \text{lengte}(x)$ \Comment Assumptie: $n$ is een tweemacht
  \If {$n == 1$}
          \State{$X \gets x$}
  \Else
          \State $E \gets FFT(x[0::2])$ \Comment{Pak alle even indices}
          \State $O \gets FFT(x[1::2])$ \Comment{Pak alle oneven indices}
          \For{$i = 0$ to $n-1$}
                  \If{$i < n/2$}
                          \State $X[i] \gets E[i] + e^{-2i \pi k/n} \cdot O[i]$
                  \Else
                          \State $X[i] \gets E[i] - e^{-2i \pi k/n} \cdot O[i]$
                  \EndIf
          \EndFor
  \EndIf
  \State \Return{$X$}
  \EndFunction
  \end{algorithmic}
}
}
\frame{\frametitle{snelheid/bewijs}
  Keuze uit:
  \lijstje{
    \item Werkt FFT nou echt sneller?
    \item Werkt FFT uberhaupt als snelle DFT?
  }
}

\section{Resultaten}
\frame{\frametitle{1 Kanaals - 1D Geluid}
  \lijstje{ 
  \item[?] spongebob.wav
  \item[?] pokemon.wav
  }
}
\frame{\frametitle{1 Kanaals - 2D plaatjes}
  \lijstje{
  \item[?] even een soort van plaatje maken
  }
}
\frame{\frametitle{4 Kanaals - 2D plaatjes}
  \lijstje{
  \item installgentoo.jpg
  }
}

\section{Toekomst}
\frame{\frametitle{Toekomst}
  \lijstje{
  \item Overschakelen naar Wavelets
  \item 3D / Filmpjes
  }
}

\section{Slides voor Chris}
\frame{
  \lijstje{
  \item Volgende slides geven antwoord op Chris zn vragen
  }
}

\end{document}
