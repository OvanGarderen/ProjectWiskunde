\documentclass{beamer}
\usepackage{beamerthemeAmsterdam}
\usepackage{amsmath}
\usepackage{graphicx}

\renewcommand{\phi}{\varphi}

\title{JPEG-2000}
\subtitle{De Wondere Wereld van Wavelets}
\author{Jan Westerdiep \and Okke van Garderen}
\date{\today}
\institute{Universiteit van Amsterdam}

\renewcommand{\figurename}{}

\begin{document}

\frame{\titlepage}

\section{Theorie}

\frame{
\frametitle{Fourier Transformatie}
\begin{columns}
\begin{column}{0.75\linewidth}
\[ \left\{ e^{i k 2 \pi x}: k \in \mathbb{N}_0 \right\} \]
\[ f = \sum \langle f, \phi_k \rangle \phi_k\text{, met $\phi_k$ genormaliseerd } \]

\begin{itemize}
    \item (2d) signaal als functie te beschouwen
    \item deze functie schrijven in de Fourier-basis
    \item co\"efficienten kleiner dan $\epsilon$``weggooien"
    \item signaal reconstrueren adhv kleinere set co\"efficienten
    \item praktisch probleem!!
\end{itemize}
\end{column}
\begin{column}{0.25\linewidth}
    \includegraphics[width=\linewidth]{Sample2.jpg}
\end{column}
\end{columns}
}

\frame{
\frametitle{Wavelet Transformatie}
\begin{columns}
\begin{column}{0.75\linewidth}
  \includegraphics[width=\linewidth]{wavelets.jpg}
\end{column}
\begin{column}{0.25\linewidth}
\vspace{4.55pt}
\includegraphics[width=\linewidth]{Sample1.jpg}
\end{column}
\end{columns}
}

\section{Begeleider}

\frame{
\frametitle{Begeleider: Rob Stevensson}
\begin{columns}
  \begin{column}{0.5\linewidth}
    \begin{itemize}
    \item Was numerieke analyse docent
    \item De helft van zijn publicaties heeeft `Wavelet' in de naam
    \item Numerieke analayse $\rightarrow$ benaderen van functies
    \item Verbinding met informatica
    \end{itemize}
  \end{column}
  \begin{column}{0.5\linewidth}
    \includegraphics[width=\linewidth]{robbierobrob.jpg}
  \end{column}
\end{columns}
}

\section{Project}

\frame{
\frametitle{Motivatie en Doelen}
Motivatie
\begin{itemize}
  \item Analytisch
  \item Programmeren :D
\end{itemize}
Doelen
\begin{itemize}
  \item Zelf iets bedenken (nieuwe wavelet, algoritme)
  \item Rob stelde 3D signalen voor
  \item Fourier en Wavelet transformaties implementeren
  \item Orienterende fase
\end{itemize}
}

\frame{
\frametitle{Afspraken}
\begin{itemize}
\item Elke dinsdagochtend om 11 uur werken aan het project
\item Afspreken met Rob wanneer nodig
\item Ori\"enteren en programmeren
\end{itemize}
}

\frame{
\includegraphics[height=\linewidth]{JPEGthestrongest.jpg}
}

\end{document}
