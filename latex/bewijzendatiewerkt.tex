\documentclass[11pt]{article}

\usepackage{a4wide}
\usepackage{amsmath}

\newcommand{\R}{\mathbb{R}}
\newcommand{\N}{\mathbb{N}}
\newcommand{\Z}{\mathbb{Z}}
\newcommand{\C}{\mathbb{C}}
\newcommand{\A}{\mathbb{A}}
\newcommand{\Q}{\mathbb{Q}}
\newcommand{\F}{\mathbb{F}}
\newcommand{\e}{\epsilon}

\newcommand{\eq}[1]{\begin{eqnarray*} #1 \end{eqnarray*}}
\newcommand{\mogelijkheden}[1]{\begin{cases} #1 \end{cases}}

\setlength\parindent{0pt}

\begin{document}

De discrete fouriertransformatie wordt gegeven door:
\eq{
 X_k = \sum^N_{n=1} x_n \cdot e^{-i2\pi k n/N} \quad k\in N
}
voor een lijst getallen van lengte $N$, we gaan controleren dat 
dit identiek is aan ons algoritme.

Ons algoritme wordt gegeven door:

Als N = 1.
\[
  X_k = x_k
\]

Anders is $N$ een macht van $2$ en splitsen we de lijst $x$ op in de lijsten:
\eq{
  \e_k   =& x_{2k}  \quad k < N/2\\
  o_k =& x_{2k+1} \quad k < N/2
}
Waarbij $e$ dus de lijst is met even indices en $o$ de lijst met oneven indices.

En voeren we hierop de discrete Fourier transformatie uit, zodat we de getransformeerde lijsten
$E$ en $O$ vinden, deze lijsten hebben dus beiden lengte $N/2$.
De volledige discrete Fourier transformatie wordt dan gegeven door:
\[
  X_k = \left\{\begin{array}{llll}
    E_k         &+& e^{-i2\pi k/N} O_k &  k< N/2 \\
    E_{(k-N/2)} &-& e^{-i2\pi k/N} O_{(k-N/2)} &  k\geq N/2 
  \end{array}\right.
\]

We voeren nu een inductiebewijs naar $N$ als macht van $2$ uit:

(basisstap) Als $N=2^0=1$ dan geldt:
\eq{
  X_k = x_k = x_1 = \sum_{n=1}^{1} x_n \cdot e^{-i2\pi} = \sum_{n=1}^{N} x_n \cdot e^{-i2\pi k n /N}
}

(inductiestap) Als het algoritme geldig is voor een lijst van $N/2 = 2^{m-1}$ dan hebben we voor $N$:
\eq{
  X_k = \left\{\begin{array}{llll}
    E_k         &+& e^{-i2\pi k/N} O_k &  k< N/2 \\
    E_{(k-N/2)} &-& e^{-i2\pi k/N} O_{(k-N/2)} &  k\geq N/2 
  \end{array}\right.
}
We kunnen vanwege de inductieaanname de directe vorm van $E_k$ en $O_k$ invullen, deze lijsten
hebben immers lengte $N/2$:
\eq{
  X_k = \left\{\begin{array}{llll}
    \sum^{N/2}_{n=1} \e_n \cdot e^{-i2\pi k 2n/N} &+& 
    e^{-i2\pi k/N} 
    \sum^{N/2}_{n=1} o_n \cdot e^{-i2\pi k 2n/N} &  k< N/2 \\
    \sum^{N/2}_{n=1} \e_n \cdot e^{-i2\pi (k-N/2) 2n/N} &-& 
    e^{-i2\pi k/N} 
    \sum^{N/2}_{n=1} o_n \cdot e^{-i2\pi (k-N/2) 2n/N} &  k\geq N/2 
  \end{array}\right.
}
We merken op dat 
\eq{
  e^{-i2\pi(k-N/2)2n/N} = e^{-i2\pi k 2n/N}\cdot e^{-i2\pi n} = - e^{-i2\pi k 2n/N}
} 
waardoor het gevalsonderscheid wegvalt, aangezien beide vergelijkingen identiek zijn.

Dan hebben nu $X_k$ als sommatie over de lijsten $e$ en $o$, we vullen de relatie voor $e,o$ met $x$ in ennemen de factor voor de oneven indices mee in de sommatie.
\eq{
  X_k = \sum^{N/2}_{n=1} x_{2n} \cdot e^{-i2\pi k (2n)/N} + 
    \sum^{N/2}_{n=1} x_{2n+1} \cdot e^{-i2\pi k (2n+1)/N} 
    = \sum^N_{n=1} x_n \cdot e^{-i2\pi k n/N}
}
\end{document}

